\documentclass[phd,oneside,british]{ThesisPUC_uk}

\usepackage[utf8]{inputenc}
\usepackage{amsmath}
\usepackage{amssymb}
\usepackage{url}
\usepackage{color}
\usepackage{multirow}

\newcommand{\Value}{\mathbf{value}}
\newcommand{\Any}{\mathbf{any}}
\newcommand{\Nil}{\mathbf{nil}}
\newcommand{\Self}{\mathbf{self}}
\newcommand{\False}{\mathbf{false}}
\newcommand{\True}{\mathbf{true}}
\newcommand{\Boolean}{\mathbf{boolean}}
\newcommand{\Integer}{\mathbf{integer}}
\newcommand{\Number}{\mathbf{number}}
\newcommand{\String}{\mathbf{string}}
\newcommand{\Void}{\mathbf{void}}
\newcommand{\Const}{\mathbf{const}}

\newcommand{\mylabel}[1]{\; (\textsc{#1})}
\newcommand{\env}{\Gamma}
\newcommand{\penv}{\Pi}
\newcommand{\senv}{\Sigma}
\newcommand{\subtype}{<:}
\newcommand{\ret}{\rho}
\newcommand{\self}{\sigma}

\newcommand{\minitab}[2][c]{\begin{tabular}{#1}#2\end{tabular}}

\def\dstart{\hbox to \hsize{\vrule depth 4pt\hrulefill\vrule depth 4pt}}
\def\dend{\hbox to \hsize{\vrule height 4pt\hrulefill\vrule height 4pt}}

\author{André Murbach Maidl}
\authorR{Maidl, André Murbach}
\adviser{Roberto Ierusalimschy}
\adviserR{Ierusalimschy, Roberto}
\coadviser{Fabio Mascarenhas de Queiroz}
\coadviserR{Queiroz, Fabio Mascarenhas de}
\coadviserInst{UFRJ}

\title{Typed Lua: An Optional Type System for Lua}
\titlebr{Typed Lua: um sistema de tipos opcional para Lua}

\day{10th} \month{April} \year{2015}

\city{Rio de Janeiro}
\CDD{004}
\department{Informática}
\program{Informática}
\school{Centro Técnico Científico}
\university{Pontifícia Universidade Católica do Rio de Janeiro}
\uni{PUC--Rio}

\jury {
\jurymember{Ana Lúcia de Moura}
           {Departamento de Informática --- PUC-Rio}
\jurymember{Edward Hermann Haeusler}
           {Departamento de Informática --- PUC-Rio}
\jurymember{Anamaria Martins Moreira}
           {UFRJ}
\jurymember{Roberto da Silva Bigonha}
           {UFMG}
\schoolhead{José Eugênio Leal}
}

\resume{
The author graduated in Computer Science from Pontifícia Universidade
Católica do Paraná --- PUCPR in 2004, and obtained the degree of Mestre at
Universidade Federal do Paraná --- UFPR in 2007.
He obtained the degree of Doutor at PUC-Rio in 2015,
where he worked in the field of programming languages.
}

\acknowledgment{
First of all I would like to thank three indispensable people in my life:
my wife, my mother, and my father.
Izabella is my inexhaustible source of inspiration, and she is
always pushing me forward.
My mother is a very brave person, and she inspires me to never give up.
My father is a very kind person, and he inspires me to try being more sensitive.
I am thankful to them not only because the support they gave me while
writing this thesis, but also because they are the reason of my life.

This work would have not happened without all the help from my advisors.
Roberto has a singular view of computer science that pushed me to
pursuit the best I could do, while Fabio gave me the confidence I
needed about the work we were doing.
I am glad that both of them had patience to guide me through this
work, and for all the nice meetings we had that led me to learn
a lot of new concepts and ideas.

It is hard to complete a thesis without having some leisure moments,
so I thank my friend and roommate André Ramiro for all the crazy and
fun moments that we had, which helped me to keep focused during working hours.

I also thank my friends at LabLua for the nice work environment,
specially Hisham and Pablo for all the support during the
hardest moments.

It may look easy moving to Rio, as it is a sunny place with lots
of things to do, but it may not be as easy as it looks for
someone that grew up in a cloudy and rainy town.
Thus, I thank my cousin Rebecca and my aunt Nazira for all the help
and affection they gave me during my adaptation in Rio, which
were fundamental to keep my goals in mind.

I thank the guys from Rio de Janeiro Fixed Gear (RJFG) for all
the fun we had during the night rides, and for helping me to discover Rio.

I am also most grateful to the professors Ana Lúcia de Moura,
Anamaria Martins Moreira, Edward Hermann Haeusler, and
Roberto da Silva Bigonha for their careful review that helped
to improve this work.

Finally, I would like to thank CAPES, Google Summer of Code, and PUC-Rio
for partially funding this work.

}

\abstract{
Dynamically typed languages such as Lua avoid static types in favor of
simplicity and flexibility, because the absence of static types means
that programmers do not need to bother with abstracting types that
should be validated by a type checker.
In contrast, statically typed languages provide the early detection of
many bugs, and a better framework for structuring large programs.
These are two advantages of static typing that may lead programmers
to migrate from a dynamically typed to a statically typed language,
when their simple scripts evolve into complex programs.
Optional type systems allow combining dynamic and static typing in
the same language, without affecting its original semantics,
making easier this code evolution from dynamic to static typing.
Designing an optional type system for a dynamically typed language
is challenging, as it should feel natural to programmers that are
already familiar with this language.
In this work we present and formalize the design of Typed Lua,
an optional type system for Lua that introduces novel features
to statically type check some Lua idioms and features.
Even though Lua shares several characteristics with other dynamically
typed languages such as JavaScript, Lua also has several unusual features
that are not present in the type system of these languages.
These features include functions with flexible arity, multiple assignment,
functions that are overloaded on the number of return values, and the
incremental evolution of record and object types.
We discuss how Typed Lua handles these features and our design decisions.
Finally, we present the evaluation results that we achieved while using
Typed Lua to type existing Lua code.

}

\keywords{
\key{Scripting languages}
\key{Lua}
\key{Type systems}
\key{Optional type systems}
\lastkey{Gradual typing}
}

\abstractbr{
Linguagens dinamicamente tipadas, tais como Lua, não usam tipos estáticos em
favor de simplicidade e flexibilidade, porque a ausência de tipos estáticos
significa que programadores não precisam se preocupar em abstrair tipos que
devem ser validados por um verificador de tipos.
Por outro lado, linguagens estaticamente tipadas ajudam na detecção prévia de
diversos \emph{bugs} e também ajudam na estruturação de programas grandes.
Tais pontos geralmente são vistos como duas vantagens que levam programadores
a migrar de uma linguagem dinamicamente tipada para uma linguagem estaticamente tipada,
quando os pequenos \emph{scripts} deles evoluem para programas complexos.
Sistemas de tipos opcionais nos permitem combinar tipagem dinâmica e estática na
mesma linguagem, sem afetar a semântica original da linguagem, tornando mais
fácil a evolução de código tipado dinamicamente para código tipado estaticamente.
Desenvolver um sistema de tipos opcional para uma linguagem dinamicamente tipada é
uma tarefa desafiadora, pois ele deve ser o mais natural possível para os programadores
que já estão familiarizados com essa linguagem.
Neste trabalho nós apresentamos e formalizamos Typed Lua, um sistema de tipos opcional
para Lua, o qual introduz novas características para tipar estaticamente alguns idiomas
e características de Lua.
Embora Lua compartilhe várias características com outras linguagens dinamicamente
tipadas, em particular JavaScript, Lua também possui várias características não usuais,
as quais não estão presentes nos sistemas de tipos dessas linguagens.
Essas características incluem funções com aridade flexível, atribuições múltiplas,
funções que são sobrecarregadas no número de valores de retorno e
a evolução incremental de registros e objetos.
Nós discutimos como Typed Lua tipa estaticamente essas características e
também discutimos nossas decisões de projeto.
Finalmente, apresentamos uma avaliação de resultados,
a qual obtivemos ao usar Typed Lua para tipar código Lua existente.

}

\keywordsbr{
\key{Linguagens de script}
\key{Lua}
\key{Sistemas de tipos}
\key{Sistemas de tipos opcionais}
\lastkey{Tipagem gradual}
}

\tablesmode{figtab}

\begin{document}

\chapter{Introduction}
\label{chap:intro}
Dynamically typed languages such as Lua avoid static types in favor of
simplicity and flexibility, because the absence of static types means
that programmers do not need to bother with abstracting types that
should be validated by a type checker.
Instead, dynamically typed languages use run-time \emph{type tags}
to classify the values they compute, so their implementation can use
these tags to perform run-time (or dynamic) type checking
\cite{pierce2002tpl}.

This simplicity and flexibility allows programmers to write code that
might require a complex type system to statically type check,
though it may also hide bugs that will be caught only after deployment
if programmers do not properly test their code.
In contrast, static type checking helps programmers detect many
bugs during the development phase.
Static types also provide a conceptual framework that helps
programmers define modules and interfaces that can be combined to
structure the development of programs.

Thus, early error detection and better program structure are two
advantages of static type checking that can lead programmers to
migrate their code from a dynamically typed to a statically
typed language, when their simple scripts evolve into complex programs
\cite{tobin-hochstadt2006ims}.
Dynamically typed languages certainly help programmers during the
beginning of a project, because their simplicity and flexibility
allows quick development and makes it easier to change code according to
changing requirements.
However, programmers tend to migrate from dynamically typed to
statically typed code as soon as the project has consolidated its
requirements, because the robustness of static types helps
programmers link requirements to abstractions.
This migration usually involves different languages that have
different syntaxes and semantics, which usually requires a complete
rewrite of existing programs instead of incremental evolution from
dynamic to static types.

Ideally, programming languages should offer programmers the
option to choose between static and dynamic typing:
\emph{optional type systems} \cite{bracha2004pluggable} and
\emph{gradual typing} \cite{siek2006gradual} are two similar
approaches for blending static and dynamic typing in the same
language.
The aim of both approaches is to offer programmers the option
to use type annotations where static typing is needed,
allowing the incremental migration from dynamic to static typing.
The difference between these two approaches is the way they treat
run-time semantics.
While optional type systems do not affect run-time semantics,
gradual typing uses run-time checks to ensure that dynamically typed
code does not violate the invariants of statically typed code.

Programmers and researchers sometimes use the term \emph{gradual typing}
to mean the incremental evolution of dynamically typed code into
statically typed code.
For this reason, gradual typing may also refer to optional type
systems and other approaches that blend static and dynamic typing to
help programmers incrementally migrate from dynamic to static typing
without having to switch to a different language, though all these
approaches differ in the way they handle static and dynamic typing
together.
We use the term \emph{gradual typing} to refer to the work of
Siek and Taha \cite{siek2006gradual}.

In this work we present the design and evaluation of Typed Lua:
an optional type system for Lua that is rich enough to
preserve some of the Lua idioms that programmers are already familiar with,
but that also includes new constructs that help programmers
structure Lua programs.

Lua is a small imperative language with first-class functions
(with proper lexical scoping) where the only data structure
mechanism is the \emph{table} --
an associative array that can represent arrays, records, maps, modules, objects, etc.
Tables also have syntactic sugar and metaprogramming support
through operator overloading built into the language.
Unlike other scripting languages, Lua has very limited coercion
among different data types.

Lua prefers to provide mechanisms instead of fixed policies due
to its primary use as an embedded language for configuration and
extension of other applications.
This means that even features such as a module system and
object orientation are a matter of convention instead of default
language constructs.
The result is a fragmented ecosystem of libraries, and different
ideas among Lua programmers on how they should use the language
features, or how they should structure programs.

The lack of standard policies is a challenge for the design of
an optional type system for Lua.
For this reason, we are not relying entirely on the semantics of
the language to design our type system.
We also run a mostly automated survey of Lua idioms used in a
large corpus of Lua libraries, which also has helped in the design of Typed Lua.

So far, Typed Lua is a Lua extension that allows statically typed
code to coexist and interact with dynamically typed code
through optional type annotations.
In addition, it adds default constructs that programmers can use
to better structure Lua programs.
The Typed Lua compiler warns programmers about type errors,
but always generates Lua code that runs in unmodified Lua implementations.
Programmers can enjoy some of the benefits of static types
even without converting existing Lua modules to Typed Lua --
they can export a statically typed interface to a dynamically typed module,
and statically typed users of the module can use the Typed Lua compiler
to check their use of the module.
Thus, implementing an optional type system for Lua offers Lua
programmers one way to obtain most of the advantages of static typing
without compromising the simplicity and flexibility of dynamic typing.
We have an implementation of the Typed Lua compiler that is
available online\footnote{https://github.com/andremm/typedlua}.

Typed Lua's intended use is as an application language, and
we believe that policies for organizing a program in modules and writing
object-oriented programs should be part of the language and
checked by its optional type system.
An application language is a programming language that helps
programmers develop applications from scratch until these
applications evolve into complex systems rather than just scripts.
We will show that Typed Lua introduces the refinement of
tables to support the common idioms that Lua programmers use
to encode both modules and objects.

We also believe that Typed Lua helps programmers give more
formal documentation to already existing Lua code, as static types
are also a useful source of documentation in languages that provide
type annotations, because type annotations are always validated by
the type checker and therefore never get outdated.
Thus, programmers can use Typed Lua to define axioms about the
interfaces and types of dynamically typed modules.
We enforce this point by using Typed Lua to statically type
the interface of the Lua standard library and other commonly used
Lua libraries, so our compiler can check Typed Lua code that uses
these libraries.

Typed Lua performs a very limited form of local type inference
\cite{pierce2000lti}, as static typing does not necessarily mean
that programmers need to insert type annotations in the code.
Several statically typed languages such as Haskell provide some
amount of type inference that automatically deduces the types of
expressions.
Still, Typed Lua only requires a small amount of type annotations
due to the nature of its optional type system.

Typed Lua does not deal with code optimization, although another
important advantage of static types is that they help the compiler
perform optimizations and generate more efficient code.
However, we believe that the formalization of our optional type
system is precise enough to aid optimization in some Lua implementations.

We use some of the ideas of gradual typing to formalize Typed Lua.
Even though Typed Lua is an optional type system and thus does not
include run-time checks between dynamic and static regions of the
code, we believe that using the foundations of gradual typing to
formalize our optional type system will allow us to include run-time
checks in the future.

Finally, we believe that designing an optional type system for Lua may
shed some light on optional type systems for scripting languages
in general, as Lua is a small scripting language that shares
some features with other scripting languages such as JavaScript.

This work is split into seven chapters.
In Chapter \ref{chap:review} we review the literature about blending
static and dynamic typing in the same language, we discuss the differences
between optional type systems and gradual typing, and we also
present the results of our survey on Lua idioms.
In Chapter \ref{chap:typedlua} we use code examples to present the
design of Typed Lua.
In Chapter \ref{chap:system} we present our type system.
In Chapter \ref{chap:evaluation} we discuss the evaluation
results that we obtained while using Typed Lua to type existing Lua code.
In Chapter \ref{chap:related} we present some related work.
In Chapter \ref{chap:conc} we outline our contributions.



\chapter{Blending static and dynamic typing}
\label{chap:review}
We begin this chapter presenting a little bit of the history behind
combining static and dynamic typing in the same language.
Then, we introduce optional type systems and gradual typing.
After that, we discuss why optional type systems and two
other approaches are often called gradual typing.
We end this chapter presenting some statistics about the usage of
some Lua features and idioms that helped us identify how we should
combine static and dynamic typing in Lua.

\section{A little bit of history}
\label{sec:history}

Common LISP \cite{steele1982ocl} introduced optional type annotations
in the early eighties, but not for static type checking.
Instead, programmers could choose to declare types of variables as
optimization hints to the compiler, that is, type declarations are
just one way to help the compiler to optimize code.
These annotations are unsafe because they can crash the program
when they are wrong.

Abadi et al. \cite{abadi1989dts} extended the simply typed
lambda calculus with the \texttt{Dynamic} type and the \texttt{dynamic}
and \texttt{typecase} constructs, with the aim to safely integrate dynamic
code in statically typed languages.
The \texttt{Dynamic} type is a pair \texttt{(v,T)} where \texttt{v} is a
value and \texttt{T} is the tag that represents the type of \texttt{v}.
The constructs \texttt{dynamic} and \texttt{typecase} are explicit
injection and projection operations, respectively.
That is, \texttt{dynamic} builds values of type \texttt{Dynamic} and
\texttt{typecase} safely inspects the type of a \texttt{Dynamic} value.
Thus, migrating code between dynamic and static type checking requires
changing type annotations and adding or removing \texttt{dynamic} and
\texttt{typecase} constructs throughout the code.

The \emph{quasi-static} type system proposed by Thatte \cite{thatte1990qst}
performs implicit coercions and run-time checks to replace the
\texttt{dynamic} and \texttt{typecase} constructs that were proposed by
Abadi et al. \cite{abadi1989dts}.
To do that, quasi-static typing relies on subtyping with a top type
$\Omega$ that represents the dynamic type, and splits type checking
into two phases.
The first phase inserts implicit coercions from the dynamic type to
the expected type, while the second phase performs what Thatte calls
\emph{plausibility checking}, that is, it rewrites the program to
guarantee that sequences of upcasts and downcasts always have a
common subtype.

\emph{Soft typing} \cite{cartwright1991soft} is another approach
to combine static and dynamic typing in the same language.
The main goal of soft typing is to add static type checking to
dynamically typed languages without compromising their flexibility.
To do that, soft typing relies on type inference for
translating dynamically typed code to statically typed code.
The type checker inserts run-time checks around inconsistent code and
warns the programmer about the insertion of these run-time checks,
as they indicate the existence of potential type errors.
However, the programmer is free to choose between inspecting the
run-time checks or simply running the code.
This means that type inference and static type checking do
not prevent the programmer from running inconsistent code.
One advantage of soft typing is the fact that the compiler for
softly typed languages can use the translated code to generate
more efficient code, as the translated code statically type checks.
One disadvantage of soft typing is that it can be cumbersome when
the inferred types are meaningless large types that just confuse the
programmer.

\emph{Dynamic typing} \cite{henglein1994dts} is an approach
that optimizes code from dynamically typed languages by eliminating
unnecessary checks of tags.
Henglein describes how to translate dynamically typed code into
statically typed code that uses a \texttt{Dynamic} type.
The translation is done through a coercion calculus that uses type
inference to insert the operations that are necessary to type check
the \texttt{Dynamic} type during run-time.
Although soft typing and dynamic typing may seem similar, they are not.
Soft typing targets statically type checking of dynamically typed
languages for detecting programming errors, while
dynamic typing targets the optimization of dynamically
typed code through the elimination of unnecessary run-time checks.
In other words, soft typing sees code optimization as a side effect
that comes with static type checking.

Findler and Felleisen \cite{findler2002chf} proposed contracts for
higher-order functions and blame annotations for run-time checks.
Contracts perform dynamic type checking instead of static type checking,
but deferring all verifications to run-time can lead to defects
that are difficult to fix, because run-time errors can show a
stack trace where it is not clear to programmers if the cause of a
certain run-time error is in application code or library code.
Even if programmers identify that the source of a certain run-time
error is in library code, they still may have problems to identify
if this run-time error is due to a violation of library's contract
or due to a bug, when the library is poorly documented.
In this approach, programmers can insert assertions in the form of
contracts that check the input and output of higher-order functions;
and the compiler adds blame annotations in the generated code
to track assertion failures back to the source of the error.

BabyJ \cite{anderson2003babyj} is an object-oriented language
without inheritance that allows programmers to incrementally annotate
the code with more specific types.
Programmers can choose between using the dynamically typed version
of BabyJ when they do not need types at all, and the statically
typed version of BabyJ when they need to annotate the code.
In statically typed BabyJ, programmers can use the
\emph{permissive type} $*$ to annotate the parts of the code that
still do not have a specific type or the parts of the code that should
have dynamic behavior.
The type system of BabyJ is nominal, so types are either class names
or the permissive type $*$.
However, the type system does not use type equality or subtyping,
but the relation $\approx$ between two types.
The relation $\approx$ holds when both types have the same name or
any of them is the permissive type $*$.
Even though the permissive type $*$ is similar to the dynamic type
from previous approaches, BabyJ does not provide any way to add
implicit or explicit run-time checks.

Ou et al. \cite{ou2004dtd} specified a language that combines
static types with dependent types.
To ensure safety, the compiler automatically inserts coercions
between dependent code and static code.
The coercions are run-time checks that ensure static code does not
crash dependent code during run-time.

\section{Optional Type Systems}
\label{sec:optional}

Optional type systems \cite{bracha2004pluggable} are an approach for
plugging static typing in dynamically typed languages.
They use optional type annotations to perform compile-time type checking,
though they do not affect the original run-time semantics
of the language.
This means that the run-time semantics should still catch type errors
independently of the static type checking.
For instance, we can view the typed lambda calculus as an optional
type system for the untyped lambda calculus, because both have the
same semantic rules and the type system serves only for discarding
programs that may have undesired behaviors \cite{bracha2004pluggable}.

Strongtalk \cite{bracha1993strongtalk,bracha1996strongtalk} is
a version of Smalltalk that comes with an optional type system.
It has a polymorphic type system that programmers can use to annotate
Smalltalk code or leave type annotations out.
Strongtalk assigns a dynamic type to unannotated expressions and allows
programmers to cast unannotated expressions to any static type.
This means that the interaction of the dynamic type with the rest of
the type system is unsound, so Strongtalk uses the original run-time
semantics of Smalltalk when executing programs, even if programs are
statically typed.

\emph{Pluggable type systems} \cite{bracha2004pluggable} generalize
the idea of optional type systems that Strongtalk put in practice.
The idea is to have different optional type systems that can be layered
on top of a dynamically typed language without affecting its original
run-time semantics.
Although these systems can be unsound in their interaction with the
dynamically typed part of the language or even by design, their
unsoundness does not affect run-time safety, as the language run-time
semantics still catches any run-time errors caused by an unsound
type system.

Dart \cite{dart} and TypeScript \cite{typescript} are new
languages that are designed with an optional type system.
Both use JavaScript as their code generation target because
their main purpose is Web development.
In fact, Dart is a new class-based object-oriented language with
optional type annotations and semantics that resembles the
semantics of Smalltalk, while TypeScript is a strict superset of
JavaScript that provides optional type annotations and class-based
object-oriented programming.
Dart has a nominal type system, while TypeScript has a structural
one, but both are unsound by design.
For instance, Dart has covariant arrays, while TypeScript has
covariant parameter types in function signatures,
besides the interaction between statically and dynamically
typed code that is also unsound.

There is no common formalization for optional type systems, and
each language ends up implementing its optional type system in
its own way.
Strongtalk, Dart, and TypeScript provide an informal description of
their optional type systems rather than a formal one.
In the next section we will show that we can use some features
of gradual typing \cite{siek2006gradual,siek2007objects} to
formalize optional type systems.

\section{Gradual Typing}
\label{sec:gradual}

\label{def:dynamictype}
The main goal of gradual typing \cite{siek2006gradual} is to allow
programmers to choose between static and dynamic typing in the same
language.
To do that, Siek and Taha \cite{siek2006gradual} extended the simply
typed lambda calculus with the dynamic type $?$, as we can see in
Figure \ref{fig:gtlc}.
In gradual typing, type annotations are optional, and an untyped
variable is syntactic sugar for a variable whose declared type is
the dynamic type $?$, that is, $\lambda x.e$ is equivalent to $\lambda x{:}?.e$.
Under these circumstances, we can view gradual typing as a way to add
a dynamic type to statically typed languages.

\begin{figure}[!ht]
\dstart
$$
\begin{array}{llr}
T ::= & & \textsc{types:}\\
& \;\; \Number & \textit{base type number}\\
& | \; \String & \textit{base type string}\\
& | \; ? & \textit{dynamic type}\\
& | \; T \rightarrow T & \textit{function types}\\
e ::= & & \textsc{expressions:}\\
& \;\; l & \textit{literals}\\
& | \; x & \textit{variables}\\
& | \; \lambda x{:}T{.}e & \textit{abstractions}\\
& | \; e_{1} e_{2} & \textit{application}
\end{array}
$$
\dend
\caption{Syntax of the gradually-typed lambda calculus}
\label{fig:gtlc}
\end{figure}

\label{def:consistency}
The central idea of gradual typing is the \emph{consistency}
relation, written $T_{1} \sim T_{2}$.
The consistency relation allows implicit conversions to and from the
dynamic type, and disallows conversions between inconsistent types
\cite{siek2006gradual}.
For instance, $\Number \sim \;?$, $? \sim \Number$,
$\String \sim \;?$, and $? \sim \String$,
but $\Number \not\sim \String$, and
$\String \not\sim \Number$.
The consistency relation is both reflexive and symmetric, but
it is not transitive.

\begin{figure}[!ht]
\dstart
$$
\begin{array}{c}
\begin{array}{c}
T \sim T \mylabel{C-REFL}
\end{array}
\;
\begin{array}{c}
T \sim \;? \mylabel{C-DYNR}
\end{array}
\;
\begin{array}{c}
? \sim T \mylabel{C-DYNL}
\end{array}
\\ \\
\begin{array}{c}
\dfrac{T_{3} \sim T_{1} \;\;\; T_{2} \sim T_{4}}
      {T_{1} \rightarrow T_{2} \sim T_{3} \rightarrow T_{4}} \mylabel{C-FUNC}
\end{array}
\end{array}
$$
\dend
\caption{The consistency relation}
\label{fig:consistency}
\end{figure}

Figure \ref{fig:consistency} defines the consistency relation.
The rule \textsc{C-REFL} is the reflexive rule.
Rules \textsc{C-DYNR} and \textsc{C-DYNL} are the rules that allow
conversions to and from the dynamic type $?$.
The rule \textsc{C-FUNC} resembles subtyping between function types,
because it is contravariant on the argument type and covariant on the
return type.

Figure \ref{fig:gts} uses the consistency relation in the typing rules
of the gradual type system of the simply typed lambda calculus extended
with the dynamic type $?$.
The environment $\env$ is a function from variables to types, and
the directive $type$ is a function from literal values to types.
The rule \textsc{T-VAR} uses the environment function $\env$ to get the
type of a variable $x$.
The rule \textsc{T-LIT} uses the directive $type$ to get the type of
a literal $l$.
The rule \textsc{T-ABS} evaluates the expression $e$ with an environment
$\env$ that binds the variable $x$ to the type $T_{1}$, and the resulting
type is the the function type $T_{1} \rightarrow T_{2}$.
The rule \textsc{T-APP1} handles function calls where the type of a
function is dynamically typed; in this case, the argument type may have
any type and the resulting type has the dynamic type.
The rule \textsc{T-APP2} handles function calls where the type of a
function is statically typed; in this case, the argument type should
be consistent with the argument type of the function's signature.

\begin{figure}[!ht]
\dstart
$$
\begin{matrix}
\dfrac{\env(x) = T}
      {\env \vdash x:T} \mylabel{T-VAR}
\;\;\;
\dfrac{type(l) = T}
      {\env \vdash l:T} \mylabel{T-LIT}
\\ \\
\dfrac{\env[x \mapsto T_{1}] \vdash e:T_{2}}
      {\env \vdash \lambda x:T_{1}.e:T_{1} \rightarrow T_{2}} \mylabel{T-ABS}
\;\;\;
\dfrac{\env \vdash e_{1}:\;? \;\;\;
       \env \vdash e_{2}:T}
      {\env \vdash e_{1} e_{2}:\;?} \mylabel{T-APP1}
\\ \\
\dfrac{\env \vdash e_{1}:T_{1} \rightarrow T_{2} \;\;\;
       \env \vdash e_{2}:T_{3} \;\;\;
       T_{3} \sim T_{1}}
      {\env \vdash e_{1} e_{2}:T_{2}} \mylabel{T-APP2}
\end{matrix}
$$
\dend
\caption{Gradual type system gradually-typed lambda calculus}
\label{fig:gts}
\end{figure}

Gradual typing \cite{siek2006gradual} is similar to two previous
approaches \cite{abadi1989dts,thatte1990qst}, because they also include
a dynamic type in a statically typed language.
However, these three approaches differ in the way they handle the
dynamic type.
While Siek and Taha \cite{siek2006gradual} rely on the consistency relation,
Abadi et al. \cite{abadi1989dts} rely on type equality with explicit
projections plus injections, and Thatte \cite{thatte1990qst} relies on subtyping.

The subtyping relation $\subtype$ is actually a pitfall on Thatte's
quasi-static typing, because it sets the dynamic type
as the top and the bottom of the subtying relation:
$T \subtype \;?$ and $? \subtype T$.
Subtyping is transitive, so we know that
\[
\frac{\Number \subtype \;? \;\;\;
      ? \subtype \String}
     {\Number \subtype \String}
\]
Therefore, downcasts combined with the transitivity of subtyping
accepts programs that should be rejected.

\label{def:consistent-subtyping}
Later, Siek and Taha \cite{siek2007objects} reported that the consistency relation
is orthogonal to the subtyping relation, so we can combine them to achieve
the \emph{consistent-subtyping} relation, written $T_{1} \lesssim T_{2}$.
This relation is essential for designing gradual type systems for
object-oriented languages.
Like the consistency relation, and unlike the subtyping relation,
the consistent-subtyping relation is not transitive.
Therefore, $\Number \lesssim \;?$, $? \lesssim \Number$,
$\String \lesssim \;?$, and $? \lesssim \String$,
but $\Number \not\lesssim \String$, and
$\String \not\lesssim \Number$.

Now, we will show how we can combine consistency and subtyping
to compose a consistent-subtyping relation for the simply typed
lambda calculus extended with the dynamic type $?$.

\begin{figure}[!ht]
\dstart
$$
\begin{array}{c}
\begin{array}{c}
\Number \subtype \Number \mylabel{S-NUM}
\end{array}
\;
\begin{array}{c}
\String \subtype \String \mylabel{S-STR}
\end{array}
\\ \\
\begin{array}{c}
? \subtype \;? \mylabel{S-ANY}
\end{array}
\;
\begin{array}{c}
\dfrac{T_{3} \subtype T_{1} \;\;\; T_{2} \subtype T_{4}}
      {T_{1} \rightarrow T_{2} \subtype T_{3} \rightarrow T_{4}} \mylabel{S-FUN}
\end{array}
\end{array}
$$
\dend
\caption{The subtyping relation}
\label{fig:subtyping}
\end{figure}

Figure \ref{fig:subtyping} presents the subtyping relation for the simply
typed lambda calculus extended with the dynamic type $?$.
Even though we could have used the reflexive rule $T \subtype T$ to express
the rules \textsc{S-NUM}, \textsc{S-STR}, and \textsc{S-ANY},
we did not combine them into a single rule to make explicit the
neutrality of the dynamic type $?$ to the subtyping rules.
The dynamic type $?$ must be neutral to subtyping to avoid the pitfall
from Thatte's quasi-static typing.
The rule \textsc{S-FUN} defines the subtyping relation for function types,
which are contravariant on the argument type and covariant on the return type.

\begin{figure}[!ht]
\dstart
$$
\begin{array}{c}
\begin{array}{c}
\Number \lesssim \Number \mylabel{C-NUM}
\end{array}
\;
\begin{array}{c}
\String \lesssim \String \mylabel{C-STR}
\end{array}
\\ \\
\begin{array}{c}
T \lesssim \;? \mylabel{C-ANY1}
\end{array}
\;
\begin{array}{c}
? \lesssim T \mylabel{C-ANY2}
\end{array}
\\ \\
\begin{array}{c}
\dfrac{T_{3} \lesssim T_{1} \;\;\; T_{2} \lesssim T_{4}}
      {T_{1} \rightarrow T_{2} \lesssim T_{3} \rightarrow T_{4}} \mylabel{C-FUN}
\end{array}
\end{array}
$$
\dend
\caption{The consistent-subtyping relation}
\label{fig:consistent-subtyping}
\end{figure}

Figure \ref{fig:consistent-subtyping} combines the consistency and subtyping
relations to compose the consistent-subtyping relation for the simply typed
lambda calculus extended with the dynamic type $?$.
When we combine consistency and subtyping, we are making subtyping handle
which casts are safe among static types, and we are making consistency
handle the casts that involve the dynamic type $?$.
The consistent-subtyping relation is not transitive, and thus
the dynamic type $?$ is not neutral to this relation.

So far, gradual typing looks like a mere formalization to optional
type systems, as a gradual type system uses the consistency or
consistent-subtyping relation to statically check the interaction
between statically and dynamically typed code, without affecting
the run-time semantics.

However, another important feature of gradual typing is the theoretic
foundation that it provides for inserting run-time checks that
prove dynamically typed code does not violate the invariants of
statically typed code, thus preserving type safety.
To do that, Siek and Taha \cite{siek2006gradual,siek2007objects}
defined the run-time semantics of gradual typing as a translation to an
intermediate language with explicit casts at the frontiers between
statically and dynamically typed code.
The semantics of these casts is based on the higher-order contracts
proposed by Findler and Felleisen \cite{findler2002chf}.

Herman et al. \cite{herman2007sgt} showed that there is an
efficiency concern regarding the run-time checks, because there are
two ways that casts can lead to unbounded space consumption.
The first affects tail recursion while the second appears when
first-class functions or objects cross the border between
static code and dynamic code, that is, some programs can apply
repeated casts to the same function or object.
Herman et al. \cite{herman2007sgt} use the coercion calculus
outlined by Henglein \cite{henglein1994dts} to express casts
as coercions and solve the problem of space efficiency.
Their approach normalizes an arbitrary sequence of coercions to a
coercion of bounded size.

Another concern about casts is how to improve debugging support,
because a cast application can be delayed and the error related
to that cast application can appear considerable distance
from the real error.
Wadler and Findler \cite{wadler2009wpc} developed \emph{blame calculus}
as a way to handle this issue, and Ahmed et al. \cite{ahmed2011bfa}
extended blame calculus with polymorphism.
Blame calculus is an intermediate language to integrate
static and dynamic typing along with the blame tracking approach
proposed by Findler and Felleisen \cite{findler2002chf}.

On the one hand, blame calculus solves the issue regarding
error reporting;
on the other hand, it has the space efficiency problem reported
by Herman et al. \cite{herman2007sgt}.
Thus, Siek et al. \cite{siek2009casts} extended the coercion
calculus outlined by Herman et al. \cite{herman2007sgt} with
blame tracking to achieve an implementation of the blame calculus that
is space efficient.
After that, Siek and Wadler \cite{siek2010blame} proposed a new solution
that also handles both problems.
This new solution is based on a concept called \emph{threesome},
which is a way to split a cast between two parties into two casts
among three parties.
A cast has a source and a target type (a \emph{twosome}),
so we can split any cast into a downcast from the source to an
intermediate type that is followed by an upcast from the intermediate
type to the target type (a \emph{threesome}).

There are some projects that incorporate gradual typing into some
programming languages.
Reticulated Python \cite{reticulated,vitousek2014deg} is a research
project that evaluates the costs of gradual typing in Python.
Gradualtalk \cite{allende2013gts} is a gradually-typed Smalltalk
that introduces a new cast insertion strategy for gradually-typed
objects \cite{allende2013cis}.
Grace \cite{black2012grace,black2013sg} is a new object-oriented,
gradually-typed, educational language.
In Grace, modules are gradually-typed objects, that is, modules
may have types with methods as attributes, and they can also have
a state \cite{homer2013modules}.
ActionScript \cite{moock2007as3} is one the first languages that
incorporated gradual typing to its implementation and
Perl 6 \cite{tang2007pri} is also being designed with gradual typing,
though there is few documentation about the gradual type systems
of these languages.

\section{Approaches that are often called Gradual Typing}
\label{sec:approaches}

Gradual typing is similar to optional type systems in that type
annotations are optional, and unannotated code is dynamically
typed, but unlike optional type systems, gradual typing changes
the run-time semantics to preserve type safety.
More precisely, programming languages that include a gradual type
system can implement the semantics of statically typed languages, so
the gradual type system inserts casts in the translated code to
guarantee that types are consistent before execution, while
programming languages that include an optional type system still
need implement the semantics of dynamically typed languages, so all
the type checking also belongs to the semantics of each operation.

Still, we can view gradual typing as a way to formalize an optional
type system when the gradual type system does not insert run-time
checks.
BabyJ \cite{anderson2003babyj} and Alore \cite{lehtosalo2011alore}
are two examples of object-oriented languages that have an
optional type system with a formalization that relates to gradual typing,
though the optional type systems of both BabyJ and Alore are nominal.
BabyJ uses the relation $\approx$ that is similar to the consistency
relation while Alore combines subtyping along with the consistency
relation to define a \emph{consistent-or-subtype} relation.
The consistent-or-subtype relation is different from the
consistent-subtyping relation of Siek and Taha \cite{siek2007objects},
but it is also written $T_{1} \lesssim T_{2}$.
The consistent-or-subtype relation holds when $T_{1} \sim T_{2}$
or $T_{1} <: T_{2}$, where $<:$ is transitive and $\sim$ is not.
Alore also extends its optional type system to include optional
monitoring of run-time type errors in the gradual typing style.

Hence, optional type annotations for software evolution are likely
the reason why optional type systems are commonly called
gradual type systems.
Typed Clojure \cite{bonnaire-sergeant2012typed-clojure} is an
optional type system for Clojure that is now adopting the
gradual typing slogan.

Flanagan \cite{flanagan2006htc} introduced \emph{hybrid type checking},
an approach that combines static types and \emph{refinement} types.
For instance, programmers can specify the refinement type
$\{x:Int \;|\; x \ge 0\}$ when they need a type for natural numbers.
The programmer can also choose between explicit or implicit casts.
When casts are not explicit, the type checker uses a theorem prover
to insert casts.
In our example of natural numbers, a cast would be inserted to check
whether an integer is greater than or equal to zero.

Sage \cite{gronski2006sage} is a programming language that
extends hybrid type checking with a dynamic type to
support dynamic and static typing in the same language.
Sage also offers optional type annotations in the gradual typing
style, that is, unannotated code is syntactic sugar for
code whose declared type is the dynamic type.

Thus, the inclusion of a dynamic type in hybrid type checking
along with optional type annotations, and the insertion of run-time
checks are likely the reason why hybrid type checking is
also viewed as a form of gradual typing.

Tobin-Hochstadt and Felleisen \cite{tobin-hochstadt2006ims} proposed
another approach for gradually migrating from dynamically typed to
statically typed code, and they coined the term
\emph{from scripts to programs} for referring to this kind of
interlanguage migration.
In their approach, the migration from dynamically typed to
statically typed code happens module-by-module, so they designed
and implemented Typed Racket \cite{tobin-hochstadt2008ts} for
this purpose.
Typed Racket is a statically typed version of Racket
(a Scheme dialect) that allows the programmer to write typed modules,
so Typed Racket modules can coexist with Racket modules,
which are untyped.

The approach used by Tobin-Hochstadt and Felleisen \cite{tobin-hochstadt2008ts}
to design and implement Typed Racket is probably also called gradual typing
because it allows the programmer to gradually migrate from untyped
scripts to typed programs.
However, Typed Racket is a statically typed language,
and what makes it gradual is a type system with a dynamic type
that handles the interaction between Racket and Typed Racket modules.

Recently, Siek et al. \cite{siek2015refined} described a formal
criteria on what is gradual typing: the \emph{gradual guarantee}.
Besides allowing static and dynamic typing in the same code
along with type soundness, the gradual guarantee states that
removing type annotations from a gradually typed program that is
well typed must continue well typed.
The other direction must be also valid, that is, adding correct type
annotations to a gradually typed program that is well typed must
continue well typed.
In other words, the gradual guarantee states that any changes to
the annotations does not change the static or the dynamic behavior
of a program \cite{siek2015refined}.
The authors prove the gradual guarantee and discuss whether
some previous projects match this criteria.

\section{Statistics about the usage of Lua}
\label{sec:statistics}

In this section we present statistics about the usage of Lua
features and idioms.
We collected statistics about how programmers use tables, functions,
dynamic type checking, object-oriented programming, and modules.
We shall see that these statistics informed important design decisions
on our optional type system.

We used the LuaRocks repository to build our statistics database;
LuaRocks \cite{hisham2013luarocks} is a package manager for Lua
modules.
We downloaded the 3928 \texttt{.lua} files that were available in
the LuaRocks repository at February 1st 2014.
However, we ignored files that were not compatible with Lua 5.2,
the latest version of Lua at that time.
We also ignored \emph{machine-generated} files and test files,
because these files may not represent idiomatic Lua code,
and might skew our statistics towards non-typical uses of Lua.
This left 2598 \texttt{.lua} files from 262 different projects for
our statistics database;
we parsed these files and processed their abstract syntax tree
to gather the statistics that we show in this section.

To verify how programmers use tables, we measured how they
initialize, index, and iterate tables.
We present these statistics in the next three paragraphs to discuss
their influence on our type system.

The table constructor appears 23185 times.
In 36\% of the occurrences it is a constructor that initializes a
record (e.g., \texttt{\{ x = 120, y = 121 \}});
in 29\% of the occurrences it is a constructor that initializes a
list (e.g., \texttt{\{ "one", "two", "three", "four" \}});
in 8\% of the occurrences it is a constructor that initializes a
record with a list part;
and in less than 1\% of the occurrences (4 times) it is a constructor
that uses only the booleans \texttt{true} and \texttt{false} as indexes.
At all, in 73\% of the occurrences it is a constructor that uses
only literal keys;
in 26\% of the occurrences it is the empty constructor;
in 1\% of the occurrences it is a constructor with non-literal keys
only, that is, a constructor that uses variables and function calls
to create the indexes of a table;
and in less than 1\% of the occurrences (19 times) it is a constructor
that mixes literal keys and non-literal keys.

The indexing of tables appears 130448 times:
86\% of them are for reading a table field while
14\% of them are for writing into a table field.
We can classify the indexing operations that are reads as follows:
89\% of the reads use a literal string key,
4\% of the reads use a literal number key,
and less than 1\% of the reads (10 times) use a literal boolean key.
At all, 93\% of the reads use literals to index a table while
7\% of the reads use non-literal expressions to index a table.
It is worth mentioning that 45\% of the reads are actually
function calls.
More precisely, 25\% of the reads use literals to call a function,
20\% of the reads use literals to call a method, that is,
a function call that uses the colon syntactic sugar, 
and less than 1\% of the reads (195 times) use non-literal expressions
to call a function.
We can also classify the indexing operations that are writes as follows: 
69\% of the writes use a literal string key,
2\% of the writes use a literal number key,
and less than 1\% of the writes (1 time) uses a literal boolean key.
At all, 71\% of the writes use literals to index a table while
29\% of the writes use non-literal expressions to index a table.

We also measured how many files have code that iterates over tables to
observe how frequently iteration is used.
We observed that 23\% of the files have code that iterates over keys
of any value, that is, the call to \texttt{pairs} appears at least
once in these files (the median is twice per file);
21\% of the files have code that iterates over integer keys, that is,
the call to \texttt{ipairs} appears at least once in these files
(the median is also twice per file);
and 10\% of the files have code that use the numeric \texttt{for}
along with the length operator (the median is once per file).

The numbers about table initialization, indexing, and iteration
show us that tables are mostly used to represent records, lists,
and associative arrays.
Therefore, Typed Lua should include a table type for handling
these uses of Lua tables.
Even though the statistics show that programmers initialize tables
more often than they use the empty constructor to
dynamically initialize tables, the statistics of the empty
constructor are still expressive and indicate that Typed Lua should
also include a way to handle this style of defining table types.

We found a total of 24858 function declarations in our database
(the median is six per file).
Next, we discuss how frequently programmers use dynamic type
checking and multiple return values inside these functions.

We observed that 9\% of the functions perform dynamic type checking
on their input parameters, that is, these functions use \texttt{type}
to inspect the tags of Lua values (the median is once per function).
We randomly selected 20 functions to sample how programmers are
using \texttt{type}, and we got the following data:
50\% of these functions use \texttt{type} for asserting the tags of
their input parameters, that is, they raise an error when the tag of a
certain parameter does not match the expected tag, and
50\% of these functions use \texttt{type} for overloading, that is,
they execute different code according to the inspected tag.

These numbers show us that Typed Lua should include union types,
because the use of the \texttt{type} idiom shows that disjoint unions
would help programmers define data structures that can hold a value of
several different, but fixed types.
Typed Lua should also use \texttt{type} as a mechanism for decomposing
unions, though it may be restricted to base types only.

We observed that 10\% of the functions explicitly return multiple
values.
We also observed that 5\% of the functions return \texttt{nil} plus
something else, for signaling an unexpected behavior;
and 1\% of the functions return \texttt{false} plus something else,
also for signaling an unexpected behavior.

Typed Lua should include function types to represent Lua functions,
and tuple types to represent the signatures of Lua functions,
multiple return values, and multiple assignments.
Tuple types require some special attention, because Typed Lua
should be able to adjust tuple types during compile-time, in a
similar way to what Lua does with function calls and multiple
assignments during run-time.
In addition, the number of functions that return \texttt{nil} and
\texttt{false} plus something else show us that overloading on the
return type is also useful to the type system.

We also measured how frequently programmers use the object-oriented
paradigm in Lua.
We observed that 23\% of the function declarations are actually
method declarations.
More precisely, 14\% of them use the colon syntactic sugar while
9\% of them use \texttt{self} as their first parameter.
We also observed that 63\% of the projects extend tables with
metatables, that is, they call \texttt{setmetatable} at least once,
and 27\% of the projects access the metatable of a given table,
that is, they call \texttt{getmetatable} at least once.
In fact, 45\% of the projects extend tables with metatables and
declare methods:
13\% using the colon syntactic sugar, 14\% using \texttt{self}, and
18\% using both.

Based on these observations, Typed Lua should include support
to object-oriented programming.
Even though Lua does not have standard policies for object-oriented
programming, it provides mechanisms that allow programmers to
abstract their code in terms of objects, and our statistics confirm
that an expressive number of programmers are relying on these mechanisms
to use the object-oriented paradigm in Lua.
Typed Lua should include some standard way of defining interfaces and classes
that the compiler can use to type check object-oriented code,
but without changing the semantics of Lua.

We also measured how programmers are defining modules.
We observed that 38\% of the files use the current way of defining
modules, that is, these files return a table that contains the
exported members of the module at the end of the file;
22\% of the files still use the deprecated way of defining modules,
that is, these files call the function \texttt{module};
and 1\% of the files use both ways.
At all, 61\% of the files are modules while 39\% of the files are
plain scripts.
The number of plain scripts is high considering the origin of
our database.
However, we did not ignore sample scripts, which usually serve to
help the users of a given module on how to use this module, and
that is the reason why we have a high number of plain scripts.

Based on these observations, Typed Lua should include a way
for defining table types that represent the type of modules.
Typed Lua should also support the deprecated style of module
definition, using global names as exported members of the module.

Typed Lua should also include some way to define the types of
userdata.
This feature should also allow programmers to define userdata
that can be used in an object-oriented style, as this is another
common idiom from modules that are written in C.

The last statistics that we collected were about variadic functions
and vararg expressions.
We observed that 8\% of the functions are variadic, that is,
their last parameter is the vararg expression.
We also observed that 5\% of the initialization of lists
(or 2\% of the occurrences of the table constructor) use solely the
vararg expression.
Typed Lua should include a \emph{vararg type} to handle variadic
functions and vararg expressions.

Table \ref{tab:statistics} summarizes the statistics that we presented in this section.

\begin{table}[!ht]
\begin{center}
\begin{tabular}{|p{6cm}|p{6cm}|c|}
\hline
\multirow{5}{*}{\minitab[l]{table constructor \\ (\% per static occurrences)}}
& create a record & 36\% \\
\cline{2-3}
& create a list & 29\% \\
\cline{2-3}
& create an empty table & 26\% \\
\cline{2-3}
& create a table with a record part and a list part & 8\% \\
\cline{2-3}
& create a table with non-literal keys & 1\% \\
\hline
\multirow{4}{*}{\minitab[l]{table access \\ (\% per static occurrences)}}
& reading with literal keys & 80\% \\
\cline{2-3}
& writing with literal keys & 10\% \\
\cline{2-3}
& reading with non-literal keys & 6\% \\
\cline{2-3}
& writing with non-literal keys & 4\% \\
\hline
\multirow{2}{*}{\minitab[l]{iteration over tables \\ (\% per files)}} &
files that iterate over a list & 27\% \\
\cline{2-3}
& files that iterate over a map & 23\% \\
\hline
\multirow{4}{*}{\minitab[l]{function declarations \\ (\% per static occurrences)}}
& inspect the tags of their input parameters & 9\% \\
\cline{2-3}
& return multiple values to signal errors & 6\% \\
\cline{2-3}
& are variadic & 8\% \\
\cline{2-3}
& are method declarations & 23\% \\
\hline
\multirow{1}{*}{\minitab[l]{object-oriented programming \\ (\% per projects)}}
& projects that use metatables and declare methods & 45\% \\
\hline
\multirow{2}{*}{\minitab[l]{modules \\ (\% per files)}}
& files that are modules & 61\% \\
\cline{2-3}
& files that are plain scripts & 39\% \\
\hline
\end{tabular}
\end{center}
\caption{Summary of the statistics about the usage of Lua}
\label{tab:statistics}
\end{table}



\chapter{Typed Lua}
\label{chap:typedlua}
\documentclass{sig-alternate}

\usepackage[utf8]{inputenc}
\usepackage[numbers]{natbib}
\usepackage{amsmath}
\usepackage{amssymb}
\usepackage{url}

\newcommand{\Any}{\mathbf{any}}
\newcommand{\Top}{\mathbf{value}}
\newcommand{\Nil}{\mathbf{nil}}
\newcommand{\False}{\mathbf{false}}
\newcommand{\True}{\mathbf{true}}
\newcommand{\Boolean}{\mathbf{boolean}}
\newcommand{\Number}{\mathbf{number}}
\newcommand{\String}{\mathbf{string}}
\newcommand{\Void}{\mathbf{void}}
\newcommand{\Const}{\mathbf{const}}
\newcommand{\Self}{\mathbf{self}}

\def\dstart{\hbox to \hsize{\vrule depth 4pt\hrulefill\vrule depth 4pt}}
\def\dend{\hbox to \hsize{\vrule height 4pt\hrulefill\vrule height 4pt}}

\begin{document}

\permission{Permission to make digital or hard copies of all or part of this work for
   personal or classroom use is granted without fee provided that copies are
   not made or distributed for profit or commercial advantage and that copies
   bear this notice and the full citation on the first page. Copyrights for
   components of this work owned by others than the author(s) must be
   honored. Abstracting with credit is permitted. To copy otherwise, or
   republish, to post on servers or to redistribute to lists, requires prior
   specific permission and/or a fee. Request permissions from
   Permissions@acm.org.}
\conferenceinfo{Dyla}{'14, June 09-11 2014, Edinburgh, United Kingdom} 
\CopyrightYear{is held by the owner/author(s). Publication rights licensed to
ACM.}
\crdata{978-1-4503-2916-3/14/06}

\title{Typed Lua: An Optional Type System for Lua}

\numberofauthors{3}

\author{
\alignauthor
André Murbach Maidl\\
  \affaddr{PUC-Rio}\\
  \affaddr{Rio de Janeiro, Brazil}\\
  \email{amaidl@inf.puc-rio.br}
\alignauthor
Fabio Mascarenhas\\
  \affaddr{UFRJ}\\
  \affaddr{Rio de Janeiro, Brazil}\\
  \email{mascarenhas@ufrj.br}
\alignauthor
Roberto Ierusalimschy\\
  \affaddr{PUC-Rio}\\
  \affaddr{Rio de Janeiro, Brazil}\\
  \email{roberto@inf.puc-rio.br}
}

\date{June 12th 2014}

\maketitle

\begin{abstract}
Dynamically typed languages trade flexibility and ease of
use for safety, while statically typed languages prioritize
the early detection of bugs, and provide a better framework
for structure large programs. The idea of optional typing
is to combine the two approaches in the same
language: the programmer can begin development with
dynamic types, and migrate to static types as the
program matures. The challenge is designing a type system
that feels natural to the programmer that is used to
programming in a dynamic language.

This paper presents the initial design of Typed Lua, an
optionally-typed extension of the Lua scripting language.
Lua is an imperative scripting language with first
class functions and lightweight metaprogramming mechanisms.
The design of Typed Lua's type system has a novel combination
of features that preserves some of the idioms that Lua
programmers are used to, while bringing static type
safety to them. We show how the major features of the
type system type these idioms with some
examples, and discuss some of the design issues we faced.
\end{abstract}

\category{D.3.1}{Programming Languages}{Formal Definitions and Theory}[Semantics]
\category{F.3.3}{Logics and Meanings of Programs}{Studies of Program Constructs}[Type structure]

\keywords{Lua programming language, type systems, optional typing, gradual typing}

\section{Introduction} \label{sec:intro}

Dynamically typed languages such as Lua avoid static types in favor of
runtime {\em type tags} that classify the values they compute, and
their implementations use these tags to perform runtime (or dynamic)
checking and guarantee that only valid operations are
performed~\citep{pierce2002tpl}.

The absence of static types means that programmers do not need to
bother about abstracting types that might require a complex type
system and type checker to validate, leading to simpler and more flexible
languages and implementations. But this absence may also hide bugs
that will be caught only after deployment if programmers do not properly
test their code.

In contrast, static type checking helps programmers detect many 
bugs during the development phase. Static types also provide a
conceptual framework that helps programmers define modules
and interfaces that can be combined to structure the development
of large programs.

The early error detection and better program structure afforded by
static type checking can lead programmers to migrate their code from
a dynamically typed to a statically typed language, once their simple
scripts become complex programs~\citep{tobin-hochstadt2006ims}.
As this migration involves languages with different syntax and
semantics, it requires a complete rewrite of existing programs instead
of incremental evolution from dynamic to static types.

Ideally, programming languages should offer programmers the
option to choose between static and dynamic typing:
\textit{optional type systems}~\citep{bracha2004pluggable} and
\textit{gradual typing}~\citep{siek2006gradual} are two approaches
that offer programmers the option to use type annotations where static
typing is needed, incrementally migrating a system from dynamic
to static types. The difference between these two approaches is the
way they treat runtime semantics: while optional type systems
do not affect the runtime semantics,
gradual typing uses runtime checks to ensure that dynamically typed
code does not violate the invariants of statically typed code.

This paper presents the initial design of Typed Lua:
an optional type system for Lua that is complex enough to
preserve some of the idioms that Lua programmers are already
used to, while adding new constructs that help programmers
structure Lua programs.

Lua is a small imperative scripting language with first-class
functions (with proper lexical scoping) where the main data
structure is the {\em table}, an associative array that can
play the part of arrays, records, maps, objects, etc.
with syntactic sugar and metaprogramming through operator overloading built into
the language. Unlike other scripting languages, Lua has very
limited coercion among different data types.

The primary use of Lua has always been as an embedded language
for configuration and extension of other applications.
Lua prefers to provide mechanisms instead of fixed policies
for structuring programs, and even features
such as a module system and object orientation are a matter of 
convention instead of built into the language.
The result is a fragmented ecosystem of libraries, and
different ideas among Lua programmers on how they should use
the language features and how they should structure programs.

The lack of standard policies
is a challenge for the design of a static type system for the Lua
language. The design of Typed Lua is informed by a (mostly
automated) survey of Lua idioms used in a large corpus of Lua
libraries, instead of relying just on the semantics of the
language.

Typed Lua allows statically typed Lua code to coexist and
interact with dynamically typed code. The Typed Lua compiler
warns the programmer about type errors, but always generates
Lua code that runs in unmodified Lua implementations. The
programmer can enjoy some of the benefits of static types even
without converting existing Lua modules to Typed Lua:
a dynamically typed module can export a statically typed
interface, and statically typed users of the module will have
their use of the module checked by the compiler.

Unlike gradual type systems, Typed Lua does not insert runtime
checks between dynamically and statically typed parts of the
program. Unlike some optional type systems, the statically
typed subset of Typed Lua is sound by design, so a later
version of the Typed Lua compiler can swicth to gradual
instead of just optional typing.

We cover the main parts of the design, along with
how they relate to Lua, in Sections~\ref{sec:atomic}
through~\ref{sec:classes}. Section~\ref{sec:review} reviews
related work on mixing static and dynamic typing in the same
language. Finally, Section~\ref{sec:con} gives some
concluding remarks and future extensions for Typed Lua.

\section{Atomic Types and Functions}
\label{sec:atomic}

Lua values can have one of eight {\em tags}: {\em nil}, {\em boolean},
{\em number}, {\em string}, {\em function}, {\em table}, {\em userdata},
and {\em thread}. In this section, we will see how Typed Lua assigns types to
values of the first five.

Figure~\ref{fig:typelang} gives the abstract syntax of Typed Lua
types. Only {\em first-class types} correspond to actual Lua
values; {\em second-class types} correspond to expression lists,
and Typed Lua uses them to type multiple assignment and function
application.

Types are ordered by a subtype relationship, where
any first-class type is a subtype of $\Top$. The rest of the 
subtype relationship is standard: union types are supertypes
of their parts, $\Number$, $\Boolean$, and $\String$ are
supertypes of their respective literal types, function
types are related by contravariance on the input part and
covariance in the output part, table types have width
subtyping, with depth subtyping on $\Const$ fields,
tuple and vararg types are covariant.

The {\em dynamic type} $\Any$ allows dynamically typed code to
interoperate with statically typed code; it is a subtype of 
$\Top$, but
neither a supertype nor a subtype of any other type. We relate 
$\Any$
to other types with the {\em consistency} and {\em consistent-
subtype}
relationships used by gradual type systems~\citep{siek2006gradual,siek2007objects}.
 In practice, we can
pass a value of the dynamic type anytime we want a value of some
other type, and can pass any value where a value of the dynamic type
is expected, but these operations are tracked by the type system,
and the programmer can choose to be warned about them.

\begin{figure}[!ht]
\textbf{Type Language}\\
\dstart
$$
\begin{array}{rlr}
\multicolumn{3}{c}{\textbf{First-class types}} \\
T ::= & \;\; L & \textit{literal types}\\
& | \; B & \textit{base types}\\
& | \; \Top & \textit{top type}\\
& | \; \Any & \textit{dynamic type}\\
& | \; \Self & \textit{self type}\\
& | \; T \cup T & \textit{disjoint union types}\\
& | \; S \rightarrow S & \textit{function types}\\
& | \; \{F, ..., F\} & \textit{table types}\\
& | \; X & \textit{type variables}\\
& | \; \mu X.T & \textit{recursive types}\\
& | \; X_i & \textit{projection types}\\
L ::= & \multicolumn{2}{l}{\False \; | \; \True \; | \; {<}{\it number}{>} \; | \; {<}{\it string}{>} } \\
B ::= & \multicolumn{2}{l}{\Nil \; | \; \Boolean \; | \; \Number \; | \; \String} \\
F ::= & T:T \; | \; \Const \; T:T & \textit{field types} \\
\multicolumn{3}{c}{} \\
\multicolumn{3}{c}{\textbf{Second-class types}} \\
S ::= &  \multicolumn{2}{l}{\; \Void \; | \; V \; | \; V \cup S} \\
V ::= & \;\; T & \\
& | \; T* & \textit{vararg types}\\
& | \; T \times V & \textit{tuple types}
\end{array}
$$
\dend
\caption{Abstract syntax of Typed Lua types}
\label{fig:typelang}
\end{figure}

Typed Lua allows optional type annotations in variable function
declarations. It assigns the dynamic type to any parameter that
does not have a type annotation, but assigns more precise
types to unannotated variables, based on the type of the
expression that gives the initial value of the variable.

In the following example, we use type annotations in a function
declaration but do not use type annotations in the declaration
of a local variable:
\begin{verbatim}
    local function factorial(n: number): number
      if n == 0 then
        return 1
      else
        return n * factorial(n - 1)
      end
    end
    local x = 5
    print(factorial(x))
\end{verbatim}

The compiler assigns the type $\Number$ to the local variable \texttt{x},
and this example compiles without warnings. Typed Lua allows
programmers to combine annotated code with
unannotated code, as we show in the following example:
\begin{verbatim}
    local function abs(n: number)
      if n < 0 then
        return -n
      else
        return n
      end
    end
    local function distance(x, y)
      return abs(x - y)
    end
\end{verbatim}

The compiler assigns the dynamic type $\Any$ to the input
parameters of \texttt{distance} because they do not have type annotations.
Subtracting a value of type $\Any$ from another also yields a value of
type $\Any$ (Lua has operator overloading, so the minus operation
is not guaranteed to return a number in this case), but
consistent subtyping lets us pass a value of type $\Any$ to a
function that expects a $\Number$. 

Even though the return types
of both {\tt abs} and {\tt distance} are not given, the compiler is able to
infer a return type of $\Number$ to both functions, as they are
local and not recursive.

Lua has first-class functions, but they have some peculiarities. First,
the number of arguments passed to a function does not need to
match the function's arity; Lua silently drops extra arguments after
evaluating them, or passes {\tt nil} in place of any missing arguments.
Second, functions can return any number of values, and the number
of values returned may not be statically known. Third, Lua also has
multiple assignment, and the semantics of argument passing match
those of multiple assignment (or vice-versa); calling a function is like
doing a multiple assignment where the left side is the parameter list
and the right side is the argument list. 

Typed Lua uses {\em second-class types} to encode the peculiarities
of argument passing, multiple returns, and multiple assignment. We call
them second-class because these types do not correspond to actual
values and cannot be assigned to variables or parameters: they are an
artifact of the interaction between the type system and the semantics of Lua.

As we can see in Figure \ref{fig:typelang}, a second-class
type in Typed Lua can be a tuple of first-class types
optionally ending in a variadic type, or a union of these
tuples. A variadic type $T*$ is a generator for a
sequence of values of type $T \cup \Nil$. Unions of tuples play
an important part in functions that are overloaded on the
return type, together with {\em projection types}. Both are
explained in the next
section.

In its default mode of operation, Typed Lua always adds a
variadic tail to the parts of a function type if none is 
specified, to match the semantics of Lua function calls. In the 
examples above, the types
of {\tt factorial} and {\tt abs} are actually
$\Number \times \Top * \rightarrow \Number \times \Nil *$,
and the type of {\tt distance} is $\Any \times \Any \times \Top * 
\rightarrow \Number \times \Nil *$.

If we call {\tt abs} with extra arguments, Typed Lua silently
ignores them, as the type signature lets {\tt abs} receive any 
number of extra arguments. If we call {\tt abs} in the right side 
of an assignment to more than one lvalue, Typed Lua checks if the 
first lvalue has a type consistent with $\Number$, and any other 
lvalues need to have a type consistent with $\Nil$.

There is an optional stricter mode of operation where
Typed Lua does not give variadic tails to the parts
of a function type unless the programmer explictly
declares it, and so will also check all function calls
for arity mismatch. 

A variadic type can only appear in the tail position of a tuple,
because Lua takes only the first value of any expression that 
appears
in an expression list that is not in tail position. The following 
example
shows the interaction between multiple returns and expression 
lists:

\begin{verbatim}
    local function multiple()
      return 2, "foo"
    end
    local function sum(x: number, y: number)
      return x + y
    end
    local x, y, z = multiple(), multiple()
    print(sum(multiple(), multiple())
\end{verbatim}

Function {\tt multiple} is
$\Top * \rightarrow \Number \times \String \times \Nil *$,
and {\tt sum} is $\Number \times \Number \times \Top * \rightarrow
\Number \times \Nil *$. In the right side of the multiple assignment,
only the first value produced by the first call to multiple gets used, so
the type of the right side is $\Number \times \Number \times \String \times \Nil*$,
and the types assigned to {\tt x}, {\tt y}, and {\tt z} are respectively $\Number$,
$\Number$, and $\String$. This also means that the call to {\tt sum} compiles
without errors, as the first two components of the tuple are consistent with
the types of the parameters, and the other components are consistent with
$\Top$.

\section{Unions}
\label{sec:unions}

Typed Lua uses union types to encode some common Lua idioms:
optional values, overloading based on the tags of input parameters,
and overloading on the return type of the functions.

Optional values are unions of some type and $\Nil$, and are so
common that Typed Lua uses the {\tt t?} syntax for these unions.
They appear any time a function has optional parameters, and
any time the program reads a value from an array or map.

\begin{verbatim}
    local function message(name: string,
                           greeting: string?)
      local greeting = greeting or "Hello "
      return greeting .. name
    end
    
    print(message("Lua"))
    print(message("Lua", "Hi"))
\end{verbatim}

In this example, the second parameter is optional but, in the
first line of the function, we declare a new variable that is
guaranteed to have type $\String$ instead of $\String \cup \Nil$.
In Lua, any value except {\tt nil} and {\tt false} are ``truthy",
so the short-circuiting {\tt or} operator is a common way of
giving a default value to an optional parameter. Typed Lua
encodes this idiom with a typing rule: if
the left side of {\tt or} has type $T \cup \Nil$ and the right side has type $T$ then the {\tt or} expression has type $T$.

Declaring a new {\tt greeting} variable that shadows the
parameter is not necessary:

\begin{verbatim}
    local function message(name: string, 
                           greeting: string?)
      greeting = greeting or "Hello "
      return greeting .. name
    end
\end{verbatim}

Typed Lua lets the assignment $x = x \ \mathbf{or} \ e$ change
the type of $x$ from $t \cup \Nil$ to $t$,
as long as the type of $e$ is a subtype of $t$, $x$ is local
to the current function, and it is not assigned in
another function. The change only affects the type of $x$
in the remainder of the current scope. In the case of {\tt
greeting}, the assignment on line three changes its type to
$\String$.

Overloaded functions use the {\tt type} function to inspect
the tag of their parameters, and perform different actions
depending on what those tags are. The simplest case overloads
on just a single parameter:

\begin{verbatim}
    local function overload(s1: string, 
                            s2: string|number)
      if type(s2) == "string" then
        return s1 .. s2
      else
        -- string.rep: (string, number) -> string
        return string.rep(s1, s2)
      end
    end
\end{verbatim}

Typed Lua has a small set of type predicates that, when used
over a local variable in a condition, constrain the type of that
variable. The function above uses the {\tt type(x) == "string"}
predicate which constrains the type of $x$ from $T \cup \String$ to
$\String$ when the predicate is true and $T$ otherwise.
This is a simplified form of {\em flow 
typing}~\cite{guha:flow,tobin:occur}. As with {\tt or}, the
variable must be local to the function, and cannot be assigned
to in another function.

The type predicates can only discriminate based on tags, so they
are limited on the kinds of unions that they can discriminate. It
is possible to discriminate a union that combines a table type with
a base type, or a table type with a function type, or a two base types,
but it is not possible to discriminate between two different function
types.

Functions that overload their return types to signal the occurrence
of errors are another common Lua idiom. In this idiom, a function
returns its normal set of return values in case of success but,
if anything fails, returns {\tt nil} as the first value, followed
by an error message or other data describing the error, as in the
following example:

\begin{verbatim}
    local function idiv(d1: number, d2: number):
          (number, number)|(nil, string)
      if d2 == 0 then
        return nil, "division by zero"
      else
        local r = d1 % d2
        local q = (d1 - r)/d2
        return q, r
      end
    end
\end{verbatim}

There is also special syntax for this idiom: we could
annotate the return type of {\tt idiv} with {\tt (number, number)?}
to denote the same union\footnote{The parentheses are always
necessary here: {\tt number?} is {\tt number|nil}, while {\tt (number)?}
is {\tt (number)|(nil, string)}.}. 

The full type of {\tt idiv} is $\Number \times \Number \times
\Top * \rightarrow (\Number \times \Number \times \Nil *) \cup
(\Nil \times \String \times \Nil *)$. A typical client of this
function would use it as follows:

\begin{verbatim}
    local q, r = idiv(n1, n2)
    -- q is number|nil, r is number|string
    if q then
      -- q and r are numbers
    else
      -- r is a string
    end
\end{verbatim}

When Typed Lua encounters a union of tuples in the right side
of an declaration, it stores the the union in a special type
environment with a fresh name and assigns {\em projection types} 
to the variables in the left side of the declaration. If the type
variable is $X$, variable {\tt q} gets type $X_1$ and variable
{\tt r} gets type $X_2$.

If we need to check a projection type
against some other type, we take the union of the corresponding
component in each tuple. But if a variable with a projection
type appears in a type predicate, the predicate discriminates
against all tuples in the union. In the example above,
$X$ is $(\Number \times \Number \times \Nil *) \cup
(\Nil \times \String \times \Nil *)$ outside of the {\tt if}
statement, but $\Number \times \Number \times \Nil *$ in the
{\tt then} block and $\Nil \times \String \times \Nil *$ in
the {\tt else} block.

Notice that we could also discriminate {\tt r} using
{\tt type(r) == "number"} as our predicate, with the same
result. The first form is more succinct, and more idiomatic.
We can also use projection types to write overloaded functions
where the type of a parameter depends on the type of another
parameter.

Assigning to a variable with a projection type is forbidden,
unless the union has been discriminated down to a single tuple,
Unrestricted assignment to these variables would be unsound,
as it could break the dependency relation between the types
in each tuple that is part of the union.

Currently, a limitation of our overloading mechanisms is that
the return type cannot depend on the input types; we cannot
write a function that is guaranteed to return a number if
passed a number and guaranteed to return a string if passed
a string, for example. While intersection types provide
a way to express the type of such a function as $\Number 
\rightarrow \Number \cap \String \rightarrow \String$,
more sophisticated flow typing is needed to actually check
that a function has this type, and we are still working on this
problem.

\section{Tables and Interfaces}
\label{sec:tables}

Tables are the main mechanism that Lua has to build data
structures. They are associative arrays where any value
(except {\tt nil}) can be a key, but with language support
for efficiently using tables as tuples, arrays (dense or sparse),
records, modules, and objects. In this section,
we show how Typed Lua encodes tables as arrays, records, tuples,
and plain maps in its type system.

Typed Lua uses the same framework to represent the different
uses that a Lua table has: {\em table types}. A table type
$\{ t_{1}:u_{1}, \ldots, t_{n}:u_{n}\}$ represents a map
from values of type $t_i$ to values of type $u_i$.

The concrete syntax of Typed Lua has syntax for
common table types. One syntax defines table types for
maps: it is written \texttt{\{ t: u \}},
and maps to the table type $\{t:u\}$.
This table type represents a map that maps values of type
$t$ to values of type $u$.
Another syntax defines table types for arrays:
it is written \texttt{\{ t \}}, and is equivalent to
the table type
$\{\Number:t\}$. A third syntax defines
table types for records:
it is written \texttt{\{ s1: t1, ..., sn: tn \}}, where
each $s_i$ is a literal number, string, or boolean, 
and maps to the table type $\{s_{1}:t_{1}, ..., s_{n}:t_{n}\}$,
where each $s_i$ is the corresponding literal type.

The example below shows how we can define a map from
strings to numbers; the dot syntax for field access
in Lua is actually syntactic sugar for indexing a table
with a string literal:

\begin{verbatim}
    local t: { string: number } = { foo = 1 }
    local x: number = t.foo       -- x gets 1
    local y: number = t["bar"]    -- runtime error
\end{verbatim}

When accessing a map, there is always the possibility that
the key is not there. In Lua, accessing a non-existing key
returns {\tt nil}. Typed Lua is stricter, and raises a runtime
error in this case. To get a map with the behavior of standard
Lua tables, the programmer can use an union:

\begin{verbatim}
    local t: { string: number? } = { foo = 1 }
    local x: number = t.foo       -- compile error
    local y: number = t.bar or 0  -- y gets 0
    local z: number? = t["bar"]   -- z gets nil
\end{verbatim}

Now the Typed Lua compiler will complain about the assignment
on line two. The following example shows how we can declare
an array:
\begin{verbatim}
    local days: { string } = { "Sunday", "Monday",
      "Tuesday", "Wednesday", "Thursday",
      "Friday", "Saturday" }
    local x = days[1]          -- x gets "Sunday"
    local y = days[8]          -- runtime error
\end{verbatim}

Notice that we have the same strictness with missing
elements, unless the type of the elements has {\tt nil}
as a possible value.

While this runtime check is an instance where the semantics
of a Typed Lua program deviates from the semantics of
plain Lua, the alternatives would be to make all
arrays and maps have an element type that includes {\tt nil},
and either make the programmer narrow the type with {\tt or}
or an {\tt if} statement, making using the elements more
incovenient, or make {\tt nil} a subtype of every type,
reducing the amount of type safety in the system. Notice 
that this does not change the semantics of dynamically
typed programs, as the runtime checks are only added when
the table has a strict static type.

If we want to declare a tuple, we can leave the
variable declaration unannotated and let Typed Lua assign
a more specific table type to the variable.
If we remove the annotation in the previous example, 
the compiler assigns the following table type to \texttt{days}:
\begin{align*}
\{{1:\String},\;{2:\String},\;{3:\String},\;{4:\String},\;\\
{5:\String},\;{6:\String},\;{7:\String}\}
\end{align*}

This type is not a subtype of $\{\Number:\String\}$, nor
is $\{\Number:\String\}$ a subtype of $\{\Number:
\String\cup\Nil\}$, because in both cases the subtype
relationship would be unsound. In the first case,
a table type with the same fields as the type above,
plus $8: \Number$, is a subtype of the table type
above, so would also be a subtype of $\{\Number:\String\}$,
which is clearly unsound. In the second case, covariance
in the type of mutable fields is also unsound, for the
same reason as the unsoundness of array covariance.

While the record type above, $\{\Number:\String\}$, and
$\{\Number: \String\cup\Nil\}$ are disjoint, all three
types are valid for the table constructed in the example.
Typed Lua actually assigns different types to a table constructor
expression depending on the context where it is used.

Finally, the next example shows how we can declare a record:

\begin{verbatim}
    local person: { "firstname": string,
                    "lastname": string } =
      { firstname = "Lou", lastname = "Reed" } 
\end{verbatim}

We could leave the type annotation out, and Typed Lua would
assign the same type to {\tt person}.

As records get bigger, and types of record fields get more
complicated, writing table types can be unwieldy, so Typed
Lua has {\em interfaces} as syntactic sugar for record types:

\begin{verbatim}
    local interface Person
      firstname: string
      lastname: string
    end
\end{verbatim}

The declaration above declares {\tt Person} as an alias to
the record type $\{$``firstname": $\String$,
 ``lastname": $\String\}$ in
the remainder of the current scope. We can now use {\tt Person} in
type declarations:

\begin{verbatim}
    local function greet(person: Person)
      return "Hello, " .. person.firstname ..
             " " .. person.lastname
    end
    local user1 = { firstname = "Lewis",
                    middlename = "Allan",
                    lastname = "Reed" }
    local user2 = { firstname = "Lou" }
    local user3 = { lastname = "Reed",
                    firstname = "Lou" }
    local user4 = { "Lou", "Reed" }
    print(greeter(user1)) -- Hello, Lewis Reed
    print(greeter(user2)) -- Error
    print(greeter(user3)) -- Hello, Lou Reed
    print(greeter(user4)) -- Error
\end{verbatim}

If our record type has fields that can be {\tt nil}, we need
to use an explicit type declaration when declaring a
variable of this record type, as the following example shows:

\begin{verbatim}
    local interface Person
      firstname: string
      middlename: string?
      lastname: string
    end
    local user1: Person = { firstname = "Lewis",
                            middlename = "Allan",
                            lastname = "Reed" }
    local user2: Person = { lastname = "Reed",
                            firstname = "Lou" }
\end{verbatim}

We need an explicit type declaration because neither of
the types that the Typed Lua compiler assigns to the
table constructors above is a subtype of $\{$``firstname":
$\String$, ``middlename": $\String \cup \Nil$, ``lastname":
$\String\}$, the type that {\tt Person} describes.

We can also use interfaces to define recursive types:

\begin{verbatim}
    local interface Element
      info: number
      next: Element?
    end
\end{verbatim}

It is common in Lua programs to build a record incrementally,
starting with an empty table, as in the following example:

\begin{verbatim}
    local person = {}
    person.firstname = "Lou"
    person.lastname = "Reed"
\end{verbatim}

Ideally, we want the type of {\tt person} to change as the
table gets built, from $\{\}$ to $\{$``firstname": $\String\}$
and finally to $\{$``firstname": $\String$, ``lastname":
 $\String\}$. This is tricker than the type change introduced
by assignment that we saw in Section 2.2, as what is changing
is not just the type of the variable {\tt person} but the
type of the value that {\tt person} points to. This is safe
in the example above, but not in the example below:

\begin{verbatim}
    local bogus = { firstname = 1 }
    local person: {} = bogus
    person.firstname = "Lou"
    person.lastname = "Reed"
\end{verbatim}

The assignment on line two is perfectly legal, as the type
of {\tt bogus} is a subtype of $\{\}$. But changing the type
of {\tt person} would be unsound: {\tt person.firstname} is
now a $\String$, but {\tt bogus.firstname} is still typed
as a $\Number$. We do not even need to declare a type for
aliasing to be a problem: 

\begin{verbatim}
    local person = {}
    local bogus = person
    bogus.firstname = 1
    person.firstname = "Lou"
    person.lastname = "Reed"
\end{verbatim}

Taken individually, the changes to the type of the two
variables look fine, but aliasing makes one of them unsound.
The location of the change also matters, as the next example
shows:

\begin{verbatim}
    local person = {}
    local bogus = { firstname = 1 }
    do
      person.firstname = 1
      bogus = person
    end
    do
      person.firstname = "Lou"
    end
    -- bogus.firstname is now "Lou"
\end{verbatim}

The initial type of {\tt person} in all of these examples is
$\{\}$, but the {\em origin} of this type judgment matters
on whether it is sound to allow a change to the type of
{\tt person} or not, even if the change is always towards
a subtype of the current type.

Typed Lua tags a variable with a table type as either
{\em open} or {\em closed}. If a variable gets its type
from a table constructor then it is open, otherwise it is
always closed. The type of an open variable may change
by field assignment, subject to three restrictions: the
variable must be local to the current block, the new type
must be a subtype of the old type, and the variable
cannot have been assigned to in another function.

An variable with an open table type may be aliased, but
these aliases are not open. Any mutation on these aliases
is not a problem, as the type of the original reference
can only change towards a subtype. For mutable fields
this means that the type of the field cannot change
once it is added to the type of the table.

Using a variable with an open type can also trigger a
type change, if the type of the missing fields has
{\tt nil} as a possible value. This lets the programmer
incrementally create an instance of an interface with
an optional type:

\begin{verbatim}
    local interface Person
      firstname: string
      middlename: string?
      lastname: string
    end
    local user = {}
    user.firstname = "Lou"
    user.lastname = "Reed"
    local person: Person = user
\end{verbatim}

Table types are the foundation for modules and objects in
Typed Lua. Type changes triggered by field assignment are
also an important part of Typed Lua's support for the
idiomatic definition of Lua modules, which are the
subject of the next section.

\section{Modules}
\label{sec:modules}

Lua's module system, like other parts of the language, is
a matter of convention. When Lua first needs to load a
module, it executes the module's source file as a function;
the value that this function returns is the module, and Lua
caches it for future loads. While a module can be any
Lua value, most modules are tables where the fields
of the table are functions and other values that the
module exports.

The modules that we surveyed build this table using
three distinct styles. In the first style, the module's
source file ends
with a {\tt return} statement that uses a table constructor
with the exported members. In the second style, the
module declares an empty table in the beginning of its
source file, adds exported members to this table
throughout the module, and returns this table at the end.

These two styles are straightforward for Typed Lua, which
can just take the type of the first value that the module
returns and use it as the type of the module.

In the third style, which has been deprecated in the current
version of Lua, a module begins with a call to the {\tt module}
function. This function installs a fresh table as the
global environment for the rest of the module, so any
assignments to global variables are field assignments to
this table. The {\tt module} function also sets an {\tt \_M}
field in this table as a circular reference to the table
itself, so the module can end with {\tt return \_M}, but
this explicit return is not necessary.

While this style has been deprecated, our survey indicated
that around a third of Lua modules in a popular module
repository still use this style, so Typed Lua also
supports this style: it treats accesses to global
variables as field accesses to an open table
in the top-level scope.

\section{Objects and Classes}
\label{sec:classes}

Lua's built-in support for object-oriented programming is
minimal. The basic mechanism is the {\tt :} syntactic
sugar for method calls and method declarations.
The Lua compiler translates {\tt obj:method(args)}
to an operation that evaluates {\tt obj}, looks
for a field named ``method" in the result, then calls
it with the result of evaluating {\tt obj} as the
first argument, followed by the result of evaluating
the argument list in the original expression.

We can use table types and the {\em self type}
to represent objects,
and Typed Lua has syntactic sugar to make defining these
types easier:

\begin{verbatim}
    interface Shape
      x, y: number
      const move: (dx: number, dy: number) => ()
    end
\end{verbatim}

The double arrow in the type of the two methods is
syntactic sugar for having a first parameter named
{\tt self} with type $\Self$. The
{\tt const} qualifier is necessary for covariance
in the types of the methods, and to make subtyping
among object types work.

While {\tt :} is syntactic sugar in plain Lua, Typed Lua
uses it to type-check method calls, to bind any ocurrence
of the $\Self$ type in the type of the method to the
receiver. Indexing a method but not immediately calling
it with the correct receiver is a compile-time error.

There is still the matter of how to construct a value with
the object type above. The following example shows one
way:

\begin{verbatim}
    local shape = { x = 0, y = 0 }
    const function shape:move(dx: number, 
                              dy: number)
      self.x = self.x + dx
      self.y = self.y + dy
    end
\end{verbatim}

The {\tt :} syntactic sugar that the example uses also comes
from Lua, and assigns a function to the field with a first
parameter named {\tt self}, plus any other parameters. Typed
Lua adds the {\tt const} annotation, and gives type $\Self$
to the implicit parameter {\tt self}. 

Lua has a mechanism for Self-like (or JavaScript-like)
 delegation of missing
fields in a table. After {\tt setmetatable(t1,
\{ \_\_index = t2 \})}, Lua looks up in {\tt t2} any missing
fields of {\tt t1}. Lua programmers often use this mechanism
to simulate classes, as in the following example:

\begin{verbatim}
    local Shape = { x = 0, y = 0 }
    const function Shape:new(x: number, y: number)
      local s = setmetatable({},
                             { __index = self })
      s.x = x
      s.y = y
      return s
    end
    const function Shape:move(dx: number,
                              dy: number)
      self.x = self.x + dx
      self.y = self.y + dy
    end
    local shape1 = Shape:new(0, 5)
    local shape2: Shape = Shape:new(10, 10)
\end{verbatim} 

In the last line of the example, notice how we can refer
to {\tt Shape} in the type annotation, as a shortcut
to the table type that Typed Lua has assigned
to this variable.

Typed Lua assigns the type
$\Self \times \Number \times \Number \times
\Top* \rightarrow \Self \times \Nil*$ to
{\tt new}. In a {\tt setmetatable} expression,
if the type $t_1$ of the first operand is a supertype
of the type $t_2$ of the second operand's {\_\_index} field,
it changes the type of the first operand to $t_2$, 
so the local variable {\tt s} 
has the same type as {\tt self}.

We can also use {\tt setmetatable} to simulate single
inheritance, as {\tt setmetatable} on a table constructor
assigned to a fresh local variable gives this variable
an open type, which lets us add new methods and override
existing ones:

\begin{verbatim}
    local Circle = setmetatable({},
                                { __index = Shape })
    Circle.radius = 0
    const function Circle:new(x: number,
                              y: number,
                              radius: value)
      local c = setmetatable(Shape:new(x, y),
                             { __index = self })
      c.radius = tonumber(radius)
      return c
    end
    const function Circle:area()
      return math.pi * self.radius * self.radius
    end
\end{verbatim} 

In the first line of the redefinition of {\tt new},
notice how we can call {\tt Shape}'s constructor inside
the overridden constructor. A limitation of this class
system is that the overridden constructor must be a
subtype of the original constructor, so the type of
{\tt radius} has to be very permissive.

If we erase all type and {\tt const} annotations, the
two examples above are valid Lua code, with the same
semantics as the Typed Lua code.

The current version of Typed Lua does not have a polymorphic
type system, so programmers currently cannot hide the
calls to {\tt setmetatable} behind nicer abstractions, as
some Lua libraries do. A few Lua programs also use
other features of {\tt setmetatable} which are currently
not typeable, such as operator overloading.

\section{Related Work}
\label{sec:review}

Common LISP introduced optional type annotations in the early
eighties~\citep{steele1982ocl}, but they were optimization
hints to the compiler instead of types for static checking.
These annotations were unsafe, and could crash the program
when wrong.

\citet{abadi1989dts} used tagged pairs and explicit
injection and projection operations (coercions) to embed dynamic
typing in the simply-typed lambda calculus. Dynamically-typed
values had a {\tt Dynamic} static type. \citet{thatte1990qst}
removes the necessity of explicit coercions with a system
that automatically inserts coercions and checks them for
correctness.

\textit{Soft typing}~\citep{cartwright1991soft} starts
with a dynamically-typed language, and layers a static
type system with a sophisticated global type inference
algorithm on top, to try to find errors in programs
without needing to rewrite them. In cases where an error
may or may not be present, it warns the programmer and
inserts a runtime check. One problem with the soft
typing approach was the complexity of the inferred
types, leading to errors that were difficult to understand
and fix.

While the goal of soft typing is catching errors,
\textit{dynamic typing}~\citep{henglein1994dts} is
another approach for optimizing dynamically-typed programs.
First the program is translated to a program that uses
a {\tt Dynamic} type and explicit coercions and runtime checks,
then a static analysis removes some  of these coercions and
checks.

Instead of trying to add static checking to a dynamic
language, \citet{findler2002chf} enhances
the dynamic checks with the possibility of {\em contracts}
that give assertions about the input and output of (possibly
higher-order) functions. In case of higher-order functions,
the actual failing check can be far away from the actual
source of the error, so contracts can also add {\em blame
annotations} to values as a way to trace failures back to
the source.

Strongtalk \citep{bracha1993strongtalk,bracha1996strongtalk} is
an optionally-typed version of Smalltalk. It has a
polymorphic structural type system that programmers can use
to annotate Smalltalk programs, but type annotations can be
left out; unannotated terms are dynamically typed, and can
be cast to any static type. The interaction of the dynamic
type with the rest of the type system is unsound, so
Strongtalk uses the dynamically-checked semantics of
Smalltalk when executing programs, even if the programs
are statically typed.

Pluggable type systems~\citep{bracha2004pluggable} generalize
the idea of Strongtalk, to have type systems that can be
layered on top of a dynamic language without influencing
its runtime semantics. These systems can be unsound in
themselves, or in their interaction with the dynamically
typed part of the language, without sacrificing runtime
safety, as the semantics of the language catch any
runtime errors caused by an unsound type system.

Dart~\citep{dart} and TypeScript~\citep{typescript} are
two recent examples of languages with optional type systems
in this style. Dart is an object-oriented language with
a semantics that is similar to Smalltalk's, while TypeScript
is an object-oriented extension of JavaScript. Dart has a
nominal type system, while TypeScript has a structural one,
but both type systems are designed on purpose with unsound
parts (such as covariant arrays in case of Dart, and
covariant function return types in case of TypeScript) to
increase programmer convenience. The interaction of
statically and dynamically typed code is also unsound.

\citet{tobin-hochstadt2006ims} shows how programs
in the untyped lambda calculus can be incrementally
translated to the simply typed lambda calculus, using
contracts to guarantee that the untyped part cannot
cause errors in the typed part, as a model on how
scripts in a dynamically typed language can be
incrementally translated to a statically typed
language. This approach has been realized in the
Typed Scheme (later Typed Racket) language~\citep{tobin-hochstadt2008ts,tobin:occur}. 

Gradual typing~\citep{siek2006gradual} combines the
optional type annotations of optional and pluggable
type systems with higher-order contracts. The gradual
type system is the simply typed lambda calculus enriched
with a {\em dynamic type} $?$, where a value with the
dynamic type can assume any type, and vice-versa.
A dynamic value that assumes a static type generates
a runtime check for a first-order value, or a wrapper
for a higher-order value. A static value that assumes
the dynamic type is tagged. The whole system is sound:
any runtime errors in a well-typed program must happen
in the dynamic parts. 

While both gradual typing and the approach of
\citep{tobin-hochstadt2006ims} have the goal of
having a sound interaction of dynamically and statically
typed code, they differ in the granularity, with the
gradual typing providing a finer-grained transition from
dynamically typed to statically typed code.

Gradual typing has been combined with subtyping in
a simple object calculus~\citet{siek2007objects}, with
a nominal, polymorphic type system~\citet{ina:gradual},
with a row polymorphism~\citet{tobin:gradual},
and with the polymorphic lambda calculus~\citet{ahmed2011bfa}.
Gradual typing also adopted blame tracking from
higher-order
contracts~\citet{siek2010blame,ahmed2001bfa,wadler2009wpc}.

While Typed Racket is the first fully-featured programming
language with a sound mixture of dynamic and static typing,
Gradualtalk~\citep{allende2013gts} is the first 
fully-featured language with a fine-grained gradual type
system. In Gradualtalk's case, the extra runtime checks
needed by gradual typing impose a big runtime
cost~\citep{allende2013cis}, and the programmer has the
option of turning off these checks, and downgrading
Gradualtalk to an optional type system.

Grace~\citet{black2013sg} is an object-oriented language
with optional typing. Grace is not an dynamically typed
language that has been extended with an optional type system,
but a language that has been designed from the ground up
to have both static and dynamic typing. \citet{homer2013modules}
explores some useful patterns that derive from Grace's
use of objects as modules and its brand of optional
structural typing, which can also be expressed with
Typed Lua's modules as tables.

The main feature of Typed Racket's type system is
{\em occurrence typing}~\citep{tobin:occur}, where
the type of a variable can be refined through tests
using predicates that check the runtime type of
the variable. As these types of checks are common in
other languages, but ocurrence typing is not sound
in the presence of mutation, related systems have
appeared~\citep{guha:flow,winther:flow,pearce:whiley}.

Tidal Lock~\citep{tidallock} is a prototype of another optional
type system for Lua. Compared with Typed Lua, Tidal Lock
has richer record types, and a more robust system of
incremental evolution of these record types, but it lacks
unions and support for expressing objects and classes.

\section{Conclusion}
\label{sec:con}

This paper presented the initial design of Typed Lua,
an optional type system for the Lua scripting language.
While Lua shares several characteristics with other
dynamic languages such as JavaScript, so Typed Lua's
type system has several parts in common with optional
and gradual type systems for dynamic languages,
Lua also has some uncommon features. 

These language features demanded features in the type
system that are not present in the type systems of these
other languages: functions with flexible arity and
their interaction with multiple assignment, functions
that are overloaded on the {\em number} of values
they return, and how to support the idiomatic way of
handling their return values, and incremental evolution
of record and object types.

Typed Lua is a work in progress: while the design covers
most of the Lua language as it is actually used by Lua
programmers, there are missing parts. A major omission
of the current version is the lack of polymorphic function
and table types, which will be present in the next version.
The {\tt setmetatable} primitive that we briefly saw in
Section~6 has several other uses that are currently
outside of the scope of the type system: operator
overloading, proxies, changing the behavior of built-in
types, multiple inheritance, etc. Finally, Lua has
one-shot delimited continuations~\citep{james2011yield}
in the form of {\em couroutines}~\citep{moura:coro},
which are also currently ignored by the type system.

Unlike some of the other optional type systems,
the type system of Typed Lua does not have deliberately
unsound parts, but we still do not have proofs that the
novel parts of our type system are sound.

Finally, while we our survey of Lua modules gives some
confidence that the design of Typed Lua is useful for
the majority of Lua programs, we have to validate the
design by using Typed Lua to to type a representative 
sample of Lua modules.

\bibliographystyle{abbrvnat}
\bibliography{typedlua}

\end{document}


\chapter{The type system}
\label{chap:system}

In the previous chapter we presented an informal overview of Typed Lua.
We showed that programmers can use Typed Lua to combine static and dynamic
typing in the same code, and it allows them to incrementally migrate from
dynamic to static typing.
This is a benefit to programmers that use dynamically typed languages
to build large applications, as static types detect many bugs
during the development phase, and also provide better documentation.

In this chapter we present the abstract syntax of Typed Lua types,
the subtyping rules, and the most interesting typing rules.
Besides its practical contributions, Typed Lua also has some interesting
contributions to the field of optional type systems for scripting
languages.
They are novel type system features that let Typed Lua cover several Lua idioms
and features, such as refinement of tables, multiple assignment, and multiple return values.

\section{Types}
\label{sec:types}

Typed Lua includes types that can appear in annotations and some
special types that cannot appear in annotations, though they
play special role in type checking some Lua idioms and handling flow typing.
Special types cannot appear in annotations because they do not have
a corresponding concrete syntax, so we also isolated them in the
formalization to make the implementation easier.
First we present the types that compose the type language of our
type system, and then we present the special types, indicating
why they are necessary.

\begin{figure*}[!ht]
\textbf{Type Language}\\
\dstart
$$
\begin{array}{rlr}
F ::= & & \textsc{first-level types:}\\
& \;\; L & \textit{literal types}\\
& | \; B & \textit{base types}\\
& | \; \Nil & \textit{nil type}\\
& | \; \Value & \textit{top type}\\
& | \; \Any & \textit{dynamic type}\\
& | \; \Self & \textit{self type}\\
& | \; F_{1} \cup F_{2} & \textit{union types}\\
& | \; S_{1} \rightarrow S_{2} & \textit{function types}\\
& | \; \{\overline{F{:}V}\}_{unique|open|fixed|closed} & \textit{table types}\\
& |\; x & \textit{type variables}\\
& | \; \mu x.F & \textit{recursive types}\\
%\multicolumn{3}{c}{}\\
L ::= & & \textsc{{\small literal types:}}\\
& \;\; \False \; | \; \True \; | \; {\it int} \; | \; {\it float} \; | \; {\it string} &\\
%\multicolumn{3}{c}{}\\
B ::= & & \textsc{{\small base types:}}\\
& \;\; \Boolean \; | \; \Integer \; | \; \Number \; | \; \String &\\
%\multicolumn{3}{c}{}\\
V ::= & & \textsc{{\small value types:}}\\
& \;\; F \; | \; \Const \; F &\\
%\multicolumn{3}{c}{}\\
S ::= & & \textsc{second-level types:}\\
& \;\; P & \textit{tuple types}\\
& | \; S_{1} \sqcup S_{2} & \textit{unions of tuple types}\\
%\multicolumn{3}{c}{}\\
P ::= & & \textsc{{\small tuple types:}}\\
& \;\; F{*} & \textit{variadic types}\\
& | \; F \times P & \textit{pair types}
\end{array}
$$
\dend
\caption{The abstract syntax of Typed Lua types}
\label{fig:typelang}
\end{figure*}

Figure \ref{fig:typelang} presents the abstract syntax of
Typed Lua types.
Typed Lua splits types into two categories:
\emph{first-level types} and \emph{second-level types}.
First-level types represent first-class Lua values and
second-level types represent tuples of values that appear in the
input and output of functions.
First-level types include literal types, base types, the type $\Nil$,
the top type $\Value$, the dynamic type $\Any$, the type $\Self$,
union types, function types, table types, and recursive types.
Second-level types include tuple types and unions of tuple types.
Tuple types include variadic types and pair types.
Types are ordered by a subtype relationship that we introduce
in Section \ref{sec:subtyping}, so Lua values may belong to
several distinct types.

Literal types represent the type of literal values.
They can be the boolean values $\False$ and $\True$,
an integer value, a floating point value, or a string value.
We will see that literal types are important in our treatment of
table types as records.

Typed Lua includes four base types: $\Boolean$, $\Integer$, $\Number$, and $\String$.
The base types $\Boolean$ and $\String$ represent the values that
Lua tags as \texttt{boolean} and \texttt{string} during run-time.
Lua 5.3 introduced two internal representations to the tag \texttt{number}:
\texttt{integer} for integer numbers and \texttt{float} for real numbers.
Lua does automatic promotion of \texttt{integer} values to \texttt{float}
values as needed.
We introduced the base type $\Number$ to represent \texttt{float} values,
and the base type $\Integer$ to represent \texttt{integer} values.
In the next section we will show that $\Integer$ is a subtype of $\Number$.
This allows programmers to keep using \texttt{integer} values where
\texttt{float} values are expected.

The type $\Nil$ is the type of \texttt{nil}, the value that Lua uses for
undefined variables, missing parameters, and missing table keys.

The type $\Value$ is the top type, which represents any Lua value.
In Section \ref{sec:rules} we will show that this type,
along with variadic types, helps the type system to drop extra values
on assignments and function calls, thus preserving the
semantics of Lua in these cases.

Typed Lua includes the dynamic type $\Any$ for allowing programmers
to mix static and dynamic typing.

Typed Lua uses the type $\Self$ to represent the \emph{receiver}
in object-oriented method definitions and method calls.
As we mentioned in Section \ref{sec:oop}, we need the type
$\Self$ to prevent programs from indexing a method without
calling it with the correct receiver.

Union types $F_{1} \cup F_{2}$ represent data types that can hold a value
of two different types.

Function types have the form $S_{1} \rightarrow S_{2}$ and represent Lua functions,
where $S$ is a second-level type.

Second-level types are either tuple types or unions of tuple types.
Tuple types are tuples of first-level types that end with a variadic type.
Typed Lua needs second-level types because tuples are not first-class
values in Lua, only appearing on argument passing, multiple returns,
and multiple assignments.
A variadic type $F{*}$ represents a sequence of values of type $F \cup \Nil$;
it is the type of a vararg expression.
Second-level types include unions of tuples because Lua programs
usually overload the return type of functions to denote error,
as we mentioned in Section \ref{sec:statistics}.
For clarity, we use the symbol $\sqcup$ to represent the union between
two different tuple types.
Note that $\cup$ represents the union between two first-level types,
while $\sqcup$ represents the union between two tuple types.

Back to first-level types, table types represent the various forms
that Lua tables can take.
The syntactical form of table types is $\{\overline{F{:}V}\}_{unique|open|fixed|closed}$,
where the notation $\overline{F{:}V}$ denotes the list $F_{1}{:}V_{1}, ..., F_{n}{:}V_{n}$.
Each $V_{i}$ represents the type of the value that table keys of type $F_{i}$ map to.
Value types represent mutable fields by default, but we can use the
$\Const$ type to make them represent immutable fields.
Making a field $\Const$ does not guarantee that its value cannot change,
as the table may have aliases with a non-$\Const$ type for that field.
Typed Lua needs immutable fields to enable depth subtyping between table types.

\label{def:tabletype}
We also use the tags \emph{unique}, \emph{open}, \emph{fixed}, and \emph{closed}
to classify table types.
The tag \emph{closed} represents table types that do not provide
any guarantees about keys with types not listed in the table type.
In particular, in the concrete syntax, type annotations, interface
declarations, and userdata declarations always describe \emph{closed} table types.
The tags \emph{unique}, \emph{open}, and \emph{fixed} represent
tables with no keys that do not inhabit one of the table's key types,
but with different guarantees about the reference to a value of
that type.
A reference to an \emph{unique} table is guaranteed to point
to a table that has no other references to it.
In particular, the type of the table constructor has this tag,
as it allows greater flexibility in reinterpreting its type.
A reference to an \emph{open} table is guaranteed to have only
\emph{closed} references pointing to the same table,
a guarantee that still lets the type system reinterpret the type
of a reference, but with more restrictions.
A reference to a \emph{fixed} table can have any number of
\emph{fixed} or \emph{closed} references point to it,
so its type cannot change anymore.
In particular, the type of a class has this tag in our type system.

A \emph{fixed} table type guarantees that there are
no keys with a type that is not one of its key types.
Even though this guarantee allows type-safe iteration on
\emph{fixed} table types, it forbids width subtyping that
is necessary for object-oriented programming, so \emph{closed}
table types remove this guarantee to allow width subytping
between other table types and \emph{closed} table types.
This means that objects have \emph{closed} table types,
while their classes have \emph{fixed} table types.

Any table type has to be \emph{well-formed}.
Informally, a table type is well-formed if key types do not overlap.
In Section \ref{sec:rules} we formalize the definition of well-formed table types.
We delay the proper formalization of well-formed table types because we use
consistent-subtyping in this formalization.

Recursive types have the form $\mu x.F$,
where $F$ is a first-level type that $x$ represents.
For instance, $\mu x.\{``info":\Integer, ``next":x \;\cup\; \Nil\}_{closed}$
is a type for singly-linked lists of integers.
In Section \ref{sec:alias} we mentioned that we can use the following
interface declaration as an alias to this type:
\begin{verbatim}
    local interface Element
      info:integer
      next:Element?
    end
\end{verbatim}

With recursive types we finish the discussion about Typed Lua types,
and we begin the discussion about special types.

\begin{figure*}[!ht]
\textbf{Special Types}\\
\dstart
$$
\begin{array}{rlr}
T ::= & & \textsc{expression types:}\\
& \;\; F & \textit{first-level types}\\
& | \; \phi(F_{1},F_{2}) & \textit{filter types}\\
& | \; \pi_{i}^{x} & \textit{projection types}\\
%\multicolumn{3}{c}{}\\
E ::= & & \textsc{expression list types:}\\
& \;\; T{*} & \textit{variadic types}\\
& | \; T \times E & \textit{pair types}\\
%\multicolumn{3}{c}{}\\
R ::= & & \textsc{filter result types:}\\
& \;\; \mathbf{void} & \textit{void type}\\
& |\; F & \textit{first-level types}
\end{array}
$$
\dend
\caption{The special types used by Typed Lua}
\label{fig:specialtypes}
\end{figure*}

Figure \ref{fig:specialtypes} presents the special types that
Typed Lua includes for typing some Lua idioms and flow typing.
Typed Lua splits special types into three categories:
\emph{expression types}, \emph{expression list types}, and
\emph{filter result types}.
Expression types represent the type of expressions in the type system,
which can be any first-level type, a filter type, or a projection type.
Expression list types represent the type of lists of expressions,
which are tuples of expression types that end with a variadic type.
Filter result types are either the type $\Void$ or a first-level type.
In Section \ref{sec:fap} we will show that Typed Lua uses filter types
in flow typing, being the type $\Void$ a type with no values, used by the
type system as a way to detect branches that are unreachable due to flow typing.

\label{def:filtertype}
Typed Lua includes filter types as a way to discriminate the type of local
variables inside conditions.
Our type system uses filter types to formalize the \texttt{type} predicates
that we mentioned in Section \ref{sec:unions}.
This means that \texttt{type} predicates use filter types of the form
$\phi(F_{1},F_{2})$ to discriminate local variables that are bound to
union types.
In a filter type $\phi(F_{1},T_{2})$, $F_{1}$ is the original type and
$F_{2}$ is the discriminated type.

\label{def:projectiontype}
Typed Lua includes projection types as a way to project
unions of tuple types into unions of first-level types.
In Section \ref{sec:fap} we will show in more detail how our type system
uses them as a mechanism for handling unions of tuple types,
when they appear in the right-hand side of the declaration of local variables,
as we mentioned in Section \ref{sec:unions}.
We also show how this feature allows our type system to constrain
the type of a local variable that depends on the type of another local variable.

\section{Subtyping}
\label{sec:subtyping}

Our type system uses subtyping \cite{cardelli1984smi,abadi1996to} to order
types and consistent-subtyping \cite{siek2007objects,siek2013mutable}
to allow the interaction between statically and dynamically typed code.
We explain the subtyping and consistent-subtyping rules throughout this section.
However, we focus the discussion on the definition of subtyping because,
as we mentioned in Section \ref{sec:gradual}, we can combine the
consistency and subtyping relations to achieve consistent-subtyping.
The differences between subtyping and consistent-subtyping are the way
they handle the dynamic type, and the fact that subtyping is transitive,
but consistent-subtyping is not.

We present the subtyping rules as a deduction system for the
subtyping relation $\senv \vdash T_{1} \subtype T_{2}$,
where $T_{1}$ and $T_{2}$ are two types of the same kind.
The variable $\senv$ is a set of pairs of recursion variables.
We need this set to record the hypotheses that we assume when checking
recursive types.

The subtyping rules for literal types and base types include the rules
for defining that literal types are subtypes of their respective base types,
and that $\Integer$ is a subtype of $\Number$:
\[
\begin{array}{c}
\begin{array}{c}
\mylabel{S-FALSE}\\
\senv \vdash \False \subtype \Boolean
\end{array}
\;
\begin{array}{c}
\mylabel{S-TRUE}\\
\senv \vdash \True \subtype \Boolean
\end{array}
\;
\begin{array}{c}
\mylabel{S-STRING}\\
\senv \vdash {\it string} \subtype \String
\end{array}
\\ \\
\begin{array}{c}
\mylabel{S-INT1}\\
\senv \vdash {\it int} \subtype \Integer
\end{array}
\;
\begin{array}{c}
\mylabel{S-INT2}\\
\senv \vdash {\it int} \subtype \Number
\end{array}
\;
\begin{array}{c}
\mylabel{S-FLOAT}\\
\senv \vdash {\it float} \subtype \Number
\end{array}
\\ \\
\begin{array}{c}
\mylabel{S-INTEGER}\\
\senv \vdash \Integer \subtype \Number
\end{array}
\end{array}
\]

Subtyping is reflexive and transitive;
therefore, we could have omitted the rule \textsc{S-INT2}.
More precisely, we could have defined a transitive rule for first-level
types instead of defining specific rules for transitive cases.
For instance, a transitive rule would allow us to derive that
\[
\dfrac{\senv \vdash 1 \subtype \Integer \;\;\;
       \senv \vdash \Integer \subtype \Number}
      {\senv \vdash 1 \subtype \Number}
\]

However, we are using the subtyping rules as the template for defining
the consistent-subtyping rules, and consistent-subtyping is not
transitive.
More precisely, we want the subtyping and consistent-subtyping rules
to differ only in the way they handle the dynamic type.
Thus, we define the subtyping rules using an algorithmic approach
that is close to the implementation, as this approach allows us to use
subtyping to easily formalize consistent-subtyping.

Our type system includes the top type $\Value$,
so any first-level type is a subtype of $\Value$:
\[
\begin{array}{c}
\mylabel{S-VALUE}\\
\senv \vdash F \subtype \Value
\end{array}
\]

Many programming languages include a bottom type to represent
an empty value that programmers can use as a default expression,
and we could have used the type $\Nil$ for this role.
However, making $\Nil$ the bottom type would lead to several expressions
that would pass the type checker, but that would fail during run-time
in the presence of a \texttt{nil} value.
Thus, our type system does not have a bottom type, and $\Nil$ is a
subtype only of itself and of $\Value$.

Another type that is only a subtype of itself and of the type $\Value$
is the type $\Self$.

The subtyping rules for union types are standard:
\[
\begin{array}{c}
\begin{array}{c}
\mylabel{S-UNION1}\\
\dfrac{\senv \vdash F_{1} \subtype F \;\;\;
       \senv \vdash F_{2} \subtype F}
      {\senv \vdash F_{1} \cup F_{2} \subtype F}
\end{array}
\;
\begin{array}{c}
\mylabel{S-UNION2}\\
\dfrac{\senv \vdash F \subtype F_{1}}
      {\senv \vdash F \subtype F_{1} \cup F_{2}}
\end{array}
\;
\begin{array}{c}
\mylabel{S-UNION3}\\
\dfrac{\senv \vdash F \subtype F_{2}}
      {\senv \vdash F \subtype F_{1} \cup F_{2}}
\end{array}
\end{array}
\]

The first rule shows that a union type $F_{1} \cup F_{2}$
is a subtype of $F$ if both $F_{1}$ and $F_{2}$ are subtypes
of $F$;
and the other rules show that a type $F$ is a subtype
of a union type $F_{1} \cup F_{2}$ if $F$ is a subtype of
either $F_{1}$ or $F_{2}$.

The subtyping rule for function types is also standard:
\[
\begin{array}{c}
\mylabel{S-FUNCTION}\\
\dfrac{\senv \vdash S_{3} \subtype S_{1} \;\;\;
       \senv \vdash S_{2} \subtype S_{4}}
      {\senv \vdash S_{1} \rightarrow S_{2} \subtype S_{3} \rightarrow S_{4}}
\end{array}
\]

The rule \textsc{S-FUNCTION} shows that subtyping between
function types is contravariant on the type of the parameter list
and covariant on the return type.
In the previous section we explained why our type system uses
second-level types to represent the type of the parameter list
and the return type.
Now, we explain their subtyping rules.

The subtyping rule for pair types is the standard covariant rule:
\[
\begin{array}{c}
\mylabel{S-PAIR1}\\
\dfrac{\senv \vdash F_{1} \subtype F_{2} \;\;\;
       \senv \vdash P_{1} \subtype P_{2}}
      {\senv \vdash F_{1} \times P_{1} \subtype F_{2} \times P_{2}}
\end{array}
\]

The subtyping rules for variadic types are not so obvious.
We need three different subtyping rules for variadic types
to handle all the cases where they can appear.

The rule \textsc{S-VARARG1} handles subtyping between two
variadic types:
\[
\begin{array}{c}
\mylabel{S-VARARG1}\\
\dfrac{\senv \vdash F_{1} \cup \Nil \subtype F_{2} \cup \Nil}
      {\senv \vdash F_{1}{*} \subtype F_{2}{*}}
\end{array}
\]

This rule shows that $F_{1}{*}$ is a subtype of $F_{2}{*}$
if $F_{1} \cup \Nil$ is a subtype of $F_{2} \cup \Nil$.
It explicitly includes $\Nil$ in both sides because otherwise
$\Nil{*}$ would not be a subtype of several other variadic types.
For instance, $\Nil{*}$ would not be a subtype of $\Number{*}$,
as $\Nil \not\subtype \Number$.

The other rules handle subtyping between a varidic type and a pair type:
\[
\begin{array}{c}
\begin{array}{c}
\mylabel{S-VARARG2}\\
\dfrac{\senv \vdash F_{1} \cup \Nil \subtype F_{2} \;\;\;
       \senv \vdash F_{1}{*} \subtype P_{2}}
      {\senv \vdash F_{1}{*} \subtype F_{2} \times P_{2}}
\end{array}
\;
\begin{array}{c}
\mylabel{S-VARARG3}\\
\dfrac{\senv \vdash F_{1} \subtype F_{2} \cup \Nil \;\;\;
       \senv \vdash P_{1} \subtype F_{2}{*}}
      {\senv \vdash F_{1} \times P_{1} \subtype F_{2}{*}}
\end{array}
\end{array}
\]

These rules state the conditions when tuple types of different
length are compatible.
In the next section we will show that we use the subtyping rules for variadic types,
along with the types $\Value$ and $\Nil$, to make our type system reflect
the semantics of Lua on discarding extra parameters and
replacing missing parameters.

The subtyping rules for unions of tuple types are similar to the
subtyping rules for unions of first-level types:
\[
\begin{array}{c}
\begin{array}{c}
\mylabel{S-UNION4}\\
\dfrac{\senv \vdash S_{1} \subtype S \;\;\;
       \senv \vdash S_{2} \subtype S}
      {\senv \vdash S_{1} \sqcup S_{2} \subtype S}
\end{array}
\;
\begin{array}{c}
\mylabel{S-UNION5}\\
\dfrac{\senv \vdash S \subtype S_{1}}
      {\senv \vdash S \subtype S_{1} \sqcup S_{2}}
\end{array}
\;
\begin{array}{c}
\mylabel{S-UNION6}\\
\dfrac{\senv \vdash S \subtype S_{2}}
      {\senv \vdash S \subtype S_{1} \sqcup S_{2}}
\end{array}
\end{array}
\]

Back to the subtyping rules between first-level types,
the subtyping rule among a \emph{fixed} or \emph{closed}
table type and another \emph{closed} table type resembles the
standard subtyping rule between records:
\[
\begin{array}{c}
\mylabel{S-TABLE1}\\
\dfrac{\forall i \; \exists j \;\;\;
       \senv \vdash F_{j} \subtype F_{i}' \;\;\;
       \senv \vdash F_{i}' \subtype F_{j} \;\;\;
       \senv \vdash V_{j} \subtype_{c} V_{i}'}
      {\senv \vdash \{\overline{F{:}V}\}_{fixed|closed} \subtype
                    \{\overline{F'{:}V'}\}_{closed}}
\end{array}
\]

The rule \textsc{S-TABLE1} allows width subtyping and introduces the
auxiliary relation $\subtype_{c}$ to handle depth subtyping on the
type of the values stored in the table fields.
We need an auxiliary relation because the subtyping of the
type of the values stored in the table fields changes according to
the tags of the table types.
We define the relation $\subtype_{c}$ as follows:
\[
\begin{array}{c}
\begin{array}{c}
\mylabel{S-FIELD1}\\
\dfrac{\senv \vdash F_{1} \subtype F_{2} \;\;\;
       \senv \vdash F_{2} \subtype F_{1}}
      {\senv \vdash F_{1} \subtype_{c} F_{2}}
\end{array}
\;
\begin{array}{c}
\mylabel{S-FIELD2}\\
\dfrac{\senv \vdash F_{1} \subtype F_{2}}
      {\senv \vdash \Const \; F_{1} \subtype_{c} \Const \; F_{2}}
\end{array}
\\ \\
\begin{array}{c}
\mylabel{S-FIELD3}\\
\dfrac{\senv \vdash F_{1} \subtype F_{2}}
      {\senv \vdash F_{1} \subtype_{c} \Const \; F_{2}}
\end{array}
\end{array}
\]

These rules allow depth subtyping on $\Const$ fields.
The rule \textsc{S-FIELD1} defines that mutable fields are invariant,
while the rule \textsc{S-FIELD2} defines that immutable fields are covariant.
The rule \textsc{S-FIELD3} defines that it is safe to promote fields
from mutable to immutable.
We do not include a rule that allows promoting fields from immutable
to mutable because this would be unsafe due to variance.

There is a limitation on \emph{closed} table types that led us to
introduce \emph{open} and \emph{unique} table types.
If the table constructor had a \emph{closed} table type, then
programmers would not be able to use it to initialize a variable with
a table type that describes a more general type.
For instance,
\begin{verbatim}
    local t:{"x":integer, "y":integer?} = { x = 1, y = 2 }
\end{verbatim}
would not type check, as the type of the table constructor would not
be a subtype of the type in the annotation.
More precisely,
\[
\{``x":1, ``y":2\}_{closed} \not\subtype
\{``x":\Integer, ``y":\Integer \cup \Nil\}_{closed}
\]

Simply promoting the type of each table value to its supertype would
not overcome this limitation, as it still would give to the table constructor
a \emph{closed} table type without covariant mutable fields.
Thus, programmers would not be able to use the table constructor to
initialize a variable with a table type that includes an optional field.
Using the previous example,
\begin{align*}
& \{``x":\Integer, ``y":\Integer\}_{closed} \not\subtype \\
& \{``x":\Integer, ``y":\Integer \cup \Nil\}_{closed}
\end{align*}

We introduced \emph{unique} table types to avoid this limitation,
as they represent the type of tables with no keys that do not
inhabit one of the table's key types, and with no alias.
In particular, this is the case of the table constructor.
The following subtyping rule defines the subtyping relation among
\emph{unique} table types and \emph{closed} table types:
\[
\begin{array}{c}
\mylabel{S-TABLE2}\\
\dfrac{\begin{array}{c}
       \forall i \; \forall j \;\;\;
       \senv \vdash F_{i} \subtype F_{j}' \to \senv \vdash V_{i} \subtype_{u} V_{j}'\\
       \forall j \; \nexists i \;\;\;
       \senv \vdash F_{i} \subtype F_{j}' \to \senv \vdash \Nil \subtype_{o} V_{j}'
       \end{array}}
      {\senv \vdash \{\overline{F{:}V}\}_{unique} \subtype
                    \{\overline{F'{:}V'}\}_{closed}}
\end{array}
\]

The rule \textsc{S-TABLE2} allows width subtyping and covariant keys.
It allows covariant keys because we also want to use \emph{unique}
table types as a way to join table fields that inhabit \emph{closed} table types.
For instance, we want to use the table constructor to initialize
a variable with a table type that describes a hash.
More precisely, this rule states that it is safe to recast the table type
$\{``x":\Integer, ``y":\Integer, ``z":\Integer\}_{unique}$ to
$\{\String:\Integer \cup \Nil\}_{closed}$, as long as the
new type becomes inaccessible with the original type.

The rule \textsc{S-TABLE2} introduced the auxiliary relations
$\subtype_{u}$ and $\subtype_{o}$.
The first allows depth subtyping on all fields,
while the second allows the omission of optional fields.
We define them as follows:
\[
\begin{array}{c}
\begin{array}{c}
\mylabel{S-FIELD4}\\
\dfrac{\senv \vdash F_{1} \subtype F_{2}}
      {\senv \vdash F_{1} \subtype_{u} F_{2}}
\end{array}
\;
\begin{array}{c}
\mylabel{S-FIELD5}\\
\dfrac{\senv \vdash F_{1} \subtype F_{2}}
      {\senv \vdash \Const \; F_{1} \subtype_{u} \Const \; F_{2}}
\end{array}
\\ \\
\begin{array}{c}
\mylabel{S-FIELD6}\\
\dfrac{\senv \vdash F_{1} \subtype F_{2}}
      {\senv \vdash \Const \; F_{1} \subtype_{u} F_{2}}
\end{array}
\;
\begin{array}{c}
\mylabel{S-FIELD7}\\
\dfrac{\senv \vdash F_{1} \subtype F_{2}}
      {\senv \vdash F_{1} \subtype_{u} \Const \; F_{2}}
\end{array}
\\ \\
\begin{array}{c}
\mylabel{S-FIELD8}\\
\dfrac{\senv \vdash \Nil \subtype F}
      {\senv \vdash \Nil \subtype_{o} F}
\end{array}
\;
\begin{array}{c}
\mylabel{S-FIELD9}\\
\dfrac{\senv \vdash \Nil \subtype F}
      {\senv \vdash \Nil \subtype_{o} \Const \; F}
\end{array}
\end{array}
\]

Using \emph{unique} table types to represent the type of the table
constructor allows our type system to type check the previous example.
More precisely,
\[
\{``x":1, ``y":2\}_{unique} \subtype 
\{``x":\Integer, ``y":\Integer \cup \Nil\}_{closed}
\]

Even though we allow width subtyping between \emph{unique} and \emph{closed}
table types, we do not allow it among \emph{unique} and other table types
because it would violate our definition of these other table types:
\[
\begin{array}{c}
\mylabel{S-TABLE3}\\
\dfrac{\begin{array}{c}
       \forall i \; \exists j \;\;\;
       \senv \vdash F_{i} \subtype F_{j}' \land \senv \vdash V_{i} \subtype_{u} V_{j}' \\
       \forall j \; \nexists i \;\;\;
       \senv \vdash F_{i} \subtype F_{j}' \to \senv \vdash \Nil \subtype_{o} V_{j}'
       \end{array}}
      {\senv \vdash \{\overline{F{:}V}\}_{unique} \subtype
                    \{\overline{F'{:}V'}\}_{unique|open|fixed}}
\end{array}
\]

The rule that handles subtyping between \emph{open} and \emph{closed} table
types allows width subtyping:
\[
\begin{array}{c}
\mylabel{S-TABLE4}\\
\dfrac{\begin{array}{c}
       \forall i \; \forall j \;\;\;
       \senv \vdash F_{i} \subtype F_{j}' \to \senv \vdash V_{i} \subtype_{c} V_{j}' \\
       \forall j \; \nexists i \;\;\;
       \senv \vdash F_{i} \subtype F_{j}' \to \senv \vdash \Nil \subtype_{o} V_{j}'
       \end{array}}
      {\senv \vdash \{\overline{F{:}V}\}_{open} \subtype
                    \{\overline{F'{:}V'}\}_{closed}}
\end{array}
\]

However, the rule that handles subtyping among \emph{open} and
\emph{open} or \emph{fixed} table types does not allow width subtyping:
\[
\begin{array}{c}
\mylabel{S-TABLE5}\\
\dfrac{\begin{array}{c}
       \forall i \; \exists j \;\;\;
       \senv \vdash F_{i} \subtype F_{j}' \land \senv \vdash V_{i} \subtype_{c} V_{j}' \\
       \forall j \; \nexists i \;\;\;
       \senv \vdash F_{i} \subtype F_{j}' \to \senv \vdash \Nil \subtype_{o} V_{j}'
       \end{array}}
      {\senv \vdash \{\overline{F{:}V}\}_{open} \subtype
                    \{\overline{F'{:}V'}\}_{open|fixed}}
\end{array}
\]

The rules \textsc{S-TABLE4} and \textsc{S-TABLE5} allow joining fields
plus omitting optional fields.
Both rules use $\subtype_{c}$ to allow depth subtyping on $\Const$ fields only.

We introduced \emph{fixed} table types because we needed a safe way
to represent the type of classes that can allow single inheritance
through the refinement of table types.
The rule that handles subtyping between \emph{fixed} table types
does not allow width subtyping, joining fields, and omitting fields,
but it allows depth subtyping on $\Const$ fields:
\[
\begin{array}{c}
\mylabel{S-TABLE6}\\
\dfrac{\begin{array}{c}
       \forall i \; \exists j \;\;\;
       \senv \vdash F_{i} \subtype F_{j}' \;\;\;
       \senv \vdash F_{j}' \subtype F_{i} \;\;\;
       \senv \vdash V_{i} \subtype_{c} V_{j}' \\
       \forall j \; \exists i \;\;\;
       \senv \vdash F_{i} \subtype F_{j}' \;\;\;
       \senv \vdash F_{j}' \subtype F_{i} \;\;\;
       \senv \vdash V_{i} \subtype_{c} V_{j}' \\
       \end{array}}
      {\senv \vdash \{\overline{F{:}V}\}_{fixed} \subtype
                    \{\overline{F'{:}V'}\}_{fixed}}
\end{array}
\]

In the next section we will show in more detail how our type system
uses these tags to handle the refinement of table types.

We use the \emph{Amber rule} \cite{cardelli1986amber} to define
subtyping between recursive types:
\[
\begin{array}{c}
\begin{array}{c}
\mylabel{S-AMBER}\\
\dfrac{\senv[x_{1} \subtype x_{2}] \vdash F_{1} \subtype F_{2}}
      {\senv \vdash \mu x_{1}.F_{1} \subtype \mu x_{2}.F_{2}}
\end{array}
\;
\begin{array}{c}
\mylabel{S-ASSUMPTION}\\
\dfrac{x_{1} \subtype x_{2} \in \senv}
      {\senv \vdash x_{1} \subtype x_{2}}
\end{array}
\end{array}
\]

The rule \textsc{S-AMBER} also uses the rule \textsc{S-ASSUMPTION}
to check whether $\mu x_{1}.F_{1} \subtype \mu x_{2}.F_{2}$.
Both rules use the set of assumptions $\senv$,
where each assumption is a pair of recursion variables.
The rule \textsc{S-AMBER} extends $\senv$ with the assumption
$x_{1} \subtype x_{2}$ to check whether $F_{1} \subtype F_{2}$.
The rule \textsc{S-ASSUMPTION} allows the rule \textsc{S-AMBER}
to check whether an assumption is valid.

A recursive type may appear inside a first-level type, and our
type system includes subtyping rules to handle subtyping between
recursive types and other first-level types:
\[
\begin{array}{c}
\begin{array}{c}
\mylabel{S-UNFOLDR}\\
\dfrac{\senv \vdash F_{1} \subtype [x \mapsto \mu x.F_{2}]F_{2}}
      {\senv \vdash F_{1} \subtype \mu x.F_{2}}
\end{array}
\;
\begin{array}{c}
\mylabel{S-UNFOLDL}\\
\dfrac{\senv \vdash [x \mapsto \mu x.F_{1}]F_{1} \subtype F_{2}}
      {\senv \vdash \mu x.F_{1} \subtype F_{2}}
\end{array}
\end{array}
\]

As an example, the rule \textsc{S-UNFOLDR} allows our type system to
type check the function \texttt{insert} from Section \ref{sec:alias}:
\begin{verbatim}
    local function insert (e:Element?, v:integer):Element
      return { info = v, next = e }
    end
\end{verbatim}
that is, the type checker uses the rule \textsc{S-UNFOLDR} to verify whether
the type of the table constructor is a subtype of \texttt{Element}:
\begin{align*}
\{ & ``info":\Integer, \\
   & ``next":\mu x.\{``info":\Integer,
                     ``next":x \;\cup\; \Nil\}_{closed} \cup \Nil \}_{unique} \subtype \\
& \mu x.\{``info":\Integer, ``next":x \;\cup\; \Nil\}_{closed}
\end{align*}

Filter types are subtypes only of themselves.
More precisely, a filter type $\phi(F_{1},F_{2})$ is a subtype of
the same filter type $\phi(F_{1},F_{2})$, which shares the same
types $F_{1}$ and $F_{2}$.

Projection types are subtypes only of themselves.
More precisely, a projection type $\pi_{i}^{x}$ is a subtype of the
same projection type $\pi_{i}^{x}$, which shares the same union of
tuples $x$ and the same index $i$.

The subtyping rules for expression list types are similar to the
subtyping rules for tuple types.

The dynamic type $\Any$ is neither the bottom nor the top type,
but a separate type that is subtype only of itself and of $\Value$.

Even though the dynamic type $\Any$ does not interact with subtyping,
it does interact with consistent-subtyping.
We present the consistent-subtyping rules as a deduction system for
the consistent-subtyping relation $\senv \vdash T_{1} \lesssim T_{2}$,
where $T_{1}$ and $T_{2}$ are two types of the same kind.
As in the subtyping relation, $\senv$ is also a set of pairs of
recursion variables.
We define the consistent-subtyping rules for the dynamic type $\Any$
as follows:
\[
\begin{array}{c}
\begin{array}{c}
\mylabel{C-ANY1}\\
\senv \vdash F \lesssim \Any
\end{array}
\;
\begin{array}{c}
\mylabel{C-ANY2}\\
\senv \vdash \Any \lesssim F
\end{array}
\end{array}
\]

If we had set the type $\Any$ as both bottom and top types of our
subtyping relation, then any type $F_{1}$ would be a subtype of
any other type $F_{2}$.
The consequence of this is that all programs would type check without errors.
This would happen due to the transitivity of subtyping, that is,
we would be able to down-cast any type $F_{1}$ to $\Any$ and then up-cast
$\Any$ to any other type $F_{2}$.
The rules \textsc{C-ANY1} and \textsc{C-ANY2} are the rules that
allow the dynamic type to interact with other first-level types,
and thus allow dynamically typed code to coexist with statically
typed code.
Because of these two rules, consistent-subtyping cannot be transitive.
These two rules are the only rules that differ between
subtyping and consistent-subtyping, if we implement the subtyping rules
as we do in this section.

In the implementation of Typed Lua we also use consistent-subtyping to
normalize and simplify union types, though we let union types free in
the formalization.
For instance, the union type \texttt{boolean|any} results in the
type \texttt{any}, because \texttt{boolean} is consistent-subtype
of \texttt{any}.
Another example is the union type \texttt{number|nil|1} that
results in the union type \texttt{number|nil}, because
\texttt{1} is consistent-subtype of \texttt{number}.

\section{Type checking}
\label{sec:rules}

In this section we use a reduced core of Typed Lua to present the
most interesting rules of our type system.
These rules type check multiple assignment, table refinement,
and overloading on the return type of functions.
Appendix \ref{app:rules} presents the full set of typing rules.

Our core limits control flow to if and while statements;
it has explicit type annotations, explicit scope for variables,
explicit method declarations, and explicit method calls.
Here is a list of features that are not present in our reduced core:
\begin{itemize}
\item labels and goto statements (they are difficult to handle along
with our simplified form of \emph{flow typing}, and they are out of
scope for now);
\item explicit blocks (we are already using explicit scope for variables);
\item other loop structures such as repeat-until, numeric for,
and generic for (we can use while to express them);
\item table fields other than $[e_{1}] = e_{2}$
(we can use this form to express the missing forms); 
\item arithmetic operators other than $+$
(other arithmetic operators have similar typing rules);
\item relational operators other than $==$ and $<$
(inequality has similar typing rules to $==$ and
other relational operators have similar typing rules to $<$);
\item bitwise operators other than $\&$
(other bitwise operators have similar typing rules).
\end{itemize}

Our reduced core does not lose much expressiveness, as it can express
any Lua program except those that use labels and goto statements.

\begin{figure}[!ht]
\textbf{Abstract Syntax}\\
\dstart
$$
\begin{array}{rlr}
s ::= & & \textsc{statements:}\\
& \;\; \mathbf{skip} & \textit{skip}\\
& | \; s_{1} \; ; \; s_{2} & \textit{sequence}\\
& | \; \overline{l} = el & \textit{multiple assignment}\\
& | \; \mathbf{while} \; e \; \mathbf{do} \; s \;
| \; \mathbf{if} \; e \; \mathbf{then} \; s_{1} \; \mathbf{else} \; s_{2} & \textit{control flow}\\
& | \; \mathbf{local} \; \overline{id{:}F} = el \; \mathbf{in} \; s & \textit{variable declaration}\\
& | \; \mathbf{local} \; \overline{id} = el \; \mathbf{in} \; s & \textit{variable declaration}\\
& | \; \mathbf{rec} \; id{:}F = e \; \mathbf{in} \; s & \textit{recursive declaration} \\
& | \; \mathbf{return} \; el & \textit{return} \\
& | \; \lfloor a \rfloor_{0} & \textit{application with no results}\\
& | \; \mathbf{fun} \; id_{1}{:}id_{2} \; (pl){:}S \; s \;;\; \mathbf{return} \; el & \textit{method declaration}\\
e ::= & & \textsc{expressions:}\\
& \;\; \mathbf{nil} & \textit{nil}\\
& | \; k & \textit{other literals}\\
& | \; id & \textit{variable access}\\
& | \; e_{1}[e_{2}] & \textit{table access}\\
& | \; {<}F{>} \; id & \textit{type coercion}\\
& | \; f & \textit{function declaration}\\
& | \; \{ \; \overline{[e_{1}] = e_{2}} \; \} \;
| \; \{ \; \overline{[e_{1}] = e_{2}},me \; \} & \textit{table constructor}\\
& | \; e_{1} + e_{2} \;
| \; e_{1} \; {..} \; e_{2} \;
| \; e_{1} == e_{2} \;
| \; e_{1} < e_{2} & \textit{binary operations}\\
& | \; e_{1} \;\&\; e_{2} \;
| \; e_{1} \; \mathbf{and} \; e_{2} \;
| \; e_{1} \; \mathbf{or} \; e_{2} & \textit{binary operations}\\
& | \; \mathbf{not} \; e \;
| \; \# \; e & \textit{unary operations} \\
& | \; \lfloor me \rfloor_{1} & \textit{expressions with one result}\\
l ::= & & \textsc{left-hand values:}\\
& \;\; id_{l} & \textit{variable assignment}\\
& | \; e_{1}[e_{2}]_{l} & \textit{table assignment}\\
& | \; id[e] \; {<}V{>} & \textit{type coercion}\\
k ::= & & \textsc{literal constants:}\\
& \;\; \mathbf{false} \; | \;
\mathbf{true} \; | \;
{\it int} \; | \;
{\it float} \; | \;
{\it string} & \\
el ::= & & \textsc{expression lists:}\\
& \;\; \overline{e} \; | \;
\overline{e}, me & \\
me ::= & & \textsc{multiple results:}\\
& \;\; a & \textit{application}\\
& | \; {...} & \textit{vararg expression}\\
a ::= & & \textsc{applications:}\\
& \;\; e(el) & \textit{function application}\\
& | \; e{:}n(el) & \textit{method application}\\
f ::= & & \textsc{function declarations:}\\
& \;\; \mathbf{fun} \; (pl){:}S \; s \;;\; \mathbf{return} \; el & \\
pl ::= & & \textsc{parameter lists:}\\
& \;\; \overline{id{:}F} \; | \;
\overline{id{:}F},{...}{:}F & \\
\end{array}
$$
\dend
\caption{The abstract syntax of Typed Lua}
\label{fig:syntax}
\end{figure}

Figure \ref{fig:syntax} presents the abstract syntax of core Typed Lua.
It splits the syntactic categories as follows:
$s$ are statements, $e$ are expressions, $l$ are left-hand values,
$k$ are literal constants, $el$ are expression lists,
$me$ are expressions with multiple results, $a$ are function and method applications,
$f$ are function declarations, $pl$ are parameter lists,
$id$ are variable names, $F$ are first-level types, and $S$ are second-level types.
The notation $\overline{id{:}F}$ denotes the list $id_{1}{:}F_{1}, ..., id_{n}{:}F_{n}$.

Our reduced core includes two statements for declaring local variables,
one with and another without type annotations.
While we use the former to formalize how our type system handles the declaration
of annotated variables, we use the latter to formalize how our type system
handles the declaration of unannotated variables through local type inference
and also the introduction of projection types.

Our reduced core also includes a truncation operator $\lfloor \rfloor$ for
function applications, method applications, and the vararg expression.
We use $\lfloor a \rfloor_{0}$ to denote function and method applications
that produce no value, because they appear as statements.
We use $\lfloor me \rfloor_{1}$ to denote function applications,
method applications, and vararg expressions that produce only one value,
even if they return multiple values.

We also include two kinds of type coercions in our core language:
the left-hand value $id[e] \; {<}V{>}$ and the expression ${<}F{>} \;id$.
Both allow the refinement of table types.
We also split variable names into two categories to have safe aliasing
of tables in the presence of refinement.
We use $id$ when variable names appear as expressions and $id_{l}$ when
variable names appear as left-hand values.

Even though we can assign only first-level types to variables,
functions and methods can return unions of second-level types,
and our type system should be able to project these unions of
second-level types into unions of first-level types.
We use two different environments to handle this feature.
The first environment is the type environment $\env$ that maps
variables to expression types, as the type of an expression can
be a first-level type, a filter type, or a projection type.
We use $\env_{1}[id \mapsto T]$ to extend the environment $\env_{1}$
with the variable $id$ that maps to type $T$.
The second environment is the projection environment $\penv$ that
maps projection variables to second-level types.
We use $\penv[x \mapsto S]$ to extend the environment $\penv$
with the projection variable $x$ that maps to type $S$.
In Section \ref{sec:fap} we will show how our type system uses the
projection environment $\penv$ for handling projection types,
and also for projecting unions of second-level types into
unions of first-level types.

We present the typing rules as a deduction system for two typing relations,
one for typing statements and another for typing expressions.

We use the relation $\env_{1}, \penv \vdash s, \env_{2}$ for typing statements.
This relation means that given a type environment $\env_{1}$
and a projection environment $\penv$, we can check that a statement $s$
produces a new type environment $\env_{2}$.

We use the relation $\env_{1}, \penv \vdash e : T, \env_{2}$ for typing expressions.
This relation means that given a type environment $\env_{1}$
and a projection environment $\penv$, we can check that an expression $e$ has
type $T$ and produces a new type environment $\env_{2}$.

\subsection*{Assignment and function application}
\label{sec:assignment}

Lua has multiple assignment, and our type system uses three
different kinds of typing rules to type check this feature.
It uses typing rules that type check the different forms
of expression lists that can appear in the right-hand side,
a typing rule that type checks a list of left-hand values,
and a general rule that uses consistent-subtyping to check
whether the type of the right-hand side is consistent with
the type of the left-hand side.

As an example, lets assume that $x$ and $y$ are variables in the
environment with types $\Integer$ and $\String$.
Let us see how our type system type checks the following assignment:
\[
x, y = 1, ``foo"
\]

First, our type system type checks the expression list in the right-hand
side of the assignment.
In our example, the right-hand side of the assignment has type
$1 \times ``foo" \times \Nil{*}$.
Note that our type system includes the type $\Nil{*}$ to replace
missing values.
The rules that type check expression lists introduce the type
$\Nil{*}$ to let the right-hand side produce fewer values than
expected in the left-hand side.
Our example uses the rule \textsc{T-EXPLIST1} to type check
the right-hand side of the assignment.
The rule \textsc{T-EXPLIST1} is the rule that type checks an
expression list where all expressions can only produce a single value:
\[
\begin{array}{c}
\mylabel{T-EXPLIST1}\\
\dfrac{\env_{1}, \penv \vdash e_{i}:F_{i}, \env_{i+1} \;\;\;
       \env_{f} = merge(\env_{1}, ..., \env_{n+1}) \;\;\;
       n = |\;\overline{e}\;|}
      {\env_{1}, \penv \vdash \overline{e}:F_{1} \times ... \times F_{n} \times \Nil{*}, \env_{f}}
\end{array}
\]

Later, in this section we will show that table refinement can
change the type environment while typing an expression or a left-hand value.
Thus, the rules that type check lists of expressions and lists of
left-hand values use a partial auxiliary function \emph{merge} to collect
all environment changes in a new environment $\env_{f}$, if there are no conflicts.
We will also show that we can only change the
environment to add new table fields in a table type, and we cannot change the
type of a variable or a table field which is already present in a table type.

After type checking the right-hand side, our type system type checks the list
of left-hand values.
In our example, the left-hand side of the assignment has type 
$\Integer \times \String \times \Value{*}$.
Note that our type system uses the type $\Value{*}$ to discard extra values.
The rule that type checks lists of left-hand values introduces the type
$\Value{*}$ to let the right-hand side produce more values than
expected in the left-hand side.
Our example uses the rule \textsc{T-LHSLIST} to type
check a list of left-hand values:
\[
\begin{array}{c}
\mylabel{T-LHSLIST}\\
\dfrac{\env_{1}, \penv \vdash l_{i}:F_{i}, \env_{i+1} \;\;\;
       \env_{f} = merge(\env_{1}, ..., \env_{n+1}) \;\;\;
       n = |\;\overline{l}\;|}
      {\env_{1}, \penv \vdash \overline{l}:F_{1} \times ... \times F_{n} \times \Value{*}, \env_{f}}
\end{array}
\]

After type checking the right-hand side and the left-hand side of an assignment,
our type system checks whether their types are consistent.
The rule \textsc{T-ASSIGNMENT} is the general rule that expresses this idea:
\[
\begin{array}{c}
\mylabel{T-ASSIGNMENT}\\
\dfrac{\env_{1}, \penv \vdash el:S_{1}, \env_{2} \;\;\;
       \env_{2}, \penv \vdash \overline{l}:S_{2}, \env_{3} \;\;\;
       S_{1} \lesssim S_{2}}
      {\env_{1}, \penv \vdash \overline{l} = el,\env_{3}}
\end{array}
\]

Back to our example, it type checks through rule \textsc{T-ASSIGNMENT} because
\[
1 \times ``foo" \times \Nil{*} \lesssim \Integer \times \String \times \Value{*}
\]

As another example, lets assume that $x$, $y$, and $z$ are variables in
the environment with types $\Integer$, $\String$, and $\String \cup \Nil$.
The assignment
\[
x, y, z = 1, ``foo"
\]
type checks because
\[
1 \times ``foo" \times \Nil{*} \lesssim \Integer \times \String \times (\String \cup \Nil) \times \Value{*}
\]

Note how $\Nil{*}$ replaces any missing values.
This example type checks because $\Nil{*}$ produces as many $\Nil$
values as we need, and $\Nil$ is consistent with $\String \cup \Nil$,
which is the type of $z$.

Conversely, the assignment
\[
x = 1, ``foo"
\]
type checks because
\[
1 \times ``foo" \times \Nil{*} \lesssim \Integer \times \Value{*}
\]

Note how $\Value{*}$ discards extra values.
This example type checks because $\Value{*}$ discards as many extra
values as we need, and $``foo"$ is consistent with $\Value$.

Rules for function applications are similar to the rule for multiple assignment.
The rule \textsc{T-APPLY1} handles the case where function applications
are expressions that produce multiple values:
\[
\begin{array}{c}
\mylabel{T-APPLY1}\\
\dfrac{\env_{1}, \penv \vdash e:S_{1} \rightarrow S_{2}, \env_{2} \;\;\;
       \env_{2}, \penv \vdash el:S_{3}, \env_{3} \;\;\;
       S_{3} \lesssim S_{1}}
      {\env_{1}, \penv \vdash e(el):S_{2}, \env_{3}}
\end{array}
\]

We also use the rule \textsc{T-APPLY1} as the base case for the rules
that handle the cases where function applications are either statements
that produce no value or expressions that produce only one value.
The rule \textsc{T-STMAPPLY1} discards the produced values,
while the rule \textsc{T-EXPAPPLY1} uses the auxiliary function
\emph{proj} to ensure that only the first value is produced:
\[
\begin{array}{c}
\begin{array}{c}
\mylabel{T-STMAPPLY1}\\
\dfrac{\env_{1}, \penv \vdash e(el):S, \env_{2}}
      {\env_{1}, \penv \vdash \lfloor e(el) \rfloor_{0},\env_{2}}
\end{array}
\;
\begin{array}{c}
\mylabel{T-EXPAPPLY1}\\
\dfrac{\env_{1}, \penv \vdash e(el):S, \env_{2}}
      {\env_{1}, \penv \vdash \lfloor e(el) \rfloor_{1}:proj(S,1), \env_{2}}
\end{array}
\end{array}
\]

We can define \emph{proj} inductively as follows:
\begin{align*}
proj(S_1 \sqcup S_2, i) & = proj(S_1, i) \cup proj(S_2, i) \\
proj(F{*}, i) & = nil(F) \\
proj(F \times P, 1) & = F \\
proj(F \times P, i) & = proj(P, i-1)\\
proj(E{*}, i) & = nil(E) \\
proj(T \times E, 1) & = T \\
proj(T \times E, i) & = proj(E, i-1) \\
nil(T) & = \left\{
\begin{array}{ll}
T & \text{if $\Nil \lesssim T$} \\
T \cup \Nil & \text{otherwise}
\end{array} \right.
\end{align*}

As an example, let us assume that $f$ is a local function in the environment,
and that $f$ has type $\String \times (\Integer \cup \Nil) \times (\Integer \cup \Nil) \times \Value{*} \rightarrow \Integer{*}$.
The function call
\[
f(``foo")
\]
type checks through the rule \textsc{T-APPLY1}, because
\[
``foo" \times \Nil{*} \lesssim \String \times (\Integer \cup \Nil) \times (\Integer \cup \Nil) \times \Value{*}
\]
and the function call
\[
f(``foo",1,2,3)
\]
also type checks through the rule \textsc{T-APPLY1}, because
\[
``foo" \times 1 \times 2 \times 3 \times \Nil{*} \lesssim \String \times (\Integer \cup \Nil) \times (\Integer \cup \Nil) \times \Value{*}
\]

Our type system also catches arity mismatch.
To do that, we end the input type of a function with type $\Nil{*}$
instead of $\Value{*}$.
For instance, let us assume that $f$ has type
$\String \times (\Integer \cup \Nil) \times (\Integer \cup \Nil) \times \Nil{*} \rightarrow \Integer{*}$.
The function call
\[
f(``foo")
\]
type checks through the rule \textsc{T-APPLY1}, because
\[
``foo" \times \Nil{*} \lesssim \String \times (\Integer \cup \Nil) \times (\Integer \cup \Nil) \times \Nil{*}
\]
but the function call
\[
f(``foo",1,2,3)
\]
does not type check through the rule \textsc{T-APPLY1}, because
\[
``foo" \times 1 \times 2 \times 3 \times \Nil{*} \not\lesssim \String \times (\Integer \cup \Nil) \times (\Integer \cup \Nil) \times \Nil{*}
\]

We just mentioned that when our type system type checks an expression list,
it always includes $\Nil{*}$ in the end of the type of this expression list
if its type does not end in a variadic type.
This behavior preserves the semantics of Lua on replacing missing values,
and it is necessary when we omit optional parameters in a function call,
like the previous example showed.

Using $\Nil{*}$ in the end of the type of expression lists also allows
our type system to catch arity mismatch in function calls without optional parameters.
For instance, let us assume that $f$ has type
$\Integer \times \Integer \times \Nil{*} \rightarrow \Integer \times \Nil{*}$.
The function call
\[
f(1)
\]
does not type check through the rule \textsc{T-APPLY1}, because
\[
1 \times \Nil{*} \not\lesssim \Integer \times \Integer \times \Nil{*}
\]
and the function call
\[
f(1,2,3)
\]
also does not type check through the rule \textsc{T-APPLY1}, because
\[
1 \times 2 \times 3 \times \Nil{*} \not\lesssim \Integer \times \Integer \times \Nil{*}
\]

\subsection*{Tables and refinement}
\label{sec:refinement}

Our abstract syntax reduces the syntactic forms of the table constructor
into two forms: $\{\;\overline{[e_{1}] = e_{2}}\;\}$ and
$\{\;\overline{[e_{1}] = e_{2}},me\;\}$.
The first uses a list of table fields $([e_{1}] = e_{2})_{1}, ..., ([e_{1}] = e_{2})_{n}$.
The second uses a list of table fields and an expression that can
produce multiple values.

The simplest expression involving tables is the empty table constructor $\{\}$;
it always has type $\{\}_{unique}$.

As a more interesting example, let us see how our type system type checks
the table constructor $\{ [1] = ``x", [2] = ``y", [3] = ``z" \}$.

First, Typed Lua uses the auxiliary relation
$\env_{1}, \penv \vdash [e_{1}] = e_{2} : (F,V), \env_{2}$ to type check each
table field.
This auxiliary relation means that given a type environment $\env_{1}$
and a projection environment $\penv$, checking a table field $[e_{1}] = e_{2}$
produces a pair $(F,V)$ and a new type environment $\env_{2}$.
A pair $(F,V)$ means that $e_{1}$ has type $F$ and $e_{2}$ has type $V$,
where $F$ is the type of the key and $V$ is the type of the field value.

After type checking each table field, our type system uses each pair $(F,V)$
to build the table type that express the type of a given constructor, and
uses the predicate \emph{wf} to check whether this table type is well-formed.
The predicate \emph{wf} also uses the auxiliary predicate \emph{tag} to
forbid \emph{unique} and \emph{open} fields.
Formally, we can define \emph{wf} inductivelly as follows:
\[
\begin{array}{rcl}
wf(\{\overline{F:V}\}_{unique|open|fixed|closed}) & = & \forall i \; ((\nexists j \; i \not= j \,\wedge\, F_{i} \lesssim F_{j}) \,\wedge\, wf(V_{i}) \,\wedge\\
& & \;\;\;\; \lnot tag(V_{i},unique) \,\wedge\, \lnot tag(V_{i},open))\\
wf({\bf const}\; F) & = & wf(F) \\
wf(F_1 \cup F_2) & = & wf(F_1) \,\wedge\, wf(F_2) \\
wf(\mu x.F) & = & wf(F) \\
wf(S_1 \rightarrow S_2) & = & wf(S_1) \,\wedge\, wf(S_2) \\
wf(S_1 \sqcup S_2) & = & wf(S_1) \,\wedge\, wf(S_2)\\
wf(F{*}) &=& wf(F) \\
wf(F \times P) &=& wf(F) \,\wedge\,wf(P)\\
wf(F) & = & \top \;\;\;\mathrm{for\; all\; other\; cases}
\end{array}
\]

Well-formed table types avoid ambiguity.
For instance, this rule detects that the table type
$\{1:\Number, \Integer:\String, \Any:\Boolean\}$ is ambiguous,
because the type of the value stored by key $1$ can be
$\Number$, $\String$, or $\Boolean$, as $1 \lesssim 1$,
$1 \lesssim \Integer$, and $1 \lesssim \Any$.
Moreover, the type of the value stored by a key of type $\Integer$,
which is not the literal type $1$, can be $\Number$ or $\Boolean$,
as $\Integer \lesssim \Integer$, and $\Integer \lesssim \Any$.

Well-formed table types also do not allow \emph{unique} and
\emph{open} table types to appear in the type of the field values.
We made this restriction because our type system does not keep
track of aliases to table fields.
This means that allowing \emph{unique} and \emph{open} table
types to appear in the type of a value would allow the
creation of unsafe aliases.
Due to this restriction, the rule that type check table fields
use the auxiliary function \emph{fix} (in the definition of \emph{vt})
to change any \emph{unique} and \emph{open} table types
used in the field initializer to \emph{fixed}.
Rule \textsc{T-FIELD} defines this behavior:
\[
\begin{array}{c}
\mylabel{T-FIELD}\\
\dfrac{\env_{1}, \penv \vdash e_{2}:V, \env_{2} \;\;\;
       \env_{2}, \penv \vdash e_{1}:F, \env_{3}}
      {\env_{1}, \penv \vdash [e_{1}] = e_{2}: (F,vt(F,V)), \env_{3}}
\end{array}
\]

The rule \textsc{T-FIELD} uses the auxiliary function \emph{vt}
to adjust the type of the value according to the type of the key.
More precisely, \emph{vt} includes the type $\Nil$ in the type of
the value when the type of the key is not a literal type.
It also uses \emph{fix} to prevent \emph{unique} and \emph{open}
table types to appear in the type of the field values.
We define these functions as follows:
\begin{align*}
vt(L, V) & = fix(V) \\
vt(F_1, F_2) & = nil(fix(F_2))\\
vt(F_1, {\bf const}\;F_2) & = {\bf const}\;nil(fix(F_2))\\
\\
fix(F_{1} \cup F_{2}) & = fix(F_{1}) \cup fix(F_{2})\\
fix(\{\overline{F{:}V}\}_{unique|open}) & = \{\overline{F{:}V}\}_{fixed} \\
fix(F) & = F
\end{align*}

The rule \textsc{T-CONSTRUCTOR1} uses these steps to type check a
table constructor with a list of table fields that do not end with
an expression that potentially returns multiple values:
\[
\begin{array}{c}
\mylabel{T-CONSTRUCTOR1}\\
\dfrac{\begin{array}{c}
       \env_{1}, \penv \vdash ([e_{1}] = e_{2})_{i}:(F_{i},V_{i}), \env_{i+1} \;\;\;
       T = \{F_{1}{:}V_{1}, ..., F_{n}{:}V_{n}\}_{unique} \\
       wf(T) \;\;\;
       n = |\;\overline{[e_{1}] = e_{2}}\;| \;\;\;
       \env_{f} = merge(\env_{1}, ..., \env_{n+1})
       \end{array}}
      {\env_{1}, \penv \vdash \{\;\overline{[e_{1}] = e_{2}}\;\}:T, \env_{f}}
\end{array}
\]

Back to our example, the constructor
$\{ [1] = ``x", [2] = ``y", [3] = ``z" \}$ has type
$\{1:``x", 2:``y", 3:``z"\}_{unique}$ through rule \textsc{T-CONSTRUCTOR1}.
The subtyping rule for \emph{unique} table types allows us assigning
this table to a variable with a more general type such as
$\{1:\String, 2:\String, 3:\String\}_{closed}$ or even
$\{\Integer:\String \cup \Nil\}_{closed}$.

As another example, the constructor $\{[``x"] = 1, [``y"] = \{[``z"] = 2\}\}$
has type $\{``x":1, ``y":\{``z":2\}_{fixed}\}_{unique}$ through rule
\textsc{T-CONSTRUCTOR1}.
The inner table is \emph{fixed} to prevent the creation of unsafe aliases.

After presenting some typing rules of the table constructor,
we start the discussion of the rules that define the most
unusual feature of our type system: the refinement of table types.
The first kind of refinement allows programmers to add new
fields to \emph{unique} or \emph{open} table types through
field assignment.
For instance, in Section \ref{sec:tables} we presented the
following example:
\begin{verbatim}
    local person = {}
    person.firstname = "Lou"
    person.lastname = "Reed"
\end{verbatim}

We can translate this example to our reduced core as follows:
\begin{center}
\begin{tabular}{ll}
\multicolumn{2}{l}{$\mathbf{local} \; person = \{\} \; \mathbf{in}$}\\
& \multicolumn{1}{l}{$person[``firstname"] \; {<}\String{>} = ``Lou";$}\\
& \multicolumn{1}{l}{$person[``lastname"] \; {<}\String{>} = ``Reed"$}
\end{tabular}
\end{center}

In this example, we assign the type $\{\}_{unique}$ to the variable
$person$, then we refine its type to $\{``firstname":\String\}_{unique}$,
and then we refine its type to $\{``firstname":\String, ``lastname":\String\}_{unique}$.
Rule \textsc{T-REFINE1} type checks this use of refinement:
\[
\begin{array}{c}
\mylabel{T-REFINE1}\\
\dfrac{\begin{array}{c}
       \env_{1}(id) = \{\overline{F{:}V}\}_{unique}\\
       \env_{1}, \penv \vdash e:F_{new}, \env_{2} \;\;\;
       \nexists i \in 1..n \; F_{new} \lesssim F_{i} \;\;\;
       V_{new} = vt(F_{new},V) \;\;\; n = |\overline{F{:}V}|
       \end{array}}
      {\env_{1}, \penv \vdash id[e] {<}V{>}:V_{new}, \env_{2}[id \mapsto \{\overline{F{:}V}, F_{new}{:}V_{new}\}_{unique}]}
\end{array}
\]

The rule for refining \emph{open} table types is similar,
changing only the tag in the type of $id$:
\[
\begin{array}{c}
\mylabel{T-REFINE2}\\
\dfrac{\begin{array}{c}
       \env_{1}(id) = \{\overline{F{:}V}\}_{open}\\
       \env_{1}, \penv \vdash e:F_{new}, \env_{2} \;\;\;
       \nexists i \in 1..n \; F_{new} \lesssim F_{i} \;\;\;
       V_{new} = vt(F_{new},V) \;\;\; n = |\overline{F{:}V}|
       \end{array}}
      {\env_{1}, \penv \vdash id[e] {<}V{>}:V_{new}, \env_{2}[id \mapsto \{\overline{F{:}V}, F_{new}{:}V_{new}\}_{open}]}
\end{array}
\]

Our type system also includes analogous rules for adding methods
to \emph{unique} and \emph{open} tables as a side-effect of
type checking a method declaration, but we will not discuss them
in this section for brevity.

We use the refinement of table types to handle the declaration of
new global variables.
In Lua, the assignment \texttt{v = v + 1} translates to the statement
\texttt{\string_ENV["v"] = \string_ENV["v"] + 1} when \texttt{v}
is not a local variable, where \texttt{\string_ENV} is a table
that stores the global environment.
For this reason, Typed Lua treats accesses to global variables as field accesses
to an \emph{open} table in the top-level scope.
In the following examples we assume that $\string_ENV$ is in the
environment and has type $\{\}_{open}$.

As an example,
\[
\string_ENV[``x"] \; {<}\String{>} = ``foo" \;;\; \string_ENV[``y"] \; {<}\Integer{>} = 1
\]
uses field assignment to add fields $``x"$ and $``y"$ to $\string_ENV$.
Therefore, after these field assignments $\string_ENV$ has type
$\{``x":\String, ``y":\Integer\}_{open}$.

We do not allow the refinement of table types to add a field if it is
already present in the table's type.
For instance,
\[
\string_ENV[``x"] \; {<}\String{>} = ``foo" \;;\; \string_ENV[``x"] \; {<}\Integer{>} = 1
\]
does not type check, as we are trying to add $``x"$ twice.

We also do not allow the refinement of table types to introduce
fields with table types that are neither \emph{fixed} nor \emph{closed}.
For instance,
\begin{center}
\begin{tabular}{l}
$\string_ENV[``x"] \; {<}\{\}_{unique}{>} = \{\}$
\end{tabular}
\end{center}
refines the type of $\string_ENV$ from $\{\}_{open}$ to $\{``x":\{\}_{fixed}\}_{open}$.
Currently, our type system can only track \emph{unique} and
\emph{open} table types that are bound to local variables.

We can also use multiple assignment to refine table types:
\[
\string_ENV[``x"] \; {<}\String{>}, \string_ENV[``y"] \; {<}\Integer{>} = ``foo", 1
\]

This example type checks because all the environment changes are consistent, and
$``foo" \times 1 \times \Nil{*} \lesssim \String \times \Integer \times \Value{*}$.
By consistent we mean that we are only adding new fields.
More precisely, the first coercion expression refines the type of $\string_ENV$
to $\{``x":\String\}_{open}$, while
the second coercion expression refines the type of $\string_ENV$
to $\{``y":\Integer\}_{open}$.
Merging the two yields $\{``x":\String, ``y":\Integer\}_{open}$.
Nevertheless, the next example does not type check because it tries to add
the same field to $\string_ENV$, but with different types:
\[
\string_ENV[``x"] \; {<}\String{>}, \string_ENV[``x"] \; {<}\Integer{>} = ``foo", 1
\]

Aliasing an \emph{unique} or an \emph{open} table type can produce
either a \emph{closed} or a \emph{fixed} table type, depending on
the context that we are using a variable.
As we mentioned in Sections \ref{sec:oop} and \ref{sec:subtyping},
we need \emph{fixed} table types to type classes in object-oriented programming.
In the implementation we fix the aliasing of \emph{unique} and \emph{open}
table types that appear in a top-level return statement, and in other cases we
close the aliasing of \emph{unique} and \emph{open} table types.
However, in the formalization we chose to define this behavior in
a not deterministic way, as it makes easier the presentation of this behavior.

As an example,
\begin{center}
\begin{tabular}{lll}
\multicolumn{3}{l}{$\mathbf{local} \; a:\{\}_{unique} = \{\} \; \mathbf{in}$}\\
& \multicolumn{2}{l}{$\mathbf{local} \; b:\{\}_{open} = a \; \mathbf{in}$}\\
& & \multicolumn{1}{l}{$a[``x"] \; {<}\String{>} = ``foo";$}\\
& & \multicolumn{1}{l}{$b[``x"] \; {<}\Integer{>} = 1$}\\
\end{tabular}
\end{center}
does not type check, as aliasing $a$ produces the type $\{\}_{closed}$
that is not a subtype of $\{\}_{open}$, the type of $b$.
Our type system has this behavior to warn programmers about
potential unsafe behaviors after this kind of alias.
In this example, it is unsafe to add the field $``x"$ to $b$,
as it changes the value that is stored in the field $``x"$ of $a$.

Rules \textsc{T-IDREAD1} and \textsc{T-IDREAD2} define this non-deterministic behavior.
Rule \textsc{T-IDREAD1} uses the auxiliary function \emph{close} to
produce a \emph{closed} alias.
It also uses the auxiliary function \emph{open} to change the type of
the original reference from \emph{unique} to \emph{open},
because aliasing an \emph{unique} table type while keeping the original
reference \emph{unique} can be unsafe.
Rule \textsc{T-IDREAD2} uses the auxiliary function \emph{fix} to
produce a \emph{fixed} alias.
It also uses \emph{fix} to change the type of the original reference
to \emph{fixed}, because a \emph{fixed} table type does not allow
width subtyping.
We define these rules as follows:
\[
\begin{array}{c}
\begin{array}{c}
\mylabel{T-IDREAD1}\\
\dfrac{\env_{1}(id) = F}
      {\env_{1}, \penv \vdash id:close(F), \env_{1}[id \mapsto open(F)]}
\end{array}
\\ \\
\begin{array}{c}
\mylabel{T-IDREAD2}\\
\dfrac{\env_{1}(id) = F}
      {\env_{1}, \penv \vdash id:fix(F), \env_{1}[id \mapsto fix(F)]}
\end{array}
\end{array}
\]

We do not need to close \emph{unique} and \emph{open} tables that
appear in the left-hand side of assignments, because \textsc{T-IDREAD1}
and \textsc{T-IDREAD2} are sufficient to forbid the creation of an alias
to another \emph{unique} or \emph{open} table.
For this reason, identifiers that appear in the left-hand side
of assignments have their own rule \textsc{T-IDWRITE1}:
\[
\begin{array}{c}
\mylabel{T-IDWRITE1}\\
\dfrac{\env_{1}(id) = F}
      {\env_{1}, \penv \vdash id_{l}:F, \env_{1}}
\end{array}
\]

Our type system also has different rules for type checking table indexing to avoid
changing table types in these operations, as they cannot create aliases.
These rules also use the auxiliary function \emph{rconst} to strip the
$\Const$ type from the type of the value, if present.
We define these rules as follows:
\[
\begin{array}{c}
\begin{array}{c}
\mylabel{T-INDEXREAD1}\\
\dfrac{\begin{array}{c}
       \env_{1}(id) = \{\overline{F{:}V}\} \;\;\;
       \env_{1}, \penv \vdash e_{2}:F, \env_{2} \;\;\;
       \exists i \in 1{..}n \; F \lesssim F_{i} \;\;\;
       n = |\overline{F{:}V}|
       \end{array}}
      {\env_{1}, \penv \vdash id[e_{2}]:rconst(V_{i}), \env_{2}}
\end{array}
\\ \\
\begin{array}{c}
\mylabel{T-INDEXREAD2}\\
\dfrac{\begin{array}{c}
       \env_{1}, \penv \vdash e_{1}:\{\overline{F{:}V}\}, \env_{2} \;\;\;
       \env_{2}, \penv \vdash e_{2}:F, \env_{3} \;\;\;
       \exists i \in 1{..}n \; F \lesssim F_{i} \;\;\;
       n = |\overline{F{:}V}|
       \end{array}}
      {\env_{1}, \penv \vdash e_{1}[e_{2}]:rconst(V_{i}), \env_{3}}
\end{array}
\end{array}
\]

Rule \textsc{T-INDEXREAD1} defines the case where using an identifier
to index a table does not create an alias, while rule \textsc{T-INDEXREAD2}
defines the case for indexing expressions where the expression
denoting the table is not an identifier.
The rules for indexing left-hand values are similar to these rules,
except that they ensure that the field is not $\Const$.

A second form of refinement happens when we want to use an
\emph{unique} or \emph{open} table type in a context that expects a
\emph{fixed} or \emph{closed} table type with a different shape.
This kind of refinement allows programmers to add optional fields
or merge existing fields.
To do that, Typed Lua includes a type coercion expression ${<}F{>} \; id$.
For instance, we can use this type coercion expression to make the following
example type check:
\begin{center}
\begin{tabular}{lll}
\multicolumn{3}{l}{$\mathbf{local} \; a:\{\}_{unique} = \{ \} \; \mathbf{in}$}\\
& \multicolumn{2}{l}{$a[``x"] \; {<}\String{>} = ``foo";$}\\
& \multicolumn{2}{l}{$a[``y"] \; {<}\String{>} = ``bar";$}\\
& \multicolumn{2}{l}{$\mathbf{local} \; b:\{``x":\String, ``y":\String \cup \Nil \}_{closed} =$}\\
& & \multicolumn{1}{l}{${<}\{``x":\String, ``y":\String \cup \Nil\}_{open}{>} \; a \; \mathbf{in} \; a[``z"] \; {<}\Integer{>} = 1$}
\end{tabular}
\end{center}

We can use $a$ to initialize $b$ because the coercion converts
the type of $a$ from $\{``x":\String, ``y":\String\}_{unique}$ to
$\{``x":\String, ``y":\String \cup \Nil\}_{open}$, and results in
$\{``x":\String, ``y":\String \cup \Nil\}_{closed}$,
which is a subtype of
$\{``x":\String, ``y":\String \cup \Nil\}_{closed}$, the type of $b$.
We can continue to refine the type of $a$ after aliasing it to $b$,
as it still holds an \emph{open} table.
At the end of this example, $a$ has type
$\{``x":\String, ``y":\String \cup \Nil, ``z":\Integer\}_{open}$.

Rules \textsc{T-COERCE1} and \textsc{T-COERCE2} define the behavior of the
type coercion expression:
\[
\begin{array}{c}
\begin{array}{c}
\mylabel{T-COERCE1}\\
\dfrac{\env_{1}(id) \subtype F \;\;\; tag(F,closed)}
      {\env_{1}, \penv \vdash {<}F{>} \; id:F, \env_{1}[id \mapsto reopen(F)]}
\end{array}
\;
\begin{array}{c}
\mylabel{T-COERCE2}\\
\dfrac{\env_{1}(id) \subtype F \;\;\; tag(F,fixed)}
      {\env_{1}, \penv \vdash {<}F{>} \; id:F, \env_{1}[id \mapsto F]}
\end{array}
\end{array}
\]

Note that the coercion rules only allow changing the type
of a variable if the new type is a supertype of the previous type,
and the resulting type is always \emph{fixed} or \emph{closed}
to prevent the creation of unsafe aliases.
If the coercion is to a \emph{closed} table type the type of the
table changes to an \emph{open} table type with the same shape,
but if the coercion is to a \emph{fixed} table type the table
has to assume the same type.

We also need to make sure to close all \emph{unique} and \emph{open}
table types before we type check a nested scope.
To do that, our type system uses some auxiliary functions to change
the type of variables before type checking a nested scope and
also to change the type of assigned and referenced variables after
type checking a nested scope.
The function \emph{closeall} closes all \emph{unique} and \emph{open} table types.
The function \emph{closeset} closes a given set of free assigned variables,
which is given by the function \emph{fav}.
The function \emph{openset} changes from \emph{unique} to \emph{open}
a given set of referenced variables, which is given by the function \emph{frv}.

As an example,
\begin{center}
\begin{tabular}{llll}
\multicolumn{4}{l}{$\mathbf{local} \; a:\{\}_{unique}, b:\{\}_{unique} = \{\}, \{\} \; \mathbf{in}$}\\
& \multicolumn{3}{l}{$\mathbf{local} \; f:\Integer \times \Nil{*} \rightarrow \Integer \times \Nil{*} =$}\\
& & \multicolumn{2}{l}{$\mathbf{fun} \; (x:\Integer):\Integer \times \Nil{*}$}\\
& & & \multicolumn{1}{l}{$b = a \;;\; \mathbf{return} \; x + 1$}\\
& \multicolumn{3}{l}{$\mathbf{in} \; a[``x"] \; {<}\Integer{>} = 1 \;;\; b[``x"] \; {<}\String{>} = ``foo" \;;\; \lfloor f(a[``x"]) \rfloor_{0}$}
\end{tabular}
\end{center}
does not type check because we cannot add the field
$``x"$ to $b$, as its type is closed.
The assignment $b = a$ type checks because, at that point,
$a$ and $b$ have the same type: $\{\}_{closed}$.
Their type was closed by \emph{closeall} before type checking
the function body.
Their type would be restored to $\{\}_{unique}$ after type checking
the function body, but that assignment also triggers other two type changes.
First, the function \emph{fav} includes $b$ in the set of variables
that should be closed by \emph{closeset}.
Then, the function \emph{frv} includes $a$ in the set of variables
that should change from \emph{unique} to \emph{open} by \emph{openset}.
After declaring $f$, $a$ has type $\{\}_{open}$ and $b$ has type $\{\}_{closed}$,
so we can refine the type of $a$, but we cannot refine the type of $b$.

Rule \textsc{T-FUNCTION1} types non-variadic function declarations,
and it illustrates this case:
\[
\begin{array}{c}
\mylabel{T-FUNCTION1}\\
\dfrac{\begin{array}{c}
       closeall(\env_{1})[\overline{id \mapsto F}], \penv[\ret \mapsto S] \vdash s, \env_{2} \\
       \env_{3} = openset(\env_{1}, frv(\mathbf{fun} \; (\overline{id{:}F}){:}S \; s)) \\
       \env_{4} = closeset(\env_{3}, fav(\mathbf{fun} \; (\overline{id{:}F}){:}S \; s))
       \end{array}}
      {\env_{1}, \penv \vdash \mathbf{fun} \; (\overline{id{:}F}){:}S \; s:F_{1} \times ... \times F_{n} \times \Nil{*} \rightarrow S, \env_{4}}
\end{array}
\]

This rule also extends the environment $\penv$, bounding the special
variable $\ret$ to the return type $S$.
Rule \textsc{T-RETURN} uses the type that is bound to $\ret$ in
$\penv$ to type check return statements:
\[
\begin{array}{c}
\mylabel{T-RETURN}\\
\dfrac{\env_{1} \vdash el:S_{1}, \env_{2} \;\;\;
       \penv(\ret) = S_{2} \;\;\;
       S_{1} \lesssim S_{2}}
      {\env_{1} \vdash \mathbf{return} \; el, \env_{2}}
\end{array}
\]

The rules for declaring variadic functions and recursive functions
are similar to \textsc{T-FUNCTION1}, and we did not discuss them in
this section for brevity.

\subsection*{Projections}
\label{sec:fap}

Lua programmers often overload the return type of functions to denote errors,
returning \texttt{nil} and an error message in case of error instead of
the usual return values, and our type system uses projection types to handle this idiom.

As an example, let us assume that \emph{idiv} and \emph{print} are functions
in the environment.
As we mentioned in Section \ref{sec:unions}, \emph{idiv} performs
integer division and has type
\[
\Integer \times \Integer \times \Nil{*} \rightarrow (\Integer \times \Integer \times \Nil{*}) \sqcup (\Nil \times \String \times \Nil{*})
\]
In case of success, it returns two integers: the result and the remainder.
In case of failure, it returns $\Nil$ plus an error message that describes
the error.
The function \emph{print} is a variadic function of type
$\Value{*} \rightarrow \Nil{*}$.
Let us also assume that $a$ and $b$ are local variables in the environment,
and that both have type $\Integer$.
Let us see how our type system type checks the following program:
\begin{center}
\begin{tabular}{ll}
\multicolumn{2}{l}{$\mathbf{local} \; q, r = idiv(a, b) \; \mathbf{in}$}\\
& \multicolumn{1}{l}{$\mathbf{if} \; q \; \mathbf{then} \; \lfloor print(q + r) \rfloor_{0} \; \mathbf{else} \; \lfloor print(``ERROR: " \; .. \; r) \rfloor_{0}$}
\end{tabular}
\end{center}

First, our type system uses the auxiliary relation
$\env_{1}, \penv \vdash el : E, \env_{2}, (x,S)$
for type checking $idiv(a, b)$.
This relation means that given a type environment $\env_{1}$ and
a projection environment $\penv$, we can check that an expression
list $el$ has type $E$ and produces a new type environment $\env_{2}$
and produces a pair $(x,S)$.
This pair means that the last expression of an expression list $el$
produces an union of second-level types $S$ that should be bound
to a fresh variable $x$ in the projection environment $\penv$,
as the resulting type of this expression is a tuple of projection
types $\pi_{i}^{x}$.
In our example, our type system uses rule \textsc{T-EXPLIST3} for
type checking $idiv(a, b)$:
\[
\begin{array}{c}
\mylabel{T-EXPLIST3}\\
\dfrac{\begin{array}{c}
       \env_{1}, \penv \vdash e_{i}:F_{i}, \env_{i+1} \;\;\;
       \env_{1}, \penv \vdash me:S, \env_{n+2}\\
       S = F_{n+1} \times ... \times F_{n+m} \times \Nil{*} \sqcup F_{n+1}' \times ... \times F_{n+m}' \times \Nil{*} \\
       \env_{f} = merge(\env_{1}, ..., \env_{n+2}) \;\;\;
       n = |\;\overline{e}\;|
       \end{array}}
      {\env_{1}, \penv \vdash \overline{e},me:F_{1} \times ... \times F_{n} \times \pi_{1}^{x} \times ... \times \pi_{m}^{x} \times \Nil{*}, \env_{f}, (x,S)}
\end{array}
\]

Note that $idiv(a, b)$ has type
$\pi_{1}^{x} \times \pi_{2}^{x} \times \Nil{*}$ and produces the pair
\[
(x,(\Integer \times \Integer \times \Nil{*}) \sqcup (\Nil \times \String \times \Nil{*})) 
\]

In the rule that type checks the declaration of unannotated variables,
our type system uses the pair $(x,S)$ to bound a union of
second-level types $S$ to a variable $x$ in the projection
environment $\penv$.
In our example, declaring $q$ and $r$
bounds the projection type $\pi_{1}^{x}$ to $q$ and
bounds the projection type $\pi_{2}^{x}$ to $r$,
where the projection variable $x$ bounds to 
\[
(\Integer \times \Integer \times \Nil{*}) \sqcup (\Nil \times \String \times \Nil{*})
\]
in the projection environment $\penv$.
Rule \textsc{T-LOCAL2} illustrates this intuition:
\[
\begin{array}{c}
\mylabel{T-LOCAL2}\\
\dfrac{\begin{array}{c}
       \env_{1}, \penv \vdash el:E, \env_{2}, (x,S)\\
       \env_{3} = \env_{2}[id_{1} \mapsto infer(E,1), ..., id_{n} \mapsto infer(E,n)] \\
       \env_{3}, \penv[x \mapsto S] \vdash s, \env_{4} \;\;\;
       n = |\;\overline{id}\;|  
       \end{array}}
      {\env_{1}, \penv \vdash \mathbf{local} \; \overline{id} = el \; \mathbf{in} \; s, (\env_{4} - \{\overline{id}\})[\overline{id \mapsto \env_{2}(id)}]}
\end{array}
\]

This rule uses the auxiliary function \emph{infer} to get the
most general types of each variable that should be introduced in
the type environment for type checking $s$.
After type checking the statement $s$, rule \textsc{T-LOCAL2} produces a
new type environment $\env_{4}$ without the variables that it introduced
before type checking $s$.
We can define \emph{infer} as follows:
\begin{align*}
infer(T_{1} \times ... \times T_{n}{*}, i) & = \left\{
\begin{array}{ll}
general(T_{i}) & \text{if $i < n$}\\
general(nil(T_{n})) & \text{if $i >= n$}
\end{array} \right.
\end{align*}

\begin{align*}
general(\False) & = \Boolean\\
general(\True) & = \Boolean\\
general({\it int}) & = \Integer\\
general({\it float}) & = \Number\\
general({\it string}) & = \String\\
general(F_{1} \cup F_{2}) & = general(F_{1}) \cup general(F_{2})\\
general(S_{1} \rightarrow S_{2}) & = general2(S_{1}) \rightarrow general2(S_{2})\\
general(\{F_{1}{:}V_{1}, ..., F_{n}{:}V_{n}\}_{tag}) & = \{F_{1}{:}general(V_{1}), ..., F_{n}{:}general(V_{n})\}_{tag}\\
general(\mu x.F) & = \mu x.general(F)\\
general(T) & = T\\
\\
general2(F{*}) & = general(F){*}\\
general2(F \times P) & = general(F) \times general2(P)\\
general2(S_{1} \sqcup S_{2}) & = general2(S_{1}) \sqcup general2(S_{2})
\end{align*}

After assigning projection types to $q$ and $r$, reading $q$ will
use the projection type $\pi_{1}^{x}$ to project the type of $q$
into the union type $\Integer \cup \Nil$, while reading $r$ will
use the projection type $\pi_{2}^{x}$ to project the type of $r$
into the union type $\Integer \cup \String$.
Our type system defines this behavior through rule \textsc{T-IDREAD4},
which uses the auxiliary function \emph{proj} to project an
union of second-level types into an union of first-level types:
\[
\begin{array}{c}
\mylabel{T-IDREAD4}\\
\dfrac{\env_{1}(id) = \pi_{i}^{x}}
      {\env_{1}, \penv \vdash id:proj(\penv(x), i), \env_{1}}
\end{array}
\]

Now, we may want to discriminate $q$ and $r$ to check whether
the function call returned with success.
Introducing a projection variable $x$ in the projection environment allows our
type system to discriminate projection types $\pi_{i}^{x}$,
as they are a general way to not compromise the dependency between
the types of $q$ and $r$ after discriminating one of them,
so flow typing can narrow the type of both variables by testing
just one of them because the projection types of both variables
bound to the same projection variable.

The rule \textsc{T-IF5} shows the case where our type system
discriminates a projection type based on the tag \texttt{nil}.
It uses the auxiliary functions \emph{fopt} and \emph{fipt}
to filter a projection $x$, affecting all variables that bind to the same projection.
More precisely, the former function filters out the tuples that contain a type $F$ in the $i$-th component,
while the latter function filters out the tuples that do not contain $F$ in the $i$-th component.
We define \textsc{T-IF5} as follows:
\[
\begin{array}{c}
\mylabel{T-IF5}\\
\dfrac{\begin{array}{c}
       \env_{1}(id) = \pi_{i}^{x} \\
       S_{t} = fopt(\penv(x), \Nil, i) \;\;\;
       S_{e} = fipt(\penv(x), \Nil, i) \\
       \env_{1}, \penv[x \mapsto S_{t}] \vdash s_{1}, \env_{2}\\
       \env_{1}, \penv[x \mapsto S_{e}] \vdash s_{2}, \env_{3}\\
       \env_{4} = join(\env_{2}, \env_{3})
      \end{array}}
      {\env_{1}, \penv \vdash \mathbf{if} \; id \; \mathbf{then} \; s_{1} \; \mathbf{else} \; s_{2}, \env_{4}}
\end{array}
\]

Our previous example type checks through rule \textsc{T-IF5},
because it uses the information provided by the projection type $\pi_{1}^{x}$,
which is the type of $q$, to make the rule \textsc{T-IF5} use the function call
\[
fopt((\Integer \times \Integer \times \Nil{*}) \sqcup (\Nil \times \String \times \Nil{*}), \Nil, 1)
\]
to discriminate the projection $x$ to the single tuple
$\Integer \times \Integer \times \Nil{*}$ inside the $\mathbf{if}$ branch,
and the function call
\[
fipt((\Integer \times \Integer \times \Nil{*}) \sqcup (\Nil \times \String \times \Nil{*}), \Nil, 1)
\]
to discriminate the projection $x$ to the single tuple
$\Nil \times \String \times \Nil{*}$ inside the $\mathbf{else}$ branch.
Thus, reading $q$ and $r$ projects $\pi_{1}^{x}$ to $\Integer$ and
$\pi_{2}^{x}$ to $\Integer$ inside the $\mathbf{if}$ branch,
but it projects $\pi_{1}^{x}$ to $\Nil$ and $\pi_{2}^{x}$ to $\String$
inside the $\mathbf{else}$ branch.
Outside the condition, $q$ and $r$ use the original projection, that is,
they project to $\Integer \cup \Nil$ and $\Integer \cup \String$, respectively.

Our type system also includes rules that check whether a branch is unreachable.
Rules \textsc{T-IF6} and \textsc{T-IF7} respectively cover the case where the
$\mathbf{else}$ branch is unreachable and the case where the $\mathbf{then}$
branch is unreachable, because either the projected type of $\pi_{i}^{x}$ is
not a supertype of $\Nil$ or it is $\Nil$.
We define these rules as follows:
\[
\begin{array}{c}
\begin{array}{c}
\mylabel{T-IF6}\\
\dfrac{\begin{array}{c}
       \env_{1}(id) = \pi_{i}^{x} \\
       S_{t} = fopt(\penv(x), \Nil, i) \\
       fit(proj(\penv(x), i), \Nil) = \Void \\
       \env_{1}, \penv[x \mapsto S_{t}] \vdash s_{1}, \env_{2}
      \end{array}}
      {\env_{1}, \penv \vdash \mathbf{if} \; id \; \mathbf{then} \; s_{1} \; \mathbf{else} \; s_{2}, \env_{2}}
\end{array}
\;
\begin{array}{c}
\mylabel{T-IF7}\\
\dfrac{\begin{array}{c}
       \env_{1}(id) = \pi_{i}^{x} \\
       S_{e} = fipt(\penv(x), \Nil, i) \\
       fot(proj(\penv(x), i), \Nil) = \Void \\
       \env_{1}, \penv[x \mapsto S_{e}] \vdash s_{2}, \env_{2}
      \end{array}}
      {\env_{1}, \penv \vdash \mathbf{if} \; id \; \mathbf{then} \; s_{1} \; \mathbf{else} \; s_{2}, \env_{2}}
\end{array}
\end{array}
\]

Typed Lua does not allow assignments to left-hand values that are bound
to a projection type.
This kind of assignment would be unsound, because it could break the
dependency relation that the components of each tuple of the union have.
For instance, the following example does not type check:
\begin{center}
\begin{tabular}{ll}
\multicolumn{2}{l}{$\mathbf{local} \; q, r = idiv(a, b) \; \mathbf{in}$}\\
& \multicolumn{1}{l}{$r = ``foo";$}\\
& \multicolumn{1}{l}{$\mathbf{if} \; q \; \mathbf{then} \; \lfloor print(q + r) \rfloor_{0} \; \mathbf{else} \; \lfloor print(``ERROR: " \; .. \; r) \rfloor_{0}$}
\end{tabular}
\end{center}

In this example, the projected type of $r$ outside of the $\mathbf{if}$
statement is $\Integer \cup \String$, so the assignment looks fine.
However, the projected type of $r$ inside the $\mathbf{if}$ branch is $\Integer$,
not matching the $\String$ value that $r$ has after the assignment.


\chapter{Evaluation}
\label{chap:evaluation}

We performed some case studies on existing Lua libraries
to evaluate the design of our type system.
For each library, we used Typed Lua to either annotate its modules
or to write statically typed interfaces to its modules through
Typed Lua's description files.
In this chapter we present our evaluation results and discuss some
interesting cases.

The Lua Standard Library \cite{luamanual} was our first case study.
We started to think about how we would type its modules at the same time
that we started to design our type system, as it could give us some
hints on our type system.
And it did: optional parameters and overloading on the return type
are two Lua features that our type system should handle to allow us
typing some of the functions that the standard library implements.

The second case study that we chose was the MD5 library \cite{lmd5},
because we wanted a simple case study to introduce Typed Lua's description
files and \texttt{userdata} declarations.
These Typed Lua's mechanisms allow programmers to give statically typed
interfaces to Lua libraries.

LuaSocket \cite{luasocket} and LuaFileSystem \cite{luafilesystem} were
the third and fourth case studies that we used to evaluate Typed Lua.
We chose them because they are the most popular Lua libraries.
We wrote a script that builds the dependency graph of Lua libraries
that are in the LuaRocks repository, and uses this dependency graph
to identify the most popular Lua libraries.

We also randomly selected three case studies from the LuaRocks
repository, they are: HTTP Digest \cite{luahttpdigest},
Typical \cite{luatypical}, and Mod 11 \cite{luamod11}.
The first provides client side HTTP digest authentication for Lua.
The second is an extension to the primitive function \texttt{type}.
The third is a generator and checker of modulo 11 numbers.
We randomly selected three case studies because we wanted to evaluate
Typed Lua for annotating existing libraries that are written in Lua,
as the previous case studies are mostly libraries that are written in C.

The Typed Lua compiler is the last case study that we evaluated.
We chose it as a case study because it is a large application.
Besides, it is a case study that evaluates the evolution of a script
to a program.

We used these case studies to evaluate two aspects of Typed Lua:
\begin{enumerate}
\item how precisely it can describe the type of the interface of a module;
\item whether it provides guarantees that the code matches the interface.
\end{enumerate}

\begin{table}[!ht]
\begin{center}
\begin{tabular}{|l|c|c|c|c|c|c|}
\cline{2-5}
\multicolumn{1}{c}{} & \multicolumn{4}{|c|}{percentage of members} & \multicolumn{1}{c}{} \\
\hline
\textbf{Case study} & \textbf{easy} & \textbf{poly} & \textbf{over} & \textbf{hard} & \textbf{\# members} \\
\hline
Lua Standard Library & 64\% & 5\% & 8\% & 23\% & 129 \\ % 100%
\hline
MD5 & 100\% & 0\% & 0\% & 0\% & 13 \\ % 100%
\hline
LuaSocket & 89\% & 1\% & 2\% & 8\% & 123 \\ % 100%
\hline
LuaFileSystem & 89\% & 0\% & 11\% & 0\% & 19 \\ % 100%
\hline
HTTP Digest & 0\% & 0\% & 100\% & 0\% & 1 \\ % 100%
\hline
Typical & 100\% & 0\% & 0\% & 0\% & 1 \\ % 100%
\hline
Modulo 11 & 78\% & 0\% & 0\% & 22\% & 9 \\ % 100%
\hline
Typed Lua Compiler & 93\% & 0\% & 1\% & 6\% & 154 \\ % 100%
\hline
\end{tabular}
\end{center}
\caption{Evaluation results for each case study}
\label{tab:evalbycase}
\end{table}

Table \ref{tab:evalbycase} summarizes our evaluation results for each
of the case studies that we used Typed Lua for typing their members.
An exported member is any Lua value that a module might export.
We split the members of each case study into four categories:
\emph{easy}, \emph{poly}, \emph{over}, and \emph{hard}.
In the next four paragraphs we explain each category in more detail.
The last column of the table shows the total number of members
of each case study.

The \emph{easy} category shows the percentage of members that
we could give a precise static type.
For instance, the function \texttt{string.len} from the Lua
standard library is in this category because we could use
Typed Lua to describe its type: \texttt{(string) -> (integer)}.
This function returns the length of a given string.
Note that the results that we obtained for this category give
a lower bound on how much static type safety
we could add to each one of our case studies.

The \emph{poly} category shows the percentage of members that
we made minimal use of the dynamic type, as a replacement for the
lack of type parameters.
For instance, the function \texttt{table.sort} from the Lua
standard library is in this category because it is a generic function.
It sorts a given list of elements, which is a generic list.
However, we had to assign to this function the type
\texttt{(\{any\}, nil|(any, any) -> (boolean)) -> ()} because
Typed Lua does not support parametric polymorphism.
A better type for it would be
\texttt{(\{<T>\}, nil|(<T>, <T>) -> (boolean)) -> ()}.

The \emph{over} category shows the percentage of members that
require intersection types to describe their precise static types,
as they are overloaded functions.
For instance, the function \texttt{math.abs} from the Lua
standard library is in this category because it has two types:
\texttt{(integer) -> (integer)} and \texttt{(number) -> (number)}.
This function returns the absolute value of a given number,
which can be either integer or float.
Even though we gave this function the more general type \texttt{(number) -> (number)},
it is not precise enough because the return type is always \texttt{number}
independently of the argument type.
In other words, the return type should be \texttt{integer} when
the argument type is also \texttt{integer}, and the return type
should be \texttt{number} when the argument type is also \texttt{number}.
However, we cannot give such a precise type to this function because
Typed Lua does not support overloaded functions.

The \emph{hard} category shows the percentage of members that
do not fit in one of the previous categories, as they are difficult to type.
For instance, the function \texttt{string.format} from the Lua
standard library is in this category because it relies on format
strings, which are difficult to type.
Still, we gave this function the type \texttt{(string, value*) -> (string)}.

In the following sections we discuss each case study in more detail.
For each case study, we split the evaluation results according
to the modules that each one of them include.
We use these split results to discuss the contributions and
limitations of our type system.

\section{Lua Standard Library}

All of the modules in the standard library are implemented in C, so
we used Typed Lua to type just the interface of each module.
The \texttt{debug} module is the only one that we did not include in our
evaluation results, because it provides several functions that violate
basic assumptions about Lua code \cite{luamanual}.
For instance, we can use the function \texttt{debug.setlocal} to change the value
of a local variable that is not visible in the current scope.
Table \ref{tab:evallsl} summarizes the evaluation results for the Lua Standard Library
(version 5.3).

\begin{table}[!ht]
\begin{center}
\begin{tabular}{|c|c|c|c|c|c|}
\cline{2-5}
\multicolumn{1}{c}{} & \multicolumn{4}{|c|}{percentage of members} & \multicolumn{1}{c}{} \\
\hline
\textbf{Module} & \textbf{easy} & \textbf{poly} & \textbf{over} & \textbf{hard} & \textbf{\# members} \\
\hline
base & 35\% & 4\% & 8\% & 53\% & 26 \\ % 100%
\hline
coroutine & 14\% & 0\% & 0\% & 86\% & 7 \\ % 100%
\hline
package & 62\% & 0\% & 0\% & 38\% & 8 \\ % 100%
\hline
string & 75\% & 0\% & 0\% & 25\% & 16 \\ % 100%
\hline
utf8 & 100\% & 0\% & 0\% & 0\% & 6 \\ % 100%
\hline
table & 14\% & 72\% & 14\% & 0\% & 7 \\ % 100%
\hline
math & 81\% & 0\% & 19\% & 0\% & 27 \\ % 100%
\hline
io & 81\% & 0\% & 0\% & 19\% & 21 \\ % 100%
\hline
os & 82\% & 0\% & 18\% & 0\% & 11 \\ % 100%
\hline
\end{tabular}
\end{center}
\caption{Evaluation results for Lua Standard Library}
\label{tab:evallsl}
\end{table}

The \texttt{base} module was very difficult to type
because it includes several functions that rely on reflection,
as the \emph{hard} category shows.
For instance, the functions \texttt{pairs} and \texttt{getmetatable}
are in this category.
While \texttt{pairs} traverses all keys and values that are stored
in a given table, \texttt{getmetatable} returns the metatable of a
given table.

There are some functions in the \texttt{base} module that
we could not give a precise static type because our type
system does not have parametric polymorphism,
as the \emph{poly} category shows.
This is the case of \texttt{ipairs}.

The \texttt{base} module also includes some overloaded functions,
as the \emph{over} category shows.
We could not type these functions because our type system does
not include intersection types.
This is the case of \texttt{tonumber} and \texttt{collectgarbage}.

In the case of \texttt{tonumber}, it has two different types:
\texttt{(value) -> (number)} and \texttt{(string, integer) -> (number)}.
This means that the type of the first parameter depends on the
type of the second parameter.
For instance, we can call \texttt{tonumber(1)}, but we cannot
call \texttt{tonumber(1,2)}.
Note that the first argument of \texttt{tonumber} can be
a value of any type if it is the only argument, but it must
be a string if there is a second argument,
which must be an integer.

In the case of \texttt{collectgarbage}, the return type changes
according to an input literal string.
For instance, calling \texttt{collectgarbage("collect")} returns an integer,
calling \texttt{collectgarbage("count")} returns a floating point,
and calling \texttt{collectgarbage("isrunning")} returns a boolean.

The \texttt{coroutine} module was also very difficult to type
because our type system cannot describe the computational effects
of a program.
The \emph{hard} category shows the amount of functions that we
could not give a precise static type for this reason.
Lua has one-shot delimited continuations \cite{james2011yield}
in the form of \emph{coroutines} \cite{moura2009rc}, and
effect systems \cite{nielson1999type} are an approach that we
could use to describe control transfers with continuations.
However, for now, coroutines are out of scope of our type
system, and we use an empty \texttt{userdata} declaration
to represent the type \texttt{thread}.

Still, we could give a precise static type to one function in
the \texttt{coroutine} module, as the \emph{easy} category shows.
The function \texttt{coroutine.isyieldable} has no input parameters,
and it simply returns a boolean that indicates whether the running
coroutine is yieldable.

We could give precise static types to the constants and functions
of the \texttt{package} module, but we could not give precise static
types to the following tables that it exports:
\texttt{package.loaded}, \texttt{package.preload}, and \texttt{package.loaders}.
The first stores loaded modules, while the others store module \emph{loaders}.
They are difficult to type because their types rely on reflection,
that is, their types depend on the modules a program loads.
For this reason, they are in the \emph{hard} category.

We could give precise static types to most of the functions of the
\texttt{string} module, but we could not give precise static types
to the functions that rely on format strings.
For instance, the type of the arguments that we pass to
\texttt{string.format} must match the format string we are using.
It is fine to call \texttt{string.format("\%d", 1)}, but
\texttt{string.format("\%d", true)} raises a run-time error.
These functions that rely on format strings are in the \emph{hard} category.

The \texttt{utf8} module was straightforward to type,
as its members are only operations over strings.

The \texttt{table} module was specially difficult to type because
most of its functions require parametric polymorphism,
as the \emph{poly} category shows.
These functions either receive or return a list of elements, and
parametric polymorphism would help us to describe them with a generic type.

However, the lack of parametric polymorphism did not prevent us from
giving a precise type to one function of the \texttt{table} module,
as the \emph{easy} category shows.
We could give a precise static type to \texttt{table.concat},
as it operates over lists where all elements are strings or numbers.

Even if our type system had parametric polymorphism, there is
still one function of the \texttt{table} module that we could
not give a precise static type because it is an overloaded function,
as the \emph{over} category shows.
This function is \texttt{table.insert}, and its type depends on
the calling arity.
That is, calling \texttt{table.insert(l, v)} inserts the
element \texttt{v} at the end of the list \texttt{l},
while \texttt{table.insert(l, p, v)} inserts the element \texttt{v}
at the position \texttt{p} of the list \texttt{l},
and generates a run-time error when \texttt{p} is out of bounds.
This function also does not follow the semantics of Lua on
discarding extra arguments, and generates a run-time error whenever
we pass more than three arguments, even if the first three arguments
match its signature.

Even though the \texttt{math} module looks straightforward to type,
the \emph{over} category shows that it includes several overloaded functions.
For instance, the function \texttt{math.random} is in this category
because it has two different types:
\texttt{() -> (number)} and \texttt{(integer, integer?) -> (integer)}.
This means that the type of \texttt{math.random} depends on the calling arity.
Calling \texttt{math.random()} returns a random floating point
between \texttt{0} and \texttt{1}.
Calling \texttt{math.random(10)} is equivalent to \texttt{math.random(1,10)},
and returns an integer between this interval.
Like \texttt{table.insert}, this function also does not follow the
semantics of Lua on discarding extra arguments,
and generates a run-time error whenever we pass more than two arguments.

The \texttt{io} module provides operations for manipulating files,
and these operations can use implicit or explicit file descriptors.
The implicit operations are functions in the \texttt{io} table,
while the explicit operations are methods of a file descriptor.
We used an \texttt{userdata} declaration to introduce the type
\texttt{file} for representing the type of a file descriptor
and its methods.
The evaluation results include both implicit and explicit operations.

We could give precise static types to most of the members of the
\texttt{io} module, but the \emph{hard} category shows that it
includes some members that we could not give a precise static type.
The functions \texttt{io.read} and \texttt{io.lines} are
in the \emph{hard} category along with the methods
\texttt{file:read} and \texttt{file:lines}.

We could not precisely type \texttt{io.read} because its return type
relies on format strings.
For instance, calling \texttt{io.read("l")} returns a string or \texttt{nil},
\texttt{io.read("n")} returns a number or \texttt{nil}, and
\texttt{io.read("l", "n")} returns a string or \texttt{nil} and a number or \texttt{nil}.
The function \texttt{io.lines}, and the methods \texttt{file:read} and
\texttt{file:lines} have the same issue.

There are two functions in the \texttt{os} module that we could
not give a precise static type because they are overloaded functions.
The functions that are in the \emph{over} category are
\texttt{os.date} and \texttt{os.execute}.

The evaluation results show that our type system should include
intersection types, parametric polymorphism, and effect types,
as these features would help us increase the static typing of the
Lua Standard Library.
Intersection types would allow us to define overloaded function types.
Parametric polymorphism would allow us to define generic function and table types.
Effect types would allow us to type coroutines.

\section{MD5}

The MD5 library is an OpenSSL based message digest library for Lua.
It contains just the \texttt{md5} module that is written in C,
so we used Typed Lua's description file to type it.
Table \ref{tab:evalmd5} summarizes the evaluation results for MD5.

\begin{table}[!ht]
\begin{center}
\begin{tabular}{|c|c|c|c|c|c|}
\cline{2-5}
\multicolumn{1}{c}{} & \multicolumn{4}{|c|}{percentage of members} & \multicolumn{1}{c}{} \\
\hline
\textbf{Module} & \textbf{easy} & \textbf{poly} & \textbf{over} & \textbf{hard} & \textbf{\# members} \\
\hline
md5 & 100\% & 0\% & 0\% & 0\% & 13 \\ % 100%
\hline
\end{tabular}
\end{center}
\caption{Evaluation results for MD5}
\label{tab:evalmd5}
\end{table}

Even tough it was straightforward to type the MD5 library,
we found a little difference between its documentation and its behavior.
The documentation suggests that the type of \texttt{md5.update}
is \texttt{(md5\string_context, string) -> (md5\string_context)},
though there is a call to this function in the test script that passes
an extra string argument.
Reading the source code, we found that its actual type is
\texttt{(md5\string_context, string*) -> (md5\string_context)},
that is, we can pass zero or more strings to \texttt{md5.update}.

This case study shows that type annotations help programmers maintain the
documentation updated, as the type checker always validates them.

\section{LuaSocket}

LuaSocket is a library that adds network support to Lua,
and it is split into two parts: a core that is written in C and a set of
Lua modules.
The C core provides TCP and UDP support, while the Lua modules provide
support for SMTP, HTTP, and FTP client protocols, MIME encoding,
URL manipulation, and LTN12 filters \cite{nehab2008ltn012}.
We used Typed Lua's description files to type both parts, as we also
wanted to use LuaSocket to test description files to statically type
the interface of modules that are written in Lua.
Table \ref{tab:evalsocket} summarizes the evaluation results for LuaSocket.

\begin{table}[!ht]
\begin{center}
\begin{tabular}{|c|c|c|c|c|c|}
\cline{2-5}
\multicolumn{1}{c}{} & \multicolumn{4}{|c|}{percentage of members} & \multicolumn{1}{c}{} \\
\hline
\textbf{Module} & \textbf{easy} & \textbf{poly} & \textbf{over} & \textbf{hard} & \textbf{\# members} \\
\hline
socket & 83\% & 0\% & 0\% & 17\% & 60 \\ % 100%
\hline
ftp & 83\% & 0\% & 17\% & 0\% & 6 \\ % 100%
\hline
http & 80\% & 0\% & 20\% & 0\% & 5 \\ % 100%
\hline
smtp & 100\% & 0\% & 0\% & 0\% & 7 \\ % 100%
\hline
mime & 100\% & 0\% & 0\% & 0\% & 17 \\ % 100%
\hline
ltn12 & 95\% & 5\% & 0\% & 0\% & 20 \\ % 100%
\hline
url & 100\% & 0\% & 0\% & 0\% & 8 \\ % 100%
\hline
\end{tabular}
\end{center}
\caption{Evaluation results for LuaSocket}
\label{tab:evalsocket}
\end{table}

We could give precise static types to most of the members
in the \texttt{socket} module, which is the C core.
However, this module includes some functions that
we could not give a precise static type because they
rely on reflection, as the \emph{hard} category shows.
For instance, \texttt{socket.skip} is a function that is
in this category.
We can use this function to choose the number of
values that we want to return.
As an example, calling \texttt{socket.skip(1, nil, "hello")}
returns only the string \texttt{"hello"}, because \texttt{1} indicates
that we do not want to return the first value.
Passing a negative number to \texttt{socket.skip} can be
dangerous, as it returns anything that might be in the stack.
As an example, calling \texttt{socket.skip(-1, nil, "hello")}
returns the tuple \texttt{(-1, nil, "hello")}, because \texttt{-1} makes
\texttt{socket.skip} not skip any values.
As another example, the code \texttt{f = socket.skip(-2)} assigns \texttt{socket.skip}
to \texttt{f}, as \texttt{-2} gets \texttt{socket.skip} from the stack.
Our type system cannot handle the type of negative numbers, as
this requires more complex types such as the refinement types from
hybrid type checking \cite{flanagan2006htc}.

We could give precise static types to most of the members of the
modules \texttt{ftp} and \texttt{http}, but we could not precisely type
two overloaded functions: \texttt{ftp.get} and \texttt{http.request}.

The function \texttt{ftp.get} downloads data from a given URL,
which can be either a string or a table.
More precisely, \texttt{ftp.get(url)} returns the tuple
\texttt{(string) | (nil, string)} if \texttt{url} is a string,
and it returns the tuple \texttt{(number) | (nil, string)} if
\texttt{url} is a table.

The function \texttt{http.request} downloads data from a given URL,
which can be either a string or a table.
More precisely, \texttt{http.request(url, body)} returns the tuple type
\texttt{(string, number, \{string:string\}, number) | (nil, string)}
if \texttt{url} is a string and \texttt{body} is another string or \texttt{nil},
but it returns the tuple type
\texttt{(number, number, \{string:string\}, number) | (nil, string)}
if \texttt{url} is a table and \texttt{body} is \texttt{nil}.

The modules \texttt{mime} and \texttt{ltn12} have a strong connection.
The \texttt{mime} module offers low-level and high-level filters
that apply and remove some text encodings.
The low-level filters are written in C, while the high-level filters
use the function \texttt{ltn12.filter.cycle} along with the low-level
filters to create standard filters.

Even though we could type all the members of the \texttt{mime} module,
the function \texttt{ltn12.filter.cycle} is the only member of the
\texttt{ltn12} module that we could not give a precise type.
This function is difficult to type because it is polymorphic.

The modules \texttt{smtp} and \texttt{url} were straightforward to type.
The \texttt{smtp} module provides functions that send e-mails.
The \texttt{url} module provides functions that manipulate URLs.

\section{LuaFileSystem}

LuaFileSystem is a library that extends the set of functions
for manipulating file systems in Lua.
It contains just the \texttt{lfs} module that is written in C,
so we used Typed Lua's description files to type it.
Table \ref{tab:evallfs} summarizes the evaluation results for LuaFileSystem.

\begin{table}[!ht]
\begin{center}
\begin{tabular}{|c|c|c|c|c|c|}
\cline{2-5}
\multicolumn{1}{c}{} & \multicolumn{4}{|c|}{percentage of members} & \multicolumn{1}{c}{} \\
\hline
\textbf{Module} & \textbf{easy} & \textbf{poly} & \textbf{over} & \textbf{hard} & \textbf{\# members} \\
\hline
lfs & 89\% & 0\% & 11\% & 0\% & 19 \\ % 100%
\hline
\end{tabular}
\end{center}
\caption{Evaluation results for LuaFileSystem}
\label{tab:evallfs}
\end{table}

Even though we could precisely type most of the functions exported
by the \texttt{lfs} module, we could not type two overloaded functions
due to the lack of intersection types in our type system.

\section{HTTP Digest}

The HTTP Digest library implements client side HTTP digest authentication for Lua.
Table \ref{tab:evalhttpdigest} summarizes the evaluation results for HTTP Digest.

\begin{table}[!ht]
\begin{center}
\begin{tabular}{|c|c|c|c|c|c|}
\cline{2-5}
\multicolumn{1}{c}{} & \multicolumn{4}{|c|}{percentage of members} & \multicolumn{1}{c}{} \\
\hline
\textbf{Module} & \textbf{easy} & \textbf{poly} & \textbf{over} & \textbf{hard} & \textbf{\# members} \\
\hline
http-digest & 0\% & 0\% & 100\% & 0\% & 1 \\ % 100%
\hline
\end{tabular}
\end{center}
\caption{Evaluation results for HTTP Digest}
\label{tab:evalhttpdigest}
\end{table}

It is difficult to type the interface of the \texttt{http-digest} module
because it is an extension to the \texttt{http} module from LuaSocket.
The \texttt{http-digest} module only exports the function
\texttt{http-digest.request}, which extends the function
\texttt{http.request} with MD5 authentication.
Like \texttt{http.request}, \texttt{http-digest.request}
is also an overloaded function.

Even though we could not precisely type the interface that \texttt{http-digest}
exports, we could use only static types to annotate this module,
and they pointed a bug in the code.
The problem was related to the way the library was loading the MD5 library
that should be used. 
This part of the code checks the existence of three different MD5 libraries
in the system, and uses the first one that is available, or generates an
error when none is available.
The code that loads the first option was fine, but the code that loads the
second and third options were trying to access an undefined global variable.

\section{Typical}

Typical is a library that extends the behavior of the function \texttt{type}. 
Table \ref{tab:evaltypical} summarizes the evaluation results for Typical.

\begin{table}[!ht]
\begin{center}
\begin{tabular}{|c|c|c|c|c|c|}
\cline{2-5}
\multicolumn{1}{c}{} & \multicolumn{4}{|c|}{percentage of members} & \multicolumn{1}{c}{} \\
\hline
\textbf{Module} & \textbf{easy} & \textbf{poly} & \textbf{over} & \textbf{hard} & \textbf{\# members} \\
\hline
typical & 100\% & 0\% & 0\% & 0\% & 1 \\ % 100%
\hline
\end{tabular}
\end{center}
\caption{Evaluation results for Typical}
\label{tab:evaltypical}
\end{table}

The interface of the \texttt{typical} module is straightforward to type,
as it contains only the function \texttt{typical.type},
which has the same type of the function \texttt{type}: \texttt{(value) -> (string)}.

However, we hit some limitations of our type system while annotating this module.

First, it uses the \texttt{getmetatable} to get a table and
checks whether this table has the field \texttt{\string_\string_type}.
We could not give a precise type to \texttt{getmetatable}, so we used the dynamic
type \texttt{any} as its return type, and this generates a warning.

Second, it returns a metatable that extends \texttt{\string_\string_call}
with \texttt{typical.type}, that is, we can use the module itself as a function,
though it is a table.
Our type system still does not support metatables, so we did not extend our version
of the \texttt{typical} module to support \texttt{\string_\string_call}.

Third, the module uses \texttt{ipairs} to iterate over an array of functions,
but our type system also has limited support to \texttt{ipairs}, and generates
a warning when we try to use the indexed value inside the \texttt{for} body.
As we mentioned in this chapter, we use the dynamic type as a replacement for
the lack of type parameters.
This means that we get warnings inside an \texttt{ipairs} iteration, because
all iterated elements have the dynamic type.
We removed this warning using the numeric \texttt{for} to perform the same loop.

\section{Modulo 11}

Modulo 11 is a library that generates and verifies modulo 11 numbers.
Table \ref{tab:evalmod11} summarizes the evaluation results for Typical.

\begin{table}[!ht]
\begin{center}
\begin{tabular}{|c|c|c|c|c|c|}
\cline{2-5}
\multicolumn{1}{c}{} & \multicolumn{4}{|c|}{percentage of members} & \multicolumn{1}{c}{} \\
\hline
\textbf{Module} & \textbf{easy} & \textbf{poly} & \textbf{over} & \textbf{hard} & \textbf{\# members} \\
\hline
mod11 & 78\% & 0\% & 0\% & 22\% & 9 \\ % 100%
\hline
\end{tabular}
\end{center}
\caption{Evaluation results for Modulo 11}
\label{tab:evalmod11}
\end{table}

The \texttt{mod11} module was written using an object-oriented idiom that
our type system does not support, and that is the reason why we could not
type all the members of its interface.
More precisely, the original code uses \texttt{setmetatable} to hide
two attributes, which our type system cannot hide.

In addition, it returns a metatable that extends \texttt{\string_\string_call}
with the class constructor.
This allows us to use the module itself to create new instances of a
Modulo 11 number.
However, our type system does not support this feature, and we need to
make explicit calls to the constructor whenever we want to create a
new instance.

Even though we had these two issues to annotate the \texttt{mod11} module,
we could use only static types to annotate it, and we found some interesting
points.
The code relies on implicit conversions between strings and numbers,
and some parts of the code keep on changing the type of local variables.
These are two practices that may hide bugs.

\section{Typed Lua Compiler}

The Typed Lua compiler is the last case study that we evaluated.
Table \ref{tab:evaltlc} summarizes its evaluation results.

\begin{table}[!ht]
\begin{center}
\begin{tabular}{|c|c|c|c|c|c|}
\cline{2-5}
\multicolumn{1}{c}{} & \multicolumn{4}{|c|}{percentage of members} & \multicolumn{1}{c}{} \\
\hline
\textbf{Module} & \textbf{easy} & \textbf{poly} & \textbf{over} & \textbf{hard} & \textbf{\# members} \\
\hline
tlast & 98\% & 0\% & 2\% & 0\% & 47 \\ % 100%
\hline
tltype & 100\% & 0\% & 0\% & 0\% & 65 \\ % 100%
\hline
tlst & 100\% & 0\% & 0\% & 0\% & 26 \\ % 100%
\hline
tllexer & 18\% & 0\% & 0\% & 82\% & 11 \\ % 100%
\hline
tlparser & 100\% & 0\% & 0\% & 0\% & 1 \\ % 100%
\hline
tldparser & 100\% & 0\% & 0\% & 0\% & 1 \\ % 100%
\hline
tlchecker & 100\% & 0\% & 0\% & 0\% & 2 \\ % 100%
\hline
tlcode & 100\% & 0\% & 0\% & 0\% & 1 \\ % 100%
\hline
\end{tabular}
\end{center}
\caption{Evaluation results for Typed Lua Compiler}
\label{tab:evaltlc}
\end{table}

The \texttt{tlast} module implements the Abstract Syntax Tree for
the compiler.
We could not precisely type just one function, because it has
an overloaded type that requires intersection types.

The \texttt{tltype} module implements the types introduced by Typed Lua.
It also implements the subtyping and consistent-subtyping relations.
The interface that this module exports was straightforward to type.

The \texttt{tlst} module implements the symbols table for the compiler.
The interface that this module exports was also straightforward to type.

The \texttt{tllexer} module defines common lexical rules for
the Typed Lua parser and the description file parser.
This module is hard to type because it uses LPeg \cite{lpeg,ierusalimschy2009lpeg}
patterns, and LPeg uses overloaded arithmetic operators to
build LPeg patterns. 
Even though LPeg is the third most popular Lua module,
we cannot precisely type LPeg patterns because our
type system still does not support overloading arithmetic
operators.
In the \texttt{tllexer} module, we could only give precise static types
to two error reporting functions that it exports.

The \texttt{tlparser} and \texttt{tldparser} modules implement the Typed Lua
parser and the description file parser, respectively.
Even though they use LPeg to implement the grammar rules,
they only export a parsing function.
Both use LPeg to parse a string and return the corresponding AST.

We could type the interfaces that modules \texttt{tlchecker} and
\texttt{tlcode} export.
The former traverses the AST to perform type checking,
while the latter traverses the AST to perform code generation.

Even though we could precisely type the interface that most of
the modules export, we had issues to write mutually recursive functions.
This kind of functions often appear in compilers construction
to traverse the data structures that they use.
However, Typed Lua still does not support mutually recursive functions.
A way to overcome this limitation was to predeclare these functions
with an empty body, and then redeclare them with their actual body.
The first declaration specifies the function type, while the
second specifies what the function actually does without changing
any type definition.

Traversing the AST would also be problematic if we had not included
a way to discriminate unions of table types, as we mentioned in Section \ref{sec:alias}.
Without a way to discriminate a union of table types, any attempt
to index this union of table types would generate a warning.

Bootstraping the compiler also helped revealing some bugs.
We found some accesses to undeclared global variables and also
to undeclared table fields.
The compiler also helped pointing the places where we should narrow
a nilable value before using it.
In fact, this point appeared in all the case studies that we used
Typed Lua to annotate Lua code.
This means that Lua programmers often use possibly \texttt{nil} values
before checking whether it is \texttt{nil}.



\chapter{Related Work}
\label{chap:related}
In this chapter we review related work, and we split it into two sections:
in the first section we review other Lua projects,
while in the second section we review other projects that are not related to Lua.

\section{Other Lua projects}

Metalua \cite{metalua} is a Lua compiler that supports compile-time
metaprogramming (CTMP).
CTMP is a kind of macro system that allows the programmers to interact
with the compiler \cite{fleutot2007contrasting}.
Metalua extends Lua 5.1 syntax to include its macro system,
and allows programmers to define their own syntax.
Metalua can provide syntactical support for several object-oriented
styles, and can also provide syntax for turning simple type
annotations into run-time assertions.

MoonScript \cite{moonscript} is a programming language that supports
class-based object-oriented programming.
MoonScript compiles to idiomatic Lua code, but
it does not perform compile-time type checking.

LuaInspect \cite{luainspect} is a tool that uses MetaLua to perform
some code analysis.
For instance, it flags unknown global variables and table fields,
it checks the number of function arguments against signatures, and
it infers function return values.
However, it does not try to analyze object-oriented code and
it does not perform compile-time type checking.

Tidal Lock \cite{tidallock} is a prototype of another optional type
system for Lua, which is written in Metalua.
Tidal Lock covers a little subset of Lua.
Statements include declaration of local variables, multiple assignment,
function application, and the return statement.
This means that Tidal Lock does not include any control-flow statement.
Expressions include primitive literals, table indexing, function application,
function declaration, and the table constructor, but they do not include
binary operations.

A remarkable feature of Tidal Lock is the refinement of table types.
This feature inspired us to also include it in Typed Lua,
but in a simpler way and with different formalization.

The table type from Tidal Lock can only represent records, that is,
it cannot describe hash tables and arrays yet, though we can refine them.
Tidal Lock also includes field types to describe the type of the fields
of a table type.
The field types describe if a table field is mutable or immutable
in a table type.
Field types are the feature that allow the refinement of table types in
Tidal Lock.

Tidal Lock is also a structural type system that relies on subtyping and
local type inference.
However, it does not support union types, recursive types, and variadic types.
It also does not type any object-oriented idiom.

Sol \cite{sol} is an experimental optional type system for Lua.
Its type system is similar to ours, as it includes literal types,
union types, and function types that handle variadic functions.
However, it does not handle the refinement of tables and it
includes different types for tables.
Sol types tables as lists, maps, and objects.
Its object types handle a specific object-oriented idiom that
Sol introduces.

Lua Analyzer \cite{luaanalyzer} is an optional type system for Lua
that is specially designed to work in the Löve Studio,
an IDE for game developing using the Löve framework.
It works in Lua 5.1 only, and uses type annotations inside comments.
It is unsound by design because its dynamic type is both
top and bottom in the subtyping relation.

Lua Analyzer shares some features with Typed Lua, and also
has some interesting features that we do not have in Typed Lua.
It has similar rules for handling the \texttt{or} idiom and
discriminating union types inside conditions.
However, these rules are limited to the \texttt{nil} tag only.
It also includes different types for typing tables.
It includes regular record types that maps names to types,
array types, and map types.
Even though it does not support the refinement of tables,
it allows the definition of nominal table types that simulate classes.
This system allows it to type check custom class systems,
which are common in Lua.
Function types also support multiple return values and
variadic functions, but they do not support overloading the
return type.
Recently, it included experimental support for type aliases and generics.

Luacheck \cite{luacheck} is a tool that performs static analysis on Lua code.
It can flag access to undeclared globals and unused local variables,
but it does not perform static type checking.

Ravi \cite{ravi} is an experimental Lua dialect.
Ravi introduces optional static typing for Lua to improve run-time performance.
To do that, Ravi extends the Lua Virtual Machine to include new
operations that take into account static type information.
Currently, Ravi extends the Lua Virtual Machine to support few types:
\texttt{integer}, \texttt{number}, arrays of integers, and arrays of numbers.

\section{Other projects}

Typed Racket \cite{tobin-hochstadt2008ts} is a statically typed version
of the Racket language, which is a Scheme dialect.
The main purpose of Typed Racket is to allow programmers to combine
untyped modules, which are written in Racket, with typed modules, which are
written in Typed Racket.
It also uses local type inference to deduce the type of unannotated expressions.

The main feature of Typed Racket's type system is \emph{occurrence typing}
\cite{tobin-hochstadt2010ltu}.
It is a novel way to use type predicates in control flow statements
to refine union types.
Occurrence typing is not sound in the presence of mutation.
As these kinds of checks are common in other languages, related systems
have appeared \cite{guha2011tlc,winther2011gtp,pearce2013ccf}.

The type system of Typed Racket also includes function types, recursive
types, and structure types.
Its function types also handle multiple return values, and there is
also a way to describe function types that have optional arguments.
Its structure types are similar to our interfaces, as they describe record types.
The type system is also structural and based on subtyping.
It also includes the dynamic type \texttt{Any}, which is the top type in the system.
Typed Racket also supports polymorphic functions and data structures.

Typed Clojure \cite{bonnaire-sergeant2012typed-clojure} is an
optional type system for Clojure.
Although Clojure is a Lisp dialect that runs on the Java Virtual Machine,
Common Language Runtime, and JavaScript, Typed Clojure runs only on
the Java Virtual Machine.
Perhaps, this restriction pushed Typed Clojure to support Java classes
and some Java types such as \texttt{Long}, \texttt{Double}, and \texttt{String}.
Typed Clojure also provides optional type annotations and uses
local type inference to deduce the type of unannotated expressions.
It also assigns the type \texttt{Any} to unannotated function parameters,
which is the top type in the type system.

The type system of Typed Clojure includes polymorphic function types,
union types, intersection types, lists, vectors, maps, sets, and recursive types.
Function types can also have rest parameters, which are similar
to our variadic types, but can only appear on the input parameter
of function types.
In fact, its function types cannot return multiple results.
It also uses occurrence typing to allow control flow statements to
refine union types.
The type system is also structural and based on subtyping.

Dart \cite{dart} is a new class-based object-oriented programming
language.
It includes optional type annotations and compiles to JavaScript.
The type system of Dart is nominal and includes base types,
function types, lists, and maps.
It also supports generics, and the programmer can define
generic functions, lists, and maps.
Unlike Typed Lua, Dart is unsound by design.

Even though Dart has optional typing and static types by
default do not affect run-time semantics, it has an
execution mode that affects run-time.
The \emph{checked mode} inserts run-time assertions that
verifies whether static types match run-time tags.
The \emph{production mode} is the default execution mode
that does not include any assertions.

TypeScript \cite{typescript} is a JavaScript extension
that includes optional type annotations and class-based
object-oriented programming.
It also uses local type inference to deduce the type
of unannotated expressions.
The type system of TypeScript is structural, based
on subtyping, and supports generics.
It includes the dynamic type, primitive types, union types,
function types, array types, tuple types, recursive types, and
object types.
Unlike Typed Lua, TypeScript uses arrays to represent variadic
functions and multiple return values.

Even though TypeScript is unsound by design,
Bierman et al. \cite{bierman2014typescript} shows how to
make TypeScript sound.
They use a reduced core of TypeScript to formalize a
sound type system for TypeScript, but also to formalize
its current unsound type system.

TeJaS \cite{lerner2013tejas} is a framework for the construction of
different type systems for JavaScript.
The authors created a base type system for JavaScript with
extensible typing rules that allow the experimentation of
different static analysis.
They used TeJaS to create a type system that simulates the
type system of TypeScript.

Politz et al. \cite{politz2012semantics} proposes semantics
and types for objects with first-class member names, a well-known
feature from scripting languages.
Their type system uses string patterns to describe the members of
an object, and define a complex subtyping relation to validate
these patterns.
They also provide an implementation of their system to JavaScript.

Gradualtalk \cite{allende2013gts} is a Smalltalk dialect that
supports gradual typing.
The type system combines nominal and structural typing.
It includes function types, union types, structural types,
nominal types, a self type, and parametric polymorphism.
The type system also relies on subtyping and consistent-subtyping.

Gradualtalk inserts run-time checks that ensure dynamically
typed code does not violate statically typed code.
Allende et al. \cite{allende2013cis} perform a careful
evaluation about cast insertion in Gradualtalk.
They report that usually cast insertions impact on execution
performance, so Gradualtalk also has an option that allows
programmers to turn them off, downgrading Gradualtalk
to an optional type system.

Reticulated Python \cite{vitousek2014deg} is a Python compiler
that supports gradual typing.
The type system is structural and based on subtyping.
It includes base types, the dynamic type, list types,
dictionary types, tuple types, function types, set types,
object types, class types, and recursive types.
It includes class and object types to differentiate the
type of class declarations and instances, respectively.
It also uses local type inference.
Besides static type checking, Reticulated Python also introduces
three different approaches for inserting run-time assertions.

Mypy \cite{mypy} is an optional type system for Python.
The type system of mypy is similar to the type system of
Reticulated Python, but mypy does not insert run-time checks
and it has parametric polymorphism.
In contrast, Reticulated Python can type variadic functions,
but mypy cannot.
Recently, Guido van Rossum, Python's author, proposed a
standard syntax for type annotations in Python \cite{PEP483}
that is extremely inspired by mypy \cite{PEP484}.
The main goal of this proposal is to make easier building
static analysis tools for Python.
Typing \cite{typing} is a tool that is being developed to
implement this proposal.

Hack \cite{hack} is a new programming language that runs on the
Hip Hop Virtual Machine (HHVM).
The HHVM is a virtual machine that executes Hack and PHP programs.
We can view Hack as an extension to PHP that combines static and
dynamic typing.
The type system of Hack includes generics, nullable types, collections,
and function types.

The Ruby Type Checker \cite{ren2013rtc} is a library that
performs type checking during run-time.
The library provides type annotations that the programmer
can use on classes and methods.
Its type system includes nominal types, union types,
intersection types, method types, parametric polymorphism,
and type casts.

Grace \cite{black2013sg} is an object-oriented language
with optional typing.
Grace is not a dynamically typed language that has been
extended with an optional type system, but a language
that has been designed from scratch to have both
static and dynamic typing.
Homer et al. \cite{homer2013modules} explores some
useful patterns that derive from Grace's use of objects as modules
and its brand of optional structural typing, which
can also be expressed with Typed Lua's modules as tables.


\chapter{Conclusions}
\label{chap:conc}
In this work we presented Typed Lua, an optional type system for Lua.
We implemented Typed Lua as a Lua extension that allows programmers to
combine static and dynamic typing in Lua code, making easier the evolution
of simple scripts into large programs.

Our main contribution is the formalization of a complete optional type
system that introduces several novel type system features to statically
type check Lua programs.
Even though Lua shares several features with other dynamically
typed languages such as JavaScript, Lua also has several unusual features.
These unusual features include tables (or associative arrays) as the sole
mechanism for structured data, besides functions with multiple return values
and flexible arity that interact with multiple assignment.
We highlight the following novel features of our type system:
\begin{itemize}
\item type refinement allows the incremental evolution of record and
object types, playing an important role in statically type checking
the idiomatic way in which Lua programmers use tables to define modules
and objects;
\item projection types handle functions that are overloaded on the
number and types of return values, allowing programmers to narrow the
types of a set of variables by narrowing the type of a single component
of this set;
\item union types and variadic types help our type system handle
functions with flexible arity, that is, union types are helpful in
describing optional parameters while variadic types are helpful in
describing the type of the vararg expression and the type of functions
that can receive or return any number of values.
\end{itemize}

A key feature in optional type systems is usability.
This means that optional type systems should not change the idioms
that programmers are already familiar with.
Instead, optional type systems should fit existing idioms to
statically type check them.
Designing a too simple type system can overload programmers by forcing
them to change the way they program in the language to fit the type system,
while designing a too complex type system can overload programmers with
types and error messages that are hard to understand, even if type inference
removes the necessity of annotating the program with these complex types.
The most challenging aspect of designing optional type systems is to find
the right amount of complexity for a type system that feels natural to the programmers.

Usability has been a concern in the design of Typed Lua since the beginning.
We realized that we should not rely on the semantics of Lua only,
as this could lead to a cumbersome type system that would not support
several Lua idioms.
For this reason, we performed a mostly automated survey of Lua idioms
and features to inform our design choices.

After designing and implementing Typed Lua, we performed several
case studies to evaluate how successful we were in our goal of
providing an usable type system.
We evaluated 29 modules from 8 different case studies,
and we could give precise static types to 83\% of the 449
members that these modules export.
For half of the modules, we could give precise static types to
at least 89\% of the members from each module.
Our evaluation results showed that our type system can statically
type check several Lua idioms and features, though the evaluation
results also exposed several limitations of our type system.
We found that the three main limitations of our type system are
the lack of intersection types, parametric polymorphism, and operator overloading.
Overcoming these limitations is our major target for future work,
as it will allow us to statically type check more programs.

Unlike other optional type systems, we designed Typed Lua without
deliberate unsound parts.
However, we still do not have proofs that the novel features of
our type system are sound.
We see a soundness proof as another major future work, as it is
necessary to use static types for code optimization.

Finally, we believe that Typed Lua is a major contribution to the Lua community,
because it offers a framework that programmers can use to document,
test, and better structure their applications.
For libraries where a full conversion to static type checking should
prove unfeasible or too much work, the community can use Typed Lua
just to document the external interfaces of the libraries,
giving the benefits of static type checking to the users of these
libraries.
In fact, we already have user feedback from Lua programmers that are
using Typed Lua in their projects.
For instance, ZeroBrane Studio is an IDE for Lua development that is
evaluating the use of Typed Lua to perform static analysis in Lua code.



\bibliography{thesis_andre}

\appendix

\chapter{Glossary}
\label{app:glossary}
\begin{description}
\item[bottom type] A type that is subtype of all types.

\item[closed table type] A table type that does not provide any guarantees
about keys with types not listed in the table type.
See complete definition in page \pageref{def:tabletype}.

\item[coercion] A relation that allows converting values from one type to
values of another type without error.

\item[consistency] A relation used by gradual typing to check the interaction
between the dynamic type and other types.
See complete definition in page \pageref{def:consistency}.

\item[consistent-subtyping] A relation that combines consistency and subtyping.
See complete definition in page \pageref{def:consistent-subtyping}.

\item[contravariant] A part of a type constructor is contravariant when it
reverses the subtyping order, that is, the part $T_{1}$ of a type being
a subtype of the corresponding part $T_{2}$ of another type implies that
$T_{2} \subtype T_{1}$.

\item[covariant] A part of a type constructor is covariant when it
preserves the subtyping order, that is, the part $T_{1}$ of a type being
a subtype of the corresponding part $T_{2}$ of another type implies that
$T_{1} \subtype T_{2}$.

\item[depth subtyping] A subtyping relation over records that allows variance
in the type of record fields.

\item[dynamic type] A type used by gradual typing to denote unknown values.
See complete definition in page \pageref{def:dynamictype}.

\item[filter type] A type used by Typed Lua to discriminate the type of
local variables inside control flow statements.
See complete definition in page \pageref{def:filtertype}.

\item[fixed table type] A table type which guarantees that there are no
keys with a type that is not one of its key types, and that can have
any number of \emph{fixed} or \emph{closed} references.
See complete definition in page \pageref{def:tabletype}.

\item[flow typing] An approach that combines static typing and flow analysis to
allow variables to have different types at different parts of the program.

\item[free assigned variable] A free variable that appears in an assignment.

\item[gradual type system] A type system that uses either the consistency relation
or the consistent-subtyping relation instead of type equality to perform
static type checking.
See complete definition in page \pageref{sec:gradual}.

\item[gradual typing] An approach that uses a gradual type system to allow
static and dynamic typing in the same code, but inserting run-time checks
between statically typed and dynamically typed code.
See complete definition in page \pageref{sec:gradual}.

\item[invariant] A part of a type constructor is invariant when it forbids variance,
that is, the part $T_{1}$ of a type being a subtype of the corresponding part $T_{2}$ of
another type implies that $T_{1} \subtype T_{2}$ and $T_{2} \subtype T_{1}$.
It is also a way to define type equality through subtyping.

\item[metatable] A Lua table that allows changing the behavior of other tables
it is attached to.

\item[nominal type system] A type system that uses the type names to check the
compatibility among them.

\item[open table type] A table type which guarantees that there are no
keys with a type that is not one of its key types, and that only have
\emph{closed} references.
See complete definition in page \pageref{def:tabletype}.

\item[optional type system] A type system that allows combining static and
dynamic typing in the same language, but without affecting the run-time semantics.
See complete definition in page \pageref{sec:optional}.

\item[projection environment] An environment used by Typed Lua to handle unions of
second-level types that are bound to projection types.

\item[projection type] A type used by Typed Lua to discriminate the type of local
variables that have a dependency relation.
See complete definition in page \pageref{def:projectiontype}.

\item[prototype object] An object that works like a class, that is, it is an object from
which other objects inherit its attributes.

\item[self-like delegation] A technique to implement inheritance in dynamically typed
languages through prototype objects.

\item[sound type system] A type system that does not type check all programs that contain a type error.

\item[structural type system] A type system that uses type structures to check the compatibility among them.

\item[table refinement] An operation from Typed Lua that allows programmers to change a table type
to include new fields or to specialize existing fields.
See complete definition in page \pageref{sec:refinement}.

\item[top type] A type that is supertype of all types.

\item[type environment] An environment that binds variable names to types.

\item[type tag] A tag that describes the type of a value during run-time in dynamically
typed languages.

\item[unique table type] A table type which guarantees that there are no
keys with a type that is not one of its key types, and that has only one reference.
See complete definition in page \pageref{def:tabletype}.

\item[unsound type system] A type system that type checks certain programs that contain type errors.

\item[userdata] A Lua data type that allows Lua to hold values from applications
or libraries that are written in C.

\item[vararg expression] A Lua expression that can result in an arbitrary number of values.

\item[variadic function] A function that can use an arbitrary number of arguments.

\item[variance] A property that defines the subtyping order between the components
of a type constructor.

\item[width subtyping] A subtyping relation over records that allows the subtype to
include fields that do not exist in the supertype.

\end{description}


\chapter{The syntax of Typed Lua}
\label{app:syntax}
This appendix presents the complete syntax of Typed Lua.
\allowdisplaybreaks
\begin{align*}
\textit{chunk} ::= & \;\; \textit{block}\\
\textit{block} ::= & \;\; \{\textit{stat}\} \; [\textit{retstat}]\\
\textit{stat} ::= & \;\; \texttt{`;'}\\
& | \; \textit{varlist} \; \texttt{`='} \; \textit{explist}\\
& | \; \textit{functioncall}\\
& | \; \textit{label}\\
& | \; \textbf{break}\\ 
& | \; \textbf{goto} \; \textit{Name}\\
& | \; \textbf{do} \; \textit{block} \; \textbf{end}\\
& | \; \textbf{while} \; \textit{exp} \; \textbf{do} \; \textit{block} \; \textbf{end}\\
& | \; \textbf{repeat} \; \textit{block} \; \textbf{until} \; \textit{exp}\\
& | \; \textbf{if} \; \textit{exp} \; \textbf{then} \; \textit{block} \;
  \{\textbf{elseif} \; \textit{exp} \; \textbf{then} \; \textit{block}\} \;
  [\textbf{else} \; \textit{block}] \; \textbf{end}\\ 
& | \; \textbf{for} \; \textit{Name} \; \texttt{`='} \; \textit{exp} \;
  \texttt{`,'} \; \textit{exp} \; [\texttt{`,'} \; \textit{exp}] \;
  \textbf{do} \; \textit{block} \; \textbf{end}\\
& | \; \textbf{for} \; \textit{namelist} \; \textbf{in} \; \textit{explist} \;
  \textbf{do} \; \textit{block} \; \textbf{end}\\
& | \; [\textbf{const}] \; \textbf{function} \; \textit{funcname} \; \textit{funcbody}\\
& | \; \textbf{local} \; \textbf{function} \; \textit{Name} \; \textit{funcbody}\\
& | \; \textbf{local} \; \textit{namelist} \; [\texttt{`='} \; \textit{explist}]\\
& | \; [\textbf{local}] \; \textbf{typealias} \; \textit{Name} \; \texttt{`='} \; \textit{type}\\
& | \; [\textbf{local}] \; \textbf{interface} \; \textit{typedec}\\
\textit{retstat} ::= & \;\; \textbf{return} \; [\textit{explist}] \; [\texttt{`;'}]\\
\textit{label} ::= & \;\; \texttt{`::'} \; \textit{Name} \; \texttt{`::'}\\
\textit{funcname} ::= & \;\; \textit{Name} \; \{\texttt{`.'} \; \textit{Name}\} \; [\texttt{`:'} \; \textit{Name}]\\
\textit{varlist} ::= & \;\; [\textbf{const}] \; \textit{var} \;
  \{\texttt{`,'} \; [\textbf{const}] \; \textit{var}\}\\
\textit{var} ::= & \;\; \textit{Name} \; | \;
  \textit{prefixexp} \; \texttt{`['} \; \textit{exp} \; \texttt{`]'} \; | \;
  \textit{prefixexp} \; \texttt{`.'} \; \textit{Name}\\
\textit{namelist} ::= & \;\; \textit{Name} \; [\texttt{`:'} \; \textit{type}] \;
  \{\texttt{`,'} \; \textit{Name} \; [\texttt{`:'} \; \textit{type}]\}\\
\textit{explist} ::= & \;\; \textit{exp} \; \{\texttt{`,'} \; \textit{exp}\}\\
\textit{exp} ::= & \;\; \textbf{nil} \; | \;
  \textbf{false} \; | \;
  \textbf{true} \; | \;
  \textit{Number} \; | \;
  \textit{String} \; | \;
  \texttt{`...'} \; | \;
  \textit{functiondef}\\
& | \; \textit{prefixexp} \; | \;
  \textit{tableconstructor} \; | \;
  \textit{exp} \; \textit{binop} \; \textit{exp} \; | \;
  \textit{unop} \; \textit{exp}\\
\textit{prefixexp} ::= & \;\; \textit{var} \; | \;
  \textit{functioncall} \; | \;
  \texttt{`('} \; \textit{exp} \; \texttt{`)'}\\
\textit{functioncall} ::= & \;\; \textit{prefixexp} \; \textit{args} \; | \;
  \textit{prefixexp} \; \texttt{`:'} \; \textit{Name} \; \textit{args}\\
\textit{args} ::= & \;\; \texttt{`('} \; [\textit{explist}] \; \texttt{`)'} \; | \;
  \textit{tableconstructor} \; | \;
  \textit{String}\\
\textit{functiondef} ::= & \;\; \textbf{function} \; \textit{funcbody}\\
\textit{funcbody} ::= & \;\; \texttt{`('} \; [\textit{parlist}] \; \texttt{`)'} \;
  [\texttt{`:'} \; \textit{rettype}] \; \textit{block} \; \textbf{end}\\
\textit{parlist} ::= & \;\; \textit{namelist} \; [\texttt{`,'} \; \texttt{`...'} \;
  [\texttt{`:'} \; \textit{type}]] \; | \;
  \texttt{`...'} \; [\texttt{`:'} \; \textit{type}]\\
\textit{tableconstructor} ::= & \;\; \texttt{`\{'} \; [\textit{fieldlist}] \; \texttt{`\}'}\\
\textit{fieldlist} ::= & \;\; [\textbf{const}] \; \textit{field} \;
  \{\textit{fieldsep} \; [\textbf{const}] \; \textit{field}\} \; [\textit{fieldsep}]\\
\textit{field} ::= & \;\; \texttt{`['} \; \textit{exp} \; \texttt{`]'} \; \texttt{`='} \; \textit{exp} \; | \;
  \textit{Name} \; \texttt{`='} \; \textit{exp} \; | \;
  \textit{exp}\\
\textit{fieldsep} ::= & \;\; \texttt{`,'} \; | \; \texttt{`;'}\\
\textit{binop} ::= & \;\; \texttt{`+'} \; | \; \texttt{`-'} \; | \; \texttt{`*'} \; | \; \texttt{`/'} \; | \;
  \texttt{`//'} \; | \; \texttt{`\textasciicircum'} \; | \; \texttt{`\%'}\\
& | \; \texttt{`\&'} \; | \; \texttt{`\textasciitilde'} \; | \; \texttt{`|'} \; | \;
  \texttt{`>>'} \; | \; \texttt{`<<'} \; | \; \texttt{`..'}\\
& | \; \texttt{`<'} \; | \; \texttt{`<='} \; | \; \texttt{`>'} \; | \; \texttt{`>='} \; | \;
  \texttt{`=='} \; | \; \texttt{`\textasciitilde='}\\
& | \; \textbf{and} \; | \; \textbf{or}\\
\textit{unop} ::= & \;\; \texttt{`-'} \; | \; \textbf{not} \; | \; \texttt{`\#'} \; | \; \texttt{`\textasciitilde'}\\
\textit{typedec} ::= & \;\; \textit{Name} \; \{\textit{decitem}\} \; \textbf{end}\\
\textit{decitem} ::= & \;\; \textit{idlist} \; \texttt{`:'} \; \textit{idtype}\\
\textit{idtype} ::= & \;\; \textit{type} \; | \; \textit{methodtype}\\
\textit{idlist} ::= & \;\; \textit{id} \; \{\texttt{`,'} \; \textit{id}\}\\
\textit{id} ::= & \;\; [\textbf{const}] \; \textit{Name}\\
\textit{type} ::= & \;\; \textit{primarytype} \; [\texttt{`?'}]\\
\textit{primarytype} ::= & \;\; \textit{literaltype} \; | \;
  \textit{basetype} \; | \;
  \textbf{nil} \; | \;
  \textbf{value} \; | \;
  \textbf{any} \; | \;
  \textbf{self} \; | \;
  \textit{Name}\\
& | \; \textit{functiontype} \; | \;
  \textit{tabletype} \; | \;
  \textit{primarytype} \; \texttt{`|'} \; \textit{primarytype}\\
\textit{literaltype} ::= & \;\; \textbf{false} \; | \;
  \textbf{true} \; | \;
  \textit{Int} \; | \;
  \textit{Float} \; | \;
  \textit{String}\\
\textit{basetype} ::= & \;\; \textbf{boolean} \; | \;
  \textbf{integer} \; | \;
  \textbf{number} \; | \;
  \textbf{string}\\
\textit{functiontype} ::= & \;\; \textit{tupletype} \; \texttt{`->'} \; \textit{rettype}\\
\textit{tupletype} ::= & \;\; \texttt{`('} \; [typelist] \; \texttt{`)'}\\
\textit{typelist} ::= & \;\; \textit{type} \; \{\texttt{`,'} \; \textit{type}\} \; [\texttt{`*'}]\\
\textit{rettype} ::= & \;\; \textit{type} \; | \;
  \textit{uniontuple} \; [\texttt{`?'}]\\
\textit{uniontuple} ::= & \;\; \textit{tupletype} \; | \;
  \textit{uniontuple} \; \texttt{`|'} \; \textit{uniontuple}\\
\textit{tabletype} ::= & \;\; \texttt{`\{'} \; [\textit{tabletypebody}] \; \texttt{`\}'}\\
\textit{tabletypebody} ::= & \;\; \textit{maptype} \; | \;
  \textit{recordtype}\\
\textit{maptype} ::= & \;\; [\textit{keytype} \; \texttt{`:'}] \; \textit{type}\\
\textit{keytype} ::= & \;\; \textit{basetype} \; | \;
  \textbf{value}\\
\textit{recordtype} ::= & \;\; \textit{recordfield} \; \{\texttt{`,'} \; \textit{recordfield}\} \; [\texttt{`,'} \; \textit{type}]\\
\textit{recordfield} ::= & \;\; [\textbf{const}] \; \textit{literaltype} \; \texttt{`:'} \; \textit{type}\\
\textit{methodtype} ::= & \;\; \textit{tupletype} \; \texttt{`=>'} \; \textit{rettype}
\end{align*}


\chapter{The type system of Typed Lua}
\label{app:rules}
This appendix presents the complete type system of Typed Lua.

\section{Subtyping rules}

\noindent

\mylabel{S-LITERAL}
\[
\senv \vdash L \subtype L
\]

\mylabel{S-FALSE}
\[
\senv \vdash \False \subtype \Boolean
\]

\mylabel{S-TRUE}
\[
\senv \vdash \True \subtype \Boolean
\]

\mylabel{S-STRING}
\[
\senv \vdash {\it string} \subtype \String
\]

\mylabel{S-INT1}
\[
\senv \vdash {\it int} \subtype \Integer
\]

\mylabel{S-INT2}
\[
\senv \vdash {\it int} \subtype \Number
\]

\mylabel{S-FLOAT}
\[
\senv \vdash {\it float} \subtype \Number
\]

\mylabel{S-BASE}
\[
\senv \vdash B \subtype B
\]

\mylabel{S-INTEGER}
\[
\senv \vdash \Integer \subtype \Number
\]

\mylabel{S-NIL}
\[
\senv \vdash \Nil \subtype \Nil
\]

\mylabel{S-VALUE}
\[
\senv \vdash F \subtype \Value
\]

\mylabel{S-ANY}
\[
\senv \vdash \Any \subtype \Any
\]

\mylabel{S-SELF}
\[
\senv \vdash \Self \subtype \Self
\]

\mylabel{S-UNION1}
\[
\dfrac{\senv \vdash F_{1} \subtype F \;\;\;
       \senv \vdash F_{2} \subtype F}
      {\senv \vdash F_{1} \cup F_{2} \subtype F}
\]

\mylabel{S-UNION2}
\[
\dfrac{\senv \vdash F \subtype F_{1}}
      {\senv \vdash F \subtype F_{1} \cup F_{2}}
\]

\mylabel{S-UNION3}
\[
\dfrac{\senv \vdash F \subtype F_{2}}
      {\senv \vdash F \subtype F_{1} \cup F_{2}}
\]

\mylabel{S-FUNCTION}
\[
\dfrac{\senv \vdash S_{3} \subtype S_{1} \;\;\;
       \senv \vdash S_{2} \subtype S_{4}}
      {\senv \vdash S_{1} \rightarrow S_{2} \subtype S_{3} \rightarrow S_{4}}
\]

\mylabel{S-PAIR}
\[
\dfrac{\senv \vdash F_{1} \subtype F_{2} \;\;\;
       \senv \vdash P_{1} \subtype P_{2}}
      {\senv \vdash F_{1} \times P_{1} \subtype F_{2} \times P_{2}}
\]

\mylabel{S-VARARG1}
\[
\dfrac{\senv \vdash F_{1} \cup \Nil \subtype F_{2} \cup \Nil}
      {\senv \vdash F_{1}{*} \subtype F_{2}{*}}
\]

\mylabel{S-VARARG2}
\[
\dfrac{\senv \vdash F_{1} \cup \Nil \subtype F_{2} \;\;\;
       \senv \vdash F_{1}{*} \subtype P_{2}}
      {\senv \vdash F_{1}{*} \subtype F_{2} \times P_{2}}
\]

\mylabel{S-VARARG3}
\[
\dfrac{\senv \vdash F_{1} \subtype F_{2} \cup \Nil \;\;\;
       \senv \vdash P_{1} \subtype F_{2}{*}}
      {\senv \vdash F_{1} \times P_{1} \subtype F_{2}{*}}
\]

\mylabel{S-UNION4}
\[
\dfrac{\senv \vdash S_{1} \subtype S \;\;\;
       \senv \vdash S_{2} \subtype S}
      {\senv \vdash S_{1} \sqcup S_{2} \subtype S}
\]

\mylabel{S-UNION5}
\[
\dfrac{\senv \vdash S \subtype S_{1}}
      {\senv \vdash S \subtype S_{1} \sqcup S_{2}}
\]

\mylabel{S-UNION6}
\[
\dfrac{\senv \vdash S \subtype S_{2}}
      {\senv \vdash S \subtype S_{1} \sqcup S_{2}}
\]

\mylabel{S-TABLE1}
\[
\dfrac{\forall i \; \exists j \;\;\;
       \senv \vdash F_{j} \subtype F_{i}' \;\;\;
       \senv \vdash F_{i}' \subtype F_{j} \;\;\;
       \senv \vdash V_{j} \subtype_{c} V_{i}'}
      {\senv \vdash \{\overline{F{:}V}\}_{fixed|closed} \subtype
                    \{\overline{F'{:}V'}\}_{closed}}
\]

\mylabel{S-TABLE2}
\[
\dfrac{\begin{array}{c}
       \forall i \; \forall j \;\;\;
       \senv \vdash F_{i} \subtype F_{j}' \to \senv \vdash V_{i} \subtype_{u} V_{j}'\\
       \forall j \; \nexists i \;\;\;
       \senv \vdash F_{i} \subtype F_{j}' \to \senv \vdash \Nil \subtype_{o} V_{j}'
       \end{array}}
      {\senv \vdash \{\overline{F{:}V}\}_{unique} \subtype
                    \{\overline{F'{:}V'}\}_{closed}}
\]

\mylabel{S-TABLE3}
\[
\dfrac{\begin{array}{c}
       \forall i \; \exists j \;\;\;
       \senv \vdash F_{i} \subtype F_{j}' \land \senv \vdash V_{i} \subtype_{u} V_{j}' \\
       \forall j \; \nexists i \;\;\;
       \senv \vdash F_{i} \subtype F_{j}' \to \senv \vdash \Nil \subtype_{o} V_{j}'
       \end{array}}
      {\senv \vdash \{\overline{F{:}V}\}_{unique} \subtype
                    \{\overline{F'{:}V'}\}_{unique|open|fixed}}
\]

\mylabel{S-TABLE4}
\[
\dfrac{\begin{array}{c}
       \forall i \; \forall j \;\;\;
       \senv \vdash F_{i} \subtype F_{j}' \to \senv \vdash V_{i} \subtype_{c} V_{j}' \\
       \forall j \; \nexists i \;\;\;
       \senv \vdash F_{i} \subtype F_{j}' \to \senv \vdash \Nil \subtype_{o} V_{j}'
       \end{array}}
      {\senv \vdash \{\overline{F{:}V}\}_{open} \subtype
                    \{\overline{F'{:}V'}\}_{closed}}
\]

\mylabel{S-TABLE5}
\[
\dfrac{\begin{array}{c}
       \forall i \; \exists j \;\;\;
       \senv \vdash F_{i} \subtype F_{j}' \land \senv \vdash V_{i} \subtype_{c} V_{j}' \\
       \forall j \; \nexists i \;\;\;
       \senv \vdash F_{i} \subtype F_{j}' \to \senv \vdash \Nil \subtype_{o} V_{j}'
       \end{array}}
      {\senv \vdash \{\overline{F{:}V}\}_{open} \subtype
                    \{\overline{F'{:}V'}\}_{open|fixed}}
\]

\mylabel{S-TABLE6}
\[
\dfrac{\begin{array}{c}
       \forall i \; \exists j \;\;\;
       \senv \vdash F_{i} \subtype F_{j}' \;\;\;
       \senv \vdash F_{j}' \subtype F_{i} \;\;\;
       \senv \vdash V_{i} \subtype_{c} V_{j}' \\
       \forall j \; \exists i \;\;\;
       \senv \vdash F_{i} \subtype F_{j}' \;\;\;
       \senv \vdash F_{j}' \subtype F_{i} \;\;\;
       \senv \vdash V_{i} \subtype_{c} V_{j}' \\
       \end{array}}
      {\senv \vdash \{\overline{F{:}V}\}_{fixed} \subtype
                    \{\overline{F'{:}V'}\}_{fixed}}
\]

\mylabel{S-FIELD1}
\[
\dfrac{\senv \vdash F_{1} \subtype F_{2} \;\;\;
       \senv \vdash F_{2} \subtype F_{1}}
      {\senv \vdash F_{1} \subtype_{c} F_{2}}
\]

\mylabel{S-FIELD2}
\[
\dfrac{\senv \vdash F_{1} \subtype F_{2}}
      {\senv \vdash \Const \; F_{1} \subtype_{c} \Const \; F_{2}}
\]

\mylabel{S-FIELD3}
\[
\dfrac{\senv \vdash F_{1} \subtype F_{2}}
      {\senv \vdash F_{1} \subtype_{c} \Const \; F_{2}}
\]

\mylabel{S-FIELD4}
\[
\dfrac{\senv \vdash F_{1} \subtype F_{2}}
      {\senv \vdash F_{1} \subtype_{u} F_{2}}
\]

\mylabel{S-FIELD5}
\[
\dfrac{\senv \vdash F_{1} \subtype F_{2}}
      {\senv \vdash \Const \; F_{1} \subtype_{u} \Const \; F_{2}}
\]

\mylabel{S-FIELD6}
\[
\dfrac{\senv \vdash F_{1} \subtype F_{2}}
      {\senv \vdash \Const \; F_{1} \subtype_{u} F_{2}}
\]

\mylabel{S-FIELD7}
\[
\dfrac{\senv \vdash F_{1} \subtype F_{2}}
      {\senv \vdash F_{1} \subtype_{u} \Const \; F_{2}}
\]

\mylabel{S-FIELD8}
\[
\dfrac{\senv \vdash \Nil \subtype F}
      {\senv \vdash \Nil \subtype_{o} F}
\]

\mylabel{S-FIELD9}
\[
\dfrac{\senv \vdash \Nil \subtype F}
      {\senv \vdash \Nil \subtype_{o} \Const \; F}
\]

\mylabel{S-AMBER}
\[
\dfrac{\senv[x_{1} \subtype x_{2}] \vdash F_{1} \subtype F_{2}}
      {\senv \vdash \mu x_{1}.F_{1} \subtype \mu x_{2}.F_{2}}
\]

\mylabel{S-ASSUMPTION}
\[
\dfrac{x_{1} \subtype x_{2} \in \senv}
      {\senv \vdash x_{1} \subtype x_{2}}
\]

\mylabel{S-UNFOLDR}
\[
\dfrac{\senv \vdash F_{1} \subtype [x \mapsto \mu x.F_{2}]F_{2}}
      {\senv \vdash F_{1} \subtype \mu x.F_{2}}
\]

\mylabel{S-UNFOLDL}
\[
\dfrac{\senv \vdash [x \mapsto \mu x.F_{1}]F_{1} \subtype F_{2}}
      {\senv \vdash \mu x.F_{1} \subtype F_{2}}
\]

\mylabel{S-EXPRESSION}
\[
\senv \vdash T \subtype T
\]

\mylabel{S-PAIR2}
\[
\dfrac{\senv \vdash T_{1} \subtype T_{2} \;\;\;
       \senv \vdash E_{1} \subtype E_{2}}
      {\senv \vdash T_{1} \times E_{1} \subtype T_{2} \times E_{2}}
\]

\mylabel{S-VARARG4}
\[
\dfrac{\senv \vdash T_{1} \cup \Nil \subtype T_{2} \cup \Nil}
      {\senv \vdash T_{1}{*} \subtype T_{2}{*}}
\]

\mylabel{S-VARARG5}
\[
\dfrac{\senv \vdash T_{1} \cup \Nil \subtype T_{2} \;\;\;
       \senv \vdash T_{1}{*} \subtype E_{2}}
      {\senv \vdash T_{1}{*} \subtype T_{2} \times E_{2}}
\]

\mylabel{S-VARARG6}
\[
\dfrac{\senv \vdash T_{1} \subtype T_{2} \cup \Nil \;\;\;
       \senv \vdash E_{1} \subtype T_{2}{*}}
      {\senv \vdash T_{1} \times E_{1} \subtype T_{2}{*}}
\]

\mylabel{C-ANY1}
\[
\senv \vdash F \lesssim \Any
\]

\mylabel{C-ANY2}
\[
\senv \vdash \Any \lesssim F
\]

\section{Typing rules}

\noindent

\mylabel{T-SKIP}
\[
\env_{1}, \penv \vdash \mathbf{skip}, \env_{1}
\]

\mylabel{T-SEQ}
\[
\dfrac{\env_{1}, \penv \vdash s_{1}, \env_{2} \;\;\;
       \env_{2}, \penv \vdash s_{2}, \env_{3}}
      {\env_{1}, \penv \vdash s_{1} \; ; \; s_{2}, \env_{3}}
\]

\mylabel{T-ASSIGNMENT}
\[
\dfrac{\env_{1}, \penv \vdash el:S_{1}, \env_{2} \;\;\;
       \env_{2}, \penv \vdash \overline{l}:S_{2}, \env_{3} \;\;\;
       S_{1} \lesssim S_{2}}
      {\env_{1}, \penv \vdash \overline{l} = el,\env_{3}}
\]

\mylabel{T-METHOD1}
\[
\dfrac{\begin{array}{c}
       \env_{1}(id_{1}) = F_{s} \;\;\;
       F_{s} = \{\overline{F{:}V}\}_{unique} \\
       \env_{1}, \penv \vdash id_{2} : L, \env_{2} \;\;\;
       \nexists i \in 1..n \; L \lesssim F_{i} \;\;\;
       n = |\overline{F{:}V}| \\
       closeall(\env_{1})[self \mapsto \Self, \overline{id \mapsto F}, \self \mapsto F_{s}],
       \penv[\ret \mapsto S] \vdash s, \env_{3}\\
       F_{o} = \{\overline{F{:}V}, L{:}\Const \; \Self \times F_{1} \times ... \times F_{n} \times \Nil{*} \rightarrow S\}_{unique}\\
       \env_{4} = openset(\env_{1}[id_{1} \mapsto F_{o}], frv(\mathbf{fun} \; (\overline{id{:}T}){:}S \; s)) \\
       \env_{5} = closeset(\env_{4}, fav(\mathbf{fun} \; (\overline{id{:}T}){:}S \; s))
       \end{array}}
      {\begin{array}{c}
       \env_{1}, \penv \vdash \mathbf{fun} \; id_{1}{:}id_{2} \; (\overline{id{:}F}){:}S \; s, \env_{5}\\
       \end{array}}
\]

\mylabel{T-METHOD2}
\[
\dfrac{\begin{array}{c}
       \env_{1}(id_{1}) = F_{s} \;\;\;
       F_{s} = \{\overline{F{:}V}\}_{open} \\
       \env_{1}, \penv \vdash id_{2} : L, \env_{2} \;\;\;
       \nexists i \in 1..n \; L \lesssim F_{i} \;\;\;
       n = |\overline{F{:}V}| \\
       closeall(\env_{1})[self \mapsto \Self, \overline{id \mapsto F}, \self \mapsto F_{s}],
       \penv[\ret \mapsto S] \vdash s, \env_{3}\\
       F_{o} = \{\overline{F{:}V}, L{:}\Const \; \Self \times F_{1} \times ... \times F_{n} \times \Nil{*} \rightarrow S\}_{open}\\
       \env_{4} = openset(\env_{1}[id_{1} \mapsto F_{o}], frv(\mathbf{fun} \; (\overline{id{:}T}){:}S \; s)) \\
       \env_{5} = closeset(\env_{4}, fav(\mathbf{fun} \; (\overline{id{:}T}){:}S \; s))
       \end{array}}
      {\begin{array}{c}
       \env_{1}, \penv \vdash \mathbf{fun} \; id_{1}{:}id_{2} \; (\overline{id{:}F}){:}S \; s, \env_{5}\\
       \end{array}}
\]

\mylabel{T-METHOD3}
\[
\dfrac{\begin{array}{c}
       \env_{1}(id_{1}) = F_{s} \;\;\;
       F_{s} = \{\overline{F{:}V}\}_{unique} \\
       \env_{1}, \penv \vdash id_{2} : L, \env_{2} \;\;\;
       \nexists i \in 1..n \; L \lesssim F_{i} \;\;\;
       n = |\overline{F{:}V}| \\
       closeall(\env_{1})[self \mapsto \Self, \overline{id \mapsto F}, {...} \mapsto F, \self \mapsto F_{s}],
       \penv[\ret \mapsto S] \vdash s, \env_{3}\\
       F_{o} = \{\overline{F{:}V}, L{:}\Const \; \Self \times F_{1} \times ... \times F_{n} \times F{*} \rightarrow S\}_{unique}\\
       \env_{4} = openset(\env_{1}[id_{1} \mapsto F_{o}], frv(\mathbf{fun} \; (\overline{id{:}T}){:}S \; s)) \\
       \env_{5} = closeset(\env_{4}, fav(\mathbf{fun} \; (\overline{id{:}T}){:}S \; s))
       \end{array}}
      {\begin{array}{c}
       \env_{1}, \penv \vdash \mathbf{fun} \; id_{1}{:}id_{2} \; (\overline{id{:}F},{...}{:}F){:}S \; s, \env_{5}\\
       \end{array}}
\]

\mylabel{T-METHOD4}
\[
\dfrac{\begin{array}{c}
       \env_{1}(id_{1}) = F_{s} \;\;\;
       F_{s} = \{\overline{F{:}V}\}_{open} \\
       \env_{1}, \penv \vdash id_{2} : L, \env_{2} \;\;\;
       \nexists i \in 1..n \; L \lesssim F_{i} \;\;\;
       n = |\overline{F{:}V}| \\
       closeall(\env_{1})[self \mapsto \Self, \overline{id \mapsto F}, {...} \mapsto F, \self \mapsto F_{s}],
       \penv[\ret \mapsto S] \vdash s, \env_{3}\\
       F_{o} = \{\overline{F{:}V}, L{:}\Const \; \Self \times F_{1} \times ... \times F_{n} \times F{*} \rightarrow S\}_{open}\\
       \env_{4} = openset(\env_{1}[id_{1} \mapsto F_{o}], frv(\mathbf{fun} \; (\overline{id{:}T}){:}S \; s)) \\
       \env_{5} = closeset(\env_{4}, fav(\mathbf{fun} \; (\overline{id{:}T}){:}S \; s))
       \end{array}}
      {\begin{array}{c}
       \env_{1}, \penv \vdash \mathbf{fun} \; id_{1}{:}id_{2} \; (\overline{id{:}F},{...}{:}F){:}S \; s, \env_{5}\\
       \end{array}}
\]

\mylabel{T-METHOD5}
\[
\dfrac{\begin{array}{c}
       \env_{1}(id_{1}) = F_{s} \;\;\;
       F_{s} = \{\overline{F{:}V}\}_{unique} \;\;\;
       \env_{1}, \penv \vdash id_{2} : L, \env_{2} \;\;\;
       n = |\overline{F{:}V}| \\
       \exists i \in 1..n \; L \subtype F_{i} \wedge F_{i} \subtype L \wedge
       \Const \; \Self \times F_{1} \times ... \times F_{n} \times \Nil{*} \rightarrow S \subtype V_{i} \\
       closeall(\env_{1})[self \mapsto \Self, \overline{id \mapsto F}, \self \mapsto F_{s}],
       \penv[\ret \mapsto S] \vdash s, \env_{3}\\
       V_{i} \mapsto \Const \; \Self \times F_{1} \times ... \times F_{n} \times \Nil{*} \rightarrow S\\
       \env_{4} = openset(\env_{1}[id_{1} \mapsto F_{s}], frv(\mathbf{fun} \; (\overline{id{:}T}){:}S \; s)) \\
       \env_{5} = closeset(\env_{4}, fav(\mathbf{fun} \; (\overline{id{:}T}){:}S \; s))
       \end{array}}
      {\begin{array}{c}
       \env_{1}, \penv \vdash \mathbf{fun} \; id_{1}{:}id_{2} \; (\overline{id{:}F}){:}S \; s, \env_{5}\\
       \end{array}}
\]

\mylabel{T-METHOD6}
\[
\dfrac{\begin{array}{c}
       \env_{1}(id_{1}) = F_{s} \;\;\;
       F_{s} = \{\overline{F{:}V}\}_{open} \;\;\;
       \env_{1}, \penv \vdash id_{2} : L, \env_{2} \;\;\;
       n = |\overline{F{:}V}| \\
       \exists i \in 1..n \; L \subtype F_{i} \wedge F_{i} \subtype L \wedge
       \Const \; \Self \times F_{1} \times ... \times F_{n} \times \Nil{*} \rightarrow S \subtype V_{i} \\
       closeall(\env_{1})[self \mapsto \Self, \overline{id \mapsto F}, \self \mapsto F_{s}],
       \penv[\ret \mapsto S] \vdash s, \env_{3}\\
       V_{i} \mapsto \Const \; \Self \times F_{1} \times ... \times F_{n} \times \Nil{*} \rightarrow S\\
       \env_{4} = openset(\env_{1}[id_{1} \mapsto F_{s}], frv(\mathbf{fun} \; (\overline{id{:}T}){:}S \; s)) \\
       \env_{5} = closeset(\env_{4}, fav(\mathbf{fun} \; (\overline{id{:}T}){:}S \; s))
       \end{array}}
      {\begin{array}{c}
       \env_{1}, \penv \vdash \mathbf{fun} \; id_{1}{:}id_{2} \; (\overline{id{:}F}){:}S \; s, \env_{5}\\
       \end{array}}
\]

\mylabel{T-METHOD7}
\[
\dfrac{\begin{array}{c}
       \env_{1}(id_{1}) = F_{s} \;\;\;
       F_{s} = \{\overline{F{:}V}\}_{unique} \;\;\;
       \env_{1}, \penv \vdash id_{2} : L, \env_{2} \;\;\;
       n = |\overline{F{:}V}| \\
       \exists i \in 1..n \; L \subtype F_{i} \wedge F_{i} \subtype L \wedge
       \Const \; \Self \times F_{1} \times ... \times F_{n} \times F{*} \rightarrow S \subtype V_{i} \\
       closeall(\env_{1})[self \mapsto \Self, \overline{id \mapsto F}, {...} \mapsto F, \self \mapsto F_{s}],
       \penv[\ret \mapsto S] \vdash s, \env_{3}\\
       V_{i} \mapsto \Const \; \Self \times F_{1} \times ... \times F_{n} \times F{*} \rightarrow S\\
       \env_{4} = openset(\env_{1}[id_{1} \mapsto F_{s}], frv(\mathbf{fun} \; (\overline{id{:}T}){:}S \; s)) \\
       \env_{5} = closeset(\env_{4}, fav(\mathbf{fun} \; (\overline{id{:}T}){:}S \; s))
       \end{array}}
      {\begin{array}{c}
       \env_{1}, \penv \vdash \mathbf{fun} \; id_{1}{:}id_{2} \; (\overline{id{:}F},{...}{:}F){:}S \; s, \env_{5}\\
       \end{array}}
\]

\mylabel{T-METHOD8}
\[
\dfrac{\begin{array}{c}
       \env_{1}(id_{1}) = F_{s} \;\;\;
       F_{s} = \{\overline{F{:}V}\}_{open} \;\;\;
       \env_{1}, \penv \vdash id_{2} : L, \env_{2} \;\;\;
       n = |\overline{F{:}V}| \\
       \exists i \in 1..n \; L \subtype F_{i} \wedge F_{i} \subtype L \wedge
       \Const \; \Self \times F_{1} \times ... \times F_{n} \times F{*} \rightarrow S \subtype V_{i} \\
       closeall(\env_{1})[self \mapsto \Self, \overline{id \mapsto F}, {...} \mapsto F, \self \mapsto F_{s}],
       \penv[\ret \mapsto S] \vdash s, \env_{3}\\
       V_{i} \mapsto \Const \; \Self \times F_{1} \times ... \times F_{n} \times F{*} \rightarrow S\\
       \env_{4} = openset(\env_{1}[id_{1} \mapsto F_{s}], frv(\mathbf{fun} \; (\overline{id{:}T}){:}S \; s)) \\
       \env_{5} = closeset(\env_{4}, fav(\mathbf{fun} \; (\overline{id{:}T}){:}S \; s))
       \end{array}}
      {\begin{array}{c}
       \env_{1}, \penv \vdash \mathbf{fun} \; id_{1}{:}id_{2} \; (\overline{id{:}F},{...}{:}F){:}S \; s, \env_{5}\\
       \end{array}}
\]

\mylabel{T-WHILE1}
\[
\dfrac{\begin{array}{c}
       \env_{1}, \penv \vdash e:F, \env_{2} \;\;\;
       closeall(\env_{2}), \penv \vdash s, \env_{3}\\
       \env_{4} = closeset(\env_{2}, fav(s)) \\
       \env_{5} = openset(\env_{4},frv(s))
       \end{array}}
      {\env_{1}, \penv \vdash \mathbf{while} \; e \; \mathbf{do} \; s,\env_{5}}
\]

\mylabel{T-WHILE2}
\[
\dfrac{\begin{array}{c}
       \env_{1}(id) = F\\
       closeall(\env_{1}[id \mapsto \phi(F, filter(F, \Nil))]), \penv \vdash s, \env_{2}\\
       \env_{3} = openset(\env_{1}[id \mapsto F], frv(s))\\
       \env_{4} = closeset(\env_{3}, fav(s))
       \end{array}}
      {\env_{1}, \penv \vdash \mathbf{while} \; id \; \mathbf{do} \; s,\env_{4}}
\]

\mylabel{T-IF1}
\[
\dfrac{\begin{array}{c}
       \env_{1}, \penv \vdash e:T, \env_{2} \;\;\;
       \env_{2}, \penv \vdash s_{1}:\env_{3} \;\;\;
       \env_{2}, \penv \vdash s_{2}:\env_{4} \;\;\;
       \env_{5} = join(\env_{3}, \env_{4})
       \end{array}}
      {\env_{1} \vdash \mathbf{if} \; e \; \mathbf{then} \; s_{1} \; \mathbf{else} \; s_{2}, \env_{5}}
\]

\mylabel{T-IF2}
\[
\dfrac{\begin{array}{c}
       \env_{1}(id) = F \;\;\;
       F_{t} = fot(F, \Nil) \;\;\;
       F_{e} = fit(F, \Nil) \\
       \env_{1}[id \mapsto \phi(F,F_{t})], \penv \vdash s_{1}, \env_{2}\\
       \env_{1}[id \mapsto \phi(F,F_{e})], \penv \vdash s_{2}, \env_{3}\\
       \env_{4} = join(\env_{2}, \env_{3})
      \end{array}}
      {\env_{1}, \penv \vdash \mathbf{if} \; id \; \mathbf{then} \; s_{1} \; \mathbf{else} \; s_{2}, \env_{4}[id \mapsto F]}
\]

\mylabel{T-IF3}
\[
\dfrac{\begin{array}{c}
       \env_{1}(id) = F \;\;\;
       F_{t} = fot(F, \Nil) \;\;\;
       F_{e} = fit(F, \Nil) \\
       F_{e} = \Void \;\;\;
       \env_{1}[id \mapsto \phi(F,F_{t})], \penv \vdash s_{1}, \env_{2}
      \end{array}}
      {\env_{1}, \penv \vdash \mathbf{if} \; id \; \mathbf{then} \; s_{1} \; \mathbf{else} \; s_{2}, \env_{2}[id \mapsto F]}
\]

\mylabel{T-IF4}
\[
\dfrac{\begin{array}{c}
       \env_{1}(id) = F \;\;\;
       F_{t} = fot(F, \Nil) \;\;\;
       F_{e} = fit(F, \Nil) \\
       F_{t} = \Void \;\;\;
       \env_{1}[id \mapsto \phi(F,F_{e})], \penv \vdash s_{2}, \env_{2}
      \end{array}}
      {\env_{1}, \penv \vdash \mathbf{if} \; id \; \mathbf{then} \; s_{1} \; \mathbf{else} \; s_{2}, \env_{2}[id \mapsto F]}
\]

\mylabel{T-IF5}
\[
\dfrac{\begin{array}{c}
       \env_{1}(id) = \pi_{i}^{x} \\
       S_{t} = fopt(\penv(x), \Nil, i) \;\;\;
       S_{e} = fipt(\penv(x), \Nil, i) \\
       \env_{1}, \penv[x \mapsto S_{t}] \vdash s_{1}, \env_{2}\\
       \env_{1}, \penv[x \mapsto S_{e}] \vdash s_{2}, \env_{3}\\
       \env_{4} = join(\env_{2}, \env_{3})
      \end{array}}
      {\env_{1}, \penv \vdash \mathbf{if} \; id \; \mathbf{then} \; s_{1} \; \mathbf{else} \; s_{2}, \env_{4}}
\]

\mylabel{T-IF6}
\[
\dfrac{\begin{array}{c}
       \env_{1}(id) = \pi_{i}^{x} \\
       S_{t} = fopt(\penv(x), \Nil, i) \\
       fit(proj(\penv(x), i), \Nil) = \Void \\
       \env_{1}, \penv[x \mapsto S_{t}] \vdash s_{1}, \env_{2}
      \end{array}}
      {\env_{1}, \penv \vdash \mathbf{if} \; id \; \mathbf{then} \; s_{1} \; \mathbf{else} \; s_{2}, \env_{2}}
\]

\mylabel{T-IF7}
\[
\dfrac{\begin{array}{c}
       \env_{1}(id) = \pi_{i}^{x} \\
       S_{e} = fipt(\penv(x), \Nil, i) \\
       fot(proj(\penv(x), i), \Nil) = \Void \\
       \env_{1}, \penv[x \mapsto S_{e}] \vdash s_{2}, \env_{2}
      \end{array}}
      {\env_{1}, \penv \vdash \mathbf{if} \; id \; \mathbf{then} \; s_{1} \; \mathbf{else} \; s_{2}, \env_{2}}
\]

\mylabel{T-IF8}
\[
\dfrac{\begin{array}{c}
       \env_{1}(id) = \phi(F_{1},F_{2}) \;\;\;
       F_{t} = fit(F_{2}, \String) \;\;\;
       F_{e} = fot(F_{2}, \String) \\
       \env_{1}[id \mapsto \phi(F_{1},F_{t})], \penv \vdash s_{1}, \env_{2}\\
       \env_{1}[id \mapsto \phi(F_{1},F_{e})], \penv \vdash s_{2}, \env_{3}\\
       \env_{4} = join(\env_{2}, \env_{3})
      \end{array}}
      {\env_{1}, \penv \vdash \mathbf{if} \; type(id) == ``string" \; \mathbf{then} \; s_{1} \; \mathbf{else} \; s_{2}, \env_{4}[id \mapsto F_{1}]}
\]

\mylabel{T-IF9}
\[
\dfrac{\begin{array}{c}
       \env_{1}(id) = \phi(F_{1},F_{2}) \;\;\;
       F_{t} = fit(F_{2}, \String) \;\;\;
       F_{e} = fot(F_{2}, \String) \\
       F_{t} = \Void \;\;\;
       \env_{1}[id \mapsto \phi(F_{1},F_{e})], \penv \vdash s_{2}, \env_{2}
      \end{array}}
      {\env_{1}, \penv \vdash \mathbf{if} \; type(id) == ``string" \; \mathbf{then} \; s_{1} \; \mathbf{else} \; s_{2}, \env_{2}[id \mapsto F_{1}]}
\]

\mylabel{T-IF10}
\[
\dfrac{\begin{array}{c}
       \env_{1}(id) = \phi(F_{1},F_{2}) \;\;\;
       F_{t} = fit(F_{2}, \String) \;\;\;
       F_{e} = fot(F_{2}, \String) \\
       F_{e} = \Void \;\;\;
       \env_{1}[id \mapsto \phi(F_{1},F_{t})], \penv \vdash s_{1}, \env_{2}
      \end{array}}
      {\env_{1}, \penv \vdash \mathbf{if} \; type(id) == ``string" \; \mathbf{then} \; s_{1} \; \mathbf{else} \; s_{2}, \env_{2}[id \mapsto F_{1}]}
\]

\mylabel{T-LOCAL1}
\[
\dfrac{\begin{array}{c}
       \env_{1}, \penv \vdash el:S, \env_{2} \\
       S \lesssim F_{1} \times ... \times F_{n} \times \Value{*} \;\;\;
       n = |\;\overline{id{:}F}\;| \\
       \env_{2}[\overline{id \mapsto F}], \penv \vdash s, \env_{3}
       \end{array}}
      {\env_{1}, \penv \vdash \mathbf{local} \; \overline{id{:}F} = el \; \mathbf{in} \; s, (\env_{3} - \{\overline{id}\})[\overline{id \mapsto \env_{2}(id)]}}
\]

\mylabel{T-LOCAL2}
\[
\dfrac{\begin{array}{c}
       \env_{1}, \penv \vdash el:E, \env_{2}, (x,S) \\
       \env_{3} = \env_{2}[id_{1} \mapsto infer(E,1), ..., id_{n} \mapsto infer(E,n)] \\
       \env_{3}, \penv[x \mapsto S] \vdash s, \env_{4} \;\;\;
       n = |\;\overline{id}\;|
       \end{array}}
      {\env_{1}, \penv \vdash \mathbf{local} \; \overline{id} = el \; \mathbf{in} \; s, (\env_{4} - \{\overline{id}\})[\overline{id \mapsto \env_{2}(id)]}}
\]

\mylabel{T-LOCALREC}
\[
\dfrac{\env_{1}[id \mapsto F], \penv \vdash e:F_{1}, \env_{2} \;\;\;
       F_{1} \lesssim F \;\;\;
       \env_{2}, \penv \vdash s, \env_{3}}
      {\env_{1}, \penv \vdash \mathbf{rec} \; id{:}F = e \; \mathbf{in} \; s, (\env_{3} - \{id\})[\overline{id \mapsto \env_{2}(id)]}}
\]

\mylabel{T-RETURN}
\[
\dfrac{\env_{1} \vdash el:S_{1}, \env_{2} \;\;\;
       \penv(\ret) = S_{2} \;\;\;
       S_{1} \lesssim S_{2}}
      {\env_{1} \vdash \mathbf{return} \; el, \env_{2}}
\]

\mylabel{T-STMAPPLY1}
\[
\dfrac{\env_{1}, \penv \vdash e(el):S, \env_{2}}
      {\env_{1}, \penv \vdash \lfloor e(el) \rfloor_{0},\env_{2}}
\]

\mylabel{T-STMINVOKE1}
\[
\dfrac{\env_{1}, \penv \vdash e{:}n(el):S, \env_{2}}
      {\env_{1}, \penv \vdash \lfloor e{:}n(el) \rfloor_{0}, \env_{2}}
\]

\mylabel{T-NIL}
\[
\env_{1}, \penv \vdash \mathbf{nil}:\Nil, \env_{1}
\]

\mylabel{T-FALSE}
\[
\env_{1}, \penv \vdash \mathbf{false}:\False, \env_{1}
\]

\mylabel{T-TRUE}
\[
\env_{1}, \penv \vdash \mathbf{true}:\True, \env_{1}
\]

\mylabel{T-INT}
\[
\env_{1}, \penv \vdash {\it int}:{\it int}, \env_{1}
\]

\mylabel{T-FLOAT}
\[
\env_{1}, \penv \vdash {\it float}:{\it float}, \env_{1}
\]

\mylabel{T-STR}
\[
\env_{1}, \penv \vdash {\it string}:{\it string}, \env_{1}
\]

\mylabel{T-IDREAD1}
\[
\dfrac{\env_{1}(id) = F}
      {\env_{1}, \penv \vdash id:close(F), \env_{1}[id \mapsto open(F)]}
\]

\mylabel{T-IDREAD2}
\[
\dfrac{\env_{1}(id) = F}
      {\env_{1}, \penv \vdash id:fix(F), \env_{1}[id \mapsto fix(F)]}
\]

\mylabel{T-IDREAD3}
\[
\dfrac{\env_{1}(id) = \phi(F_{1},F_{2})}
      {\env_{1}, \penv \vdash id:F_{2}, \env_{1}}
\]

\mylabel{T-IDREAD4}
\[
\dfrac{\env_{1}(id) = \pi_{i}^{x}}
      {\env_{1}, \penv \vdash id:proj(\penv(x), i), \env_{1}}
\]

\mylabel{T-INDEXREAD1}
\[
\dfrac{\begin{array}{c}
       \env_{1}(id) = \{\overline{F{:}V}\} \;\;\;
       \env_{1}, \penv \vdash e_{2}:F, \env_{2} \;\;\;
       \exists i \in 1{..}n \; F \lesssim F_{i} \;\;\;
       n = |\overline{F{:}V}|
       \end{array}}
      {\env_{1}, \penv \vdash id[e_{2}]:rconst(V_{i}), \env_{2}}
\]

\mylabel{T-INDEXREAD2}
\[
\dfrac{\begin{array}{c}
       \env_{1}, \penv \vdash e_{1}:\{\overline{F{:}V}\}, \env_{2} \;\;\;
       \env_{2}, \penv \vdash e_{2}:F, \env_{3} \;\;\;
       \exists i \in 1{..}n \; F \lesssim F_{i} \;\;\;
       n = |\overline{F{:}V}|
       \end{array}}
      {\env_{1}, \penv \vdash e_{1}[e_{2}]:rconst(V_{i}), \env_{3}}
\]

\mylabel{T-INDEXREAD3}
\[
\dfrac{\env_{1}, \penv \vdash e_{1}:\Any, \env_{2} \;\;\;
       \env_{2}, \penv \vdash e_{2}:F, \env_{3}}
      {\env_{1}, \penv \vdash e_{1}[e_{2}]:\Any, \env_{3}}
\]

\mylabel{T-COERCE1}
\[
\dfrac{\env_{1}(id) \subtype F \;\;\; tag(F,closed)}
      {\env_{1}, \penv \vdash {<}F{>} \; id:F, \env_{1}[id \mapsto reopen(F)]}
\]

\mylabel{T-COERCE2}
\[
\dfrac{\env_{1}(id) \subtype F \;\;\; tag(F,fixed)}
      {\env_{1}, \penv \vdash {<}F{>} \; id:F, \env_{1}[id \mapsto F]}
\]

\mylabel{T-FUNCTION1}
\[
\dfrac{\begin{array}{c}
       closeall(\env_{1})[\overline{id \mapsto F}], \penv[\ret \mapsto S] \vdash s, \env_{2} \\
       \env_{3} = openset(\env_{1}, frv(\mathbf{fun} \; (\overline{id{:}F}){:}S \; s)) \\
       \env_{4} = closeset(\env_{3}, fav(\mathbf{fun} \; (\overline{id{:}F}){:}S \; s))
       \end{array}}
      {\env_{1}, \penv \vdash \mathbf{fun} \; (\overline{id{:}F}){:}S \; s:F_{1} \times ... \times F_{n} \times \Nil{*} \rightarrow S, \env_{4}}
\]

\mylabel{T-FUNCTION2}
\[
\dfrac{\begin{array}{c}
       closeall(\env_{1})[\overline{id \mapsto F}, {...} \mapsto F], \penv[\ret \mapsto S] \vdash s, \env_{2} \\
       \env_{3} = openset(\env_{1}, frv(\mathbf{fun} \; (\overline{id{:}F}){:}S \; s)) \\
       \env_{4} = closeset(\env_{3}, fav(\mathbf{fun} \; (\overline{id{:}F}){:}S \; s))
       \end{array}}
      {\env_{1}, \penv \vdash \mathbf{fun} \; (\overline{id{:}F},{...}{:}F){:}S \; s:F_{1} \times ... \times F_{n} \times F{*} \rightarrow S, \env_{4}}
\]

\mylabel{T-CONSTRUCTOR1}
\[
\dfrac{\begin{array}{c}
       \env_{1}, \penv \vdash ([e_{1}] = e_{2})_{i}:(F_{i},V_{i}), \env_{i+1} \;\;\;
       T = \{F_{1}{:}V_{1}, ..., F_{n}{:}V_{n}\}_{unique} \\
       wf(T) \;\;\;
       n = |\;\overline{[e_{1}] = e_{2}}\;| \;\;\;
       \env_{f} = merge(\env_{1}, ..., \env_{n+1})
       \end{array}}
      {\env_{1}, \penv \vdash \{\;\overline{[e_{1}] = e_{2}}\;\}:T, \env_{f}}
\]

\mylabel{T-CONSTRUCTOR2}
\[
\dfrac{\begin{array}{c}
       \env_{1}, \penv \vdash ([e_{1}] = e_{2})_{i}:(F_{i},V_{i}), \env_{i+1} \\
       \env_{1}, \penv \vdash me : F_{n+1} \times ... \times F_{n+m} \times F_{n+m+1}{*}, \env_{n+2}\\
       T = \{F_{1}{:}V_{1}, ..., F_{n}{:}V_{n}, 1{:}F_{n+1}, ..., m{:}F_{n+m}, \Integer{:}F_{n+m+1} \cup \Nil\}_{unique}\\
       wf(T) \;\;\;
       n = |\;\overline{[e_{1}] = e_{2}}\;| \;\;\;
       \env_{f} = merge(\env_{1}, ..., \env_{n+2})
       \end{array}}
      {\env_{1}, \penv \vdash \{\;\overline{[e_{1}] = e_{2}}\;\}:T, \env_{f}}
\]

\mylabel{T-FIELD}\\
\[
\dfrac{\env_{1}, \penv \vdash e_{2}:V, \env_{2} \;\;\;
       \env_{2}, \penv \vdash e_{1}:F, \env_{3}}
      {\env_{1}, \penv \vdash [e_{1}] = e_{2}: (F,vt(F,V)), \env_{3}}
\]

\mylabel{T-ARITH1}
\[
\dfrac{\env_{1}, \penv \vdash e_{1}:F_{1}, \env_{2} \;\;\;
       \env_{2}, \penv \vdash e_{2}:F_{2}, \env_{3} \;\;\;
       F_{1} \subtype \Integer \;\;\;
       F_{2} \subtype \Integer}
      {\env_{1}, \penv \vdash e_{1} + e_{2}:\Integer, \env_{3}}
\]

\mylabel{T-ARITH2}
\[
\dfrac{\env_{1}, \penv \vdash e_{1}:F_{1}, \env_{2} \;\;\;
       \env_{2}, \penv \vdash e_{2}:F_{2}, \env_{3} \;\;\;
       F_{1} \subtype \Integer \;\;\;
       F_{2} \subtype \Number}
      {\env_{1}, \penv \vdash e_{1} + e_{2}:\Number, \env_{3}}
\]

\mylabel{T-ARITH3}
\[
\dfrac{\env_{1}, \penv \vdash e_{1}:F_{1}, \env_{2} \;\;\;
       \env_{2}, \penv \vdash e_{2}:F_{2}, \env_{3} \;\;\;
       F_{1} \subtype \Number \;\;\;
       F_{2} \subtype \Integer}
      {\env_{1}, \penv \vdash e_{1} + e_{2}:\Number, \env_{3}}
\]

\mylabel{T-ARITH4}
\[
\dfrac{\env_{1}, \penv \vdash e_{1}:F_{1}, \env_{2} \;\;\;
       \env_{2}, \penv \vdash e_{2}:F_{2}, \env_{3} \;\;\;
       F_{1} \subtype \Number \;\;\;
       F_{2} \subtype \Number}
      {\env_{1}, \penv \vdash e_{1} + e_{2}:\Number, \env_{3}}
\]

\mylabel{T-ARITH5}
\[
\dfrac{\env_{1}, \penv \vdash e_{1}:\Any, \env_{2} \;\;\;
       \env_{2}, \penv \vdash e_{2}:F, \env_{3}}
      {\env_{1}, \penv \vdash e_{1} + e_{2}:\Any, \env_{3}}
\]

\mylabel{T-ARITH6}
\[
\dfrac{\env_{1}, \penv \vdash e_{1}:F, \env_{2} \;\;\;
       \env_{2}, \penv \vdash e_{2}:\Any, \env_{3}}
      {\env_{1}, \penv \vdash e_{1} + e_{2}:\Any, \env_{3}}
\]

\mylabel{T-CONCAT1}
\[
\dfrac{\env_{1}, \penv \vdash e_{1}:F_{1}, \env_{2} \;\;\;
       \env_{2}, \penv \vdash e_{2}:F_{2}, \env_{3} \;\;\;
       F_{1} \subtype \String \;\;\;
       F_{2} \subtype \String}
      {\env_{1}, \penv \vdash e_{1} \; {..} \; e_{2}:\String, \env_{3}}
\]

\mylabel{T-CONCAT2}
\[
\dfrac{\env_{1}, \penv \vdash e_{1}:\Any, \env_{2} \;\;\;
       \env_{2}, \penv \vdash e_{2}:F, \env_{3}}
      {\env_{1}, \penv \vdash e_{1} \; {..} \; e_{2}:\Any, \env_{3}}
\]

\mylabel{T-CONCAT3}
\[
\dfrac{\env_{1}, \penv \vdash e_{1}:F, \env_{2} \;\;\;
       \env_{2}, \penv \vdash e_{2}:\Any, \env_{3}}
      {\env_{1}, \penv \vdash e_{1} \; {..} \; e_{2}:\Any, \env_{3}}
\]

\mylabel{T-EQUAL}
\[
\dfrac{\env_{1}, \penv \vdash e_{1}:F_{1}, \env_{2} \;\;\;
       \env_{2}, \penv \vdash e_{2}:F_{2}, \env_{3}}
      {\env_{1}, \penv \vdash e_{1} == e_{2}:\Boolean, \env_{3}}
\]

\mylabel{T-ORDER1}
\[
\dfrac{\env_{1}, \penv \vdash e_{1}:F_{1}, \env_{2} \;\;\;
       \env_{2}, \penv \vdash e_{2}:F_{2}, \env_{3} \;\;\;
       F_{1} \subtype \Number \;\;\;
       F_{2} \subtype \Number}
      {\env, \penv \vdash e_{1} < e_{2}:\Boolean, \env_{3}}
\]

\mylabel{T-ORDER2}
\[
\dfrac{\env_{1}, \penv \vdash e_{1}:F_{1}, \env_{2} \;\;\;
       \env_{2}, \penv \vdash e_{2}:F_{2}, \env_{3} \;\;\;
       F_{1} \subtype \String \;\;\;
       F_{2} \subtype \String}
      {\env_{1}, \penv \vdash e_{1} < e_{2}:\Boolean}
\]

\mylabel{T-ORDER3}
\[
\dfrac{\env_{1}, \penv \vdash e_{1}:\Any, \env_{2} \;\;\;
       \env_{2}, \penv \vdash e_{2}:F, \env_{3}}
      {\env_{1}, \penv \vdash e_{1} < e_{2}:\Any, \env_{3}}
\]

\mylabel{T-ORDER4}
\[
\dfrac{\env_{1}, \penv \vdash e_{1}:F, \env_{2} \;\;\;
       \env_{2}, \penv \vdash e_{2}:\Any, \env_{3}}
      {\env_{1}, \penv \vdash e_{1} < e_{2}:\Any, \env_{3}}
\]

\mylabel{T-BITWISE1}
\[
\dfrac{\env_{1}, \penv \vdash e_{1}:F_{1}, \env_{2} \;\;\;
       \env_{2}, \penv \vdash e_{2}:F_{2}, \env_{3} \;\;\;
       F_{1} \subtype \Integer \;\;\;
       F_{2} \subtype \Integer}
      {\env_{1}, \penv \vdash e_{1} \;\&\; e_{2}:\Integer, \env_{3}}
\]

\mylabel{T-BITWISE2}
\[
\dfrac{\env_{1}, \penv \vdash e_{1}:\Any, \env_{2} \;\;\;
       \env_{2}, \penv \vdash e_{2}:F, \env_{3}}
      {\env_{1}, \penv \vdash e_{1} \;\&\; e_{2}:\Any, \env_{3}}
\]

\mylabel{T-BITWISE3}
\[
\dfrac{\env_{1}, \penv \vdash e_{1}:F, \env_{2} \;\;\;
       \env_{2}, \penv \vdash e_{2}:\Any, \env_{3}}
      {\env_{1}, \penv \vdash e_{1} \;\&\; e_{2}:\Any, \env_{3}}
\]

\mylabel{T-AND1}
\[
\dfrac{\env_{1}, \penv \vdash e_{1}:\Nil, \env_{2}}
      {\env_{1}, \penv \vdash e_{1} \; \mathbf{and} \; e_{2}:\Nil, \env_{2}}
\]

\mylabel{T-AND2}
\[
\dfrac{\env_{1}, \penv \vdash e_{1}:\False, \env_{2}}
      {\env_{1}, \penv \vdash e_{1} \; \mathbf{and} \; e_{2}:\False, \env_{2}}
\]

\mylabel{T-AND3}
\[
\dfrac{\env_{1}, \penv \vdash e_{1}:\Nil \cup \False, \env_{2}}
      {\env_{1}, \penv \vdash e_{1} \; \mathbf{and} \; e_{2}:\Nil \cup \False, \env_{2}}
\]

\mylabel{T-AND4}
\[
\dfrac{\env_{1}, \penv \vdash e_{1}:F_{1}, \env_{2} \;\;\;
       \env_{2}, \penv \vdash e_{2}:F_{2}, \env_{3} \;\;\;
       \Nil \not\lesssim F_{1} \;\;\;
       \False \not\lesssim F_{1}}
      {\env_{1}, \penv \vdash e_{1} \; \mathbf{and} \; e_{2}:F_{2}, \env_{3}}
\]

\mylabel{T-AND5}
\[
\dfrac{\env_{1}, \penv \vdash e_{1}:F_{1}, \env_{2} \;\;\;
       \env_{2}, \penv \vdash e_{2}:F_{2}, \env_{3}}
      {\env_{1}, \penv \vdash e_{1} \; \mathbf{and} \; e_{2}:F_{1} \cup F_{2}, \env_{3}}
\]

\mylabel{T-OR1}
\[
\dfrac{\env_{1}, \penv \vdash e_{1}:F, \env_{2} \;\;\;
       \Nil \not\lesssim F \;\;\;
       \False \not\lesssim F}
      {\env_{1}, \penv \vdash e_{1} \; \mathbf{or} \; e_{2}:F, \env_{2}}
\]

\mylabel{T-OR2}
\[
\dfrac{\env_{1}, \penv \vdash e_{1}:\Nil, \env_{2} \;\;\;
       \env_{2}, \penv \vdash e_{2}:F, \env_{3}}
      {\env_{1}, \penv \vdash e_{1} \; \mathbf{or} \; e_{2}:F, \env_{3}}
\]

\mylabel{T-OR3}
\[
\dfrac{\env_{1}, \penv \vdash e_{1}:\False, \env_{2} \;\;\;
       \env_{2}, \penv \vdash e_{2}:F, \env_{3}}
      {\env_{1}, \penv \vdash e_{1} \; \mathbf{or} \; e_{2}:F, \env_{3}}
\]

\mylabel{T-OR4}
\[
\dfrac{\env_{1}, \penv \vdash e_{1}:\Nil \cup \False, \env_{2} \;\;\;
       \env_{2}, \penv \vdash e_{2}:F, \env_{3}}
      {\env_{1}, \penv \vdash e_{1} \; \mathbf{or} \; e_{2}:F, \env_{3}}
\]

\mylabel{T-OR5}
\[
\dfrac{\env_{1}, \penv \vdash e_{1}:F_{1}, \env_{2} \;\;\;
       \env_{2}, \penv \vdash e_{2}:F_{2}, \env_{3}}
      {\env_{1}, \penv \vdash e_{1} \; \mathbf{or} \; e_{2}:filter(filter(F_{1}, \Nil), \False) \cup F_{2}, \env_{3}}
\]

\mylabel{T-NOT1}
\[
\dfrac{\env_{1}, \penv \vdash e:\Nil, \env_{2}}
      {\env_{1}, \penv \vdash \mathbf{not} \; e:\True, \env_{2}}
\]

\mylabel{T-NOT2}
\[
\dfrac{\env_{1}, \penv \vdash e:\False, \env_{2}}
      {\env_{1}, \penv \vdash \mathbf{not} \; e:\True, \env_{2}}
\]

\mylabel{T-NOT3}
\[
\dfrac{\env_{1}, \penv \vdash e:\Nil \cup \False, \env_{2}}
      {\env_{1}, \penv \vdash \mathbf{not} \; e:\True, \env_{2}}
\]

\mylabel{T-NOT4}
\[
\dfrac{\env_{1}, \penv \vdash e:F \;\;\;
       \Nil \not\lesssim F \;\;\;
       \False \not\lesssim F}
      {\env_{1}, \penv \vdash \mathbf{not} \; e:\False, \env_{2}}
\]

\mylabel{T-NOT5}
\[
\dfrac{\env_{1}, \penv \vdash e:F, \env_{2}}
      {\env_{1}, \penv \vdash \mathbf{not} \; e:\Boolean, \env_{2}}
\]

\mylabel{T-LEN1}
\[
\dfrac{\env_{1}, \penv \vdash e:F, \env_{2} \;\;\;
       F \subtype \String}
      {\env_{1}, \penv \vdash \# \; e:\Integer, \env_{2}}
\]

\mylabel{T-LEN2}
\[
\dfrac{\env_{1}, \penv \vdash e:F, \env_{2} \;\;\;
       F \subtype \{\}_{closed}}
      {\env_{1}, \penv \vdash \# \; e:\Integer, \env_{2}}
\]

\mylabel{T-LEN3}
\[
\dfrac{\env_{1}, \penv \vdash e:\Any, \env_{2}}
      {\env_{1}, \penv \vdash \# \; e:\Any, \env_{2}}
\]

\mylabel{T-EXPAPPLY1}
\[
\dfrac{\env_{1}, \penv \vdash e(el):S, \env_{2}}
      {\env_{1}, \penv \vdash \lfloor e(el) \rfloor_{1}:proj(S,1), \env_{2}}
\]

\mylabel{T-EXPINVOKE1}
\[
\dfrac{\env_{1}, \penv \vdash e{:}n(el):S, \env_{2}}
      {\env_{1}, \penv \vdash \lfloor e{:}n(el) \rfloor_{1}:proj(S,1), \env_{2}}
\]

\mylabel{T-EXPDOTS}
\[
\dfrac{\env_{1}, \penv \vdash {...}:F{*}, \env_{2}}
      {\env_{1}, \penv \vdash \lfloor {...} \rfloor_{1}:F \cup \Nil, \env_{2}}
\]

\mylabel{T-IDWRITE1}
\[
\dfrac{\env_{1}(id) = F}
      {\env_{1}, \penv \vdash id_{l}:F, \env_{1}}
\]

\mylabel{T-IDWRITE2}
\[
\dfrac{\env_{1}(id) = \phi(F_{1},F_{2})}
      {\env_{1}, \penv \vdash id_{l}:F_{1}, \env_{1}[id \mapsto F_{1}]}
\]

\mylabel{T-INDEXWRITE1}
\[
\dfrac{\begin{array}{c}
       \env_{1}, \penv \vdash e_{1}:\{\overline{F{:}V}\}, \env_{2} \\
       \env_{2}, \penv \vdash e_{2}:F, \env_{3} \;\;\;
       \exists i \in 1{..}n \; F \lesssim F_{i} \wedge \neg const(V_{i}) \;\;\;
       n = |\overline{F{:}V}|
       \end{array}}
      {\env_{1}, \penv \vdash e_{1}[e_{2}]_{l}:V_{i}, \env_{3}}
\]

\mylabel{T-INDEXWRITE2}
\[
\dfrac{\env_{1}, \penv \vdash e_{1}:\Any, \env_{2} \;\;\;
       \env_{2}, \penv \vdash e_{2}:F, \env_{3}}
      {\env_{1}, \penv \vdash e_{1}[e_{2}]_{l}:\Any, \env_{3}}
\]

\mylabel{T-REFINE1}
\[
\dfrac{\begin{array}{c}
       \env_{1}(id) = \{\overline{F{:}V}\}_{unique}\\
       \env_{1}, \penv \vdash e:F_{new}, \env_{2} \;\;\;
       \nexists i \in 1..n \; F_{new} \lesssim F_{i} \;\;\;
       V_{new} = vt(F_{new},V) \;\;\; n = |\overline{F{:}V}|
       \end{array}}
      {\env_{1}, \penv \vdash id[e] {<}V{>}:V_{new}, \env_{2}[id \mapsto \{\overline{F{:}V}, F_{new}{:}V_{new}\}_{unique}]}
\]

\mylabel{T-REFINE2}
\[
\dfrac{\begin{array}{c}
       \env_{1}(id) = \{\overline{F{:}V}\}_{open}\\
       \env_{1}, \penv \vdash e:F_{new}, \env_{2} \;\;\;
       \nexists i \in 1..n \; F_{new} \lesssim F_{i} \;\;\;
       V_{new} = vt(F_{new},V) \;\;\; n = |\overline{F{:}V}|
       \end{array}}
      {\env_{1}, \penv \vdash id[e] {<}V{>}:V_{new}, \env_{2}[id \mapsto \{\overline{F{:}V}, F_{new}{:}V_{new}\}_{open}]}
\]

\mylabel{T-LHSLIST}
\[
\dfrac{\env_{1}, \penv \vdash l_{i}:F_{i}, \env_{i+1} \;\;\;
       \env_{f} = merge(\env_{1}, ..., \env_{n+1}) \;\;\;
       n = |\;\overline{l}\;|}
      {\env_{1}, \penv \vdash \overline{l}:F_{1} \times ... \times F_{n} \times \Value{*}, \env_{f}}
\]

\mylabel{T-EXPLIST1}
\[
\dfrac{\env_{1}, \penv \vdash e_{i}:F_{i}, \env_{i+1} \;\;\;
       \env_{f} = merge(\env_{1}, ..., \env_{n+1}) \;\;\;
       n = |\;\overline{e}\;|}
      {\env_{1}, \penv \vdash \overline{e}:F_{1} \times ... \times F_{n} \times \Nil{*}, \env_{f}}
\]

\mylabel{T-EXPLIST2}
\[
\dfrac{\begin{array}{c}
       \env_{1}, \penv \vdash e_{i}:F_{i}, \env_{i+1} \;\;\;
       \env_{1}, \penv \vdash me:F_{n+1} \times ... \times F_{n+m} \times F_{n+m+1}{*}, \env_{n+2}\\
       \env_{f} = merge(\env_{1}, ..., \env_{n+2}) \;\;\;
       n = |\;\overline{e}\;|
       \end{array}}
      {\env_{1}, \penv \vdash \overline{e},me:F_{1} \times ... \times F_{n+m} \times F_{n+m+1}{*}, \env_{f}}
\]

\mylabel{T-EXPLIST3}
\[
\dfrac{\begin{array}{c}
       \env_{1}, \penv \vdash e_{i}:F_{i}, \env_{i+1} \;\;\;
       \env_{1}, \penv \vdash me:S, \env_{n+2}\\
       S = F_{n+1} \times ... \times F_{n+m} \times \Nil{*} \sqcup F_{n+1}' \times ... \times F_{n+m}' \times \Nil{*} \\
       \env_{f} = merge(\env_{1}, ..., \env_{n+2}) \;\;\;
       n = |\;\overline{e}\;|
       \end{array}}
      {\env_{1}, \penv \vdash \overline{e},me:F_{1} \times ... \times F_{n} \times \pi_{1}^{x} \times ... \times \pi_{m}^{x} \times \Nil{*}, \env_{f}, (x,S)}
\]

\mylabel{T-APPLY1}
\[
\dfrac{\env_{1}, \penv \vdash e:S_{1} \rightarrow S_{2}, \env_{2} \;\;\;
       \env_{2}, \penv \vdash el:S_{3}, \env_{3} \;\;\;
       S_{3} \lesssim S_{1}}
      {\env_{1}, \penv \vdash e(el):S_{2}, \env_{3}}
\]

\mylabel{T-APPLY2}
\[
\dfrac{\env_{1}, \penv \vdash e:\Any, \env_{2} \;\;\;
       \env_{2}, \penv \vdash el:S, \env_{3}}
      {\env_{1}, \penv \vdash e_{1}(el):\Any{*}, \env_{3}}
\]

\mylabel{T-INVOKE1}
\[
\dfrac{\begin{array}{c}
       \env_{1}, \penv \vdash e:F, \env_{2}\\
       \env_{2}[\sigma \mapsto F], \penv \vdash e[id]:\Const \; S_{1} \rightarrow S_{2}, \env_{3}\\
       \env_{3}[\sigma \mapsto F], \penv \vdash el:S_{3}, \env_{4} \;\;\;
       F \times S_{3} \lesssim [\Self \mapsto F]S_{1}
       \end{array}}
      {\env_{1}, \penv \vdash e{:}id(el):[\Self \mapsto F]S_{2}, \env_{4}}
\]

\mylabel{T-INVOKE2}
\[
\dfrac{\env_{1}, \penv \vdash e:\Any, \env_{2} \;\;\;
       \env_{2}, \penv \vdash el:S, \env_{3}}
      {\env_{1}, \penv \vdash e{:}id(el):\Any{*}, \env_{3}}
\]

\mylabel{T-DOTS}
\[
\dfrac{\env_{1}({...}) = F}
      {\env_{1}, \penv \vdash {...}:F{*}, \env_{1}}
\]

\mylabel{T-SELF}
\[
\dfrac{\env_{1}, \penv \vdash e:\Self, \env_{2} \;\;\;
       \env_{2}(\self) = F}
      {\env_{1}, \penv \vdash e:F, \env_{2}}
\]

\mylabel{T-SETMETATABLE1}
\[
\dfrac{\env_{1}(id) = \Self}
      {\env_{1}, \penv \vdash setmetatable(\{\}, \{[``\string_\string_index"] = id\}):\Self, \env_{1}}
\]

\mylabel{T-SETMETATABLE2}
\[
\dfrac{\env_{1}(id) = \{F_{1}{:}V_{1}, ..., F_{n}{:}V_{n}\}_{fixed}}
      {\env_{1}, \penv \vdash setmetatable(\{\}, \{[``\string_\string_index"] = id\}):\{F_{1}{:}V_{1}, ..., F_{n}{:}V_{n}\}_{open}, \env_{1}}
\]

\mylabel{T-SETMETATABLE3}
\[
\dfrac{\env_{1}, \penv \vdash e : T, \env_{2} \;\;\;
       T = \{F_{1}{:}V_{1}, ..., F_{n}{:}V_{n}\}_{closed} \;\;\;
       \env_{1}(id) = \Self \;\;\; \env_{1}(\sigma) \subtype T}
      {\env_{1}, \penv \vdash setmetatable(e, \{[``\string_\string_index"] = id\}):\Self, \env_{2}[\sigma \mapsto T]}
\]

\mylabel{T-UNFOLD}
\[
\dfrac{\env_{1}, \penv \vdash e:\mu x.F, \env_{2}}
      {\env_{1}, \penv \vdash e:[x \mapsto \mu x.F]F, \env_{2}}
\]

\mylabel{T-FOLD}
\[
\dfrac{\env_{1}, \penv \vdash e:[x \mapsto \mu x.F]F, \env_{2}}
      {\env_{1}, \penv \vdash e:\mu x.F, \env_{2}}
\]

\mylabel{T-TERNARY}
\[
\dfrac{\env_{1}, \penv \vdash e_{1}:F_{1}, \env_{2} \;\;\;
       \env_{2}, \penv \vdash e_{2}:F_{2}, \env_{3} \;\;\;
       \env_{3}, \penv \vdash e_{3}:F_{2}, \env_{4}}
      {\env_{1}, \penv \vdash e_{1} \; \mathbf{and} \; e_{2} \; \mathbf{or} \; e_{3}:F_{2}, \env_{4}}
\]

\section{Auxiliary functions}

\noindent

\begin{align*}
wf(\{\overline{F:V}\}_{unique|open|fixed|closed}) & = \forall i \; ((\nexists j \; i \not= j \,\wedge\, F_{i} \lesssim F_{j}) \,\wedge\, wf(V_{i}) \,\wedge\\
& \;\;\;\; \lnot tag(V_{i},unique) \,\wedge\, \lnot tag(V_{i},open))\\
wf({\bf const}\; F) & = wf(F) \\
wf(F_1 \cup F_2) & = wf(F_1) \,\wedge\, wf(F_2) \\
wf(\mu x.F) & = wf(F) \\
wf(S_1 \rightarrow S_2) & = wf(S_1) \,\wedge\, wf(S_2) \\
wf(S_1 \sqcup S_2) & = wf(S_1) \,\wedge\, wf(S_2)\\
wf(F{*}) & = wf(F) \\
wf(F \times P) & = wf(F) \,\wedge\,wf(P)\\
wf(F) & = \top \;\;\;\mathrm{for\; all\; other\; cases}
\end{align*}

\begin{align*}
tag(F_1 \cup F_2, t) & = tag(F_1, t) \,\vee\, tag(F_2,t) \\
tag(\{\overline{F{:}V}\}_{t}, t) & = \top\\
tag(\{\overline{F{:}V}\}_{t_{1}}, t_{2}) & = \bot\\
tag(F, t) & = \bot
\end{align*}

\begin{align*}
vt(L, V) & = fix(V) \\
vt(F_1, F_2) & = nil(fix(F_2))\\
vt(F_1, {\bf const}\;F_2) & = {\bf const}\;nil(fix(F_2))
\end{align*}

\begin{align*}
nil(T) & = \left\{
\begin{array}{ll}
T & \text{if $\Nil \lesssim T$} \\
T \cup \Nil & \text{otherwise}
\end{array} \right.
\end{align*}

\begin{align*}
fix(F_{1} \cup F_{2}) & = fix(F_{1}) \cup fix(F_{2})\\
fix(\{\overline{F{:}V}\}_{unique|open}) & = \{\overline{F{:}V}\}_{fixed} \\
fix(F) & = F
\end{align*}

\begin{align*}
close(F_{1} \cup F_{2}) & = close(F_{1}) \cup close(F_{2})\\
close(\{\overline{F{:}V}\}_{unique|open}) & = \{\overline{F{:}V}\}_{closed} \\
close(F) & = F
\end{align*}

\begin{align*}
open(F_{1} \cup F_{2}) & = open(F_{1}) \cup open(F_{2})\\
open(\{\overline{F{:}V}\}_{unique}) & = \{\overline{F{:}V}\}_{open} \\
open(F) & = F
\end{align*}

\begin{align*}
reopen(\{\overline{F:V}\}_{closed}) & = \{\overline{F:V}\}_{open}\\
reopen(F) & = F
\end{align*}

\begin{align*}
rconst({\bf const}\;F) & = F\\
rconst(F) & = F
\end{align*}

\begin{align*}
const({\bf const}\;F) & = \top\\
const(F) & = \bot
\end{align*}

\begin{align*}
proj(S_1 \sqcup S_2, i) & = proj(S_1, i) \cup proj(S_2, i) \\
proj(F{*}, i) & = nil(F) \\
proj(F \times P, 1) & = F \\
proj(F \times P, i) & = proj(P, i-1)\\
proj(E{*}, i) & = nil(E) \\
proj(T \times E, 1) & = T \\
proj(T \times E, i) & = proj(E, i-1)
\end{align*}

\begin{align*}
infer(T_{1} \times ... \times T_{n}{*}, i) & = \left\{
\begin{array}{ll}
general(T_{i}) & \text{if $i < n$}\\
general(nil(T_{n})) & \text{if $i >= n$}
\end{array} \right.
\end{align*}

\begin{align*}
general(\False) & = \Boolean\\
general(\True) & = \Boolean\\
general({\it int}) & = \Integer\\
general({\it float}) & = \Number\\
general({\it string}) & = \String\\
general(F_{1} \cup F_{2}) & = general(F_{1}) \cup general(F_{2})\\
general(S_{1} \rightarrow S_{2}) & = general2(S_{1}) \rightarrow general2(S_{2})\\
general(\{F_{1}{:}V_{1}, ..., F_{n}{:}V_{n}\}_{tag}) & = \{F_{1}{:}general(V_{1}), ..., F_{n}{:}general(V_{n})\}_{tag}\\
general(\mu x.F) & = \mu x.general(F)\\
general(T) & = T
\end{align*}

\begin{align*}
general2(F{*}) & = general(F){*}\\
general2(F \times P) & = general(F) \times general2(P)\\
general2(S_{1} \sqcup S_{2}) & = general2(S_{1}) \sqcup general2(S_{2})
\end{align*}

\begin{align*}
closeall(\env[id_{1} \mapsto T_{1}, ..., id_{n} \mapsto T_{n}]) & = \env[id_{1} \mapsto close(T_{1}), ..., id_{n} \mapsto close(T_{n})]
\end{align*}

\begin{align*}
closeset(\env, \{id_{1}, ..., id_{n}\}) & = \env[id_{1} \mapsto close(\env(id_{1})), ..., id_{n} \mapsto close(\env(id_{n}))]
\end{align*}

\begin{align*}
openset(\env, \{id_{1}, ..., id_{n}\}) & = \env[id_{1} \mapsto open(\env(id_{1})), ..., id_{n} \mapsto open(\env(id_{n}))]
\end{align*}

\begin{align*}
merge(\overline{\env}) & = reduce(\overline{\env}, merge2)\\
merge2(\env_1,\env_2) & = \{\overline{(id,merget(\env_1(id),\env_2(id))}\}\\
merget(T_1, T_2) & = T_1 \;\;\; \mathrm{if\;} T_2 \lesssim T_1\\
merget(T_1, T_2) & = T_2 \;\;\; \mathrm{if\;} T_1 \lesssim T_2\\
& \;\;\;\;\; \mathrm{the\;next\;case\;applies\;if} \\
& \;\;\;\;\; \overline{V^l \lesssim_u V_r \,\vee\, V^r \lesssim_u V_l}\\
& \;\;\;\;\; \mathrm{and\;the\;right\; side\; is\;}wf\\
merget(\{\overline{F:V^l},\overline{F^{\prime}:V^\prime}\}_{unique},\\
\{\overline{F:V^r},\overline{F^{\prime\prime}:V^{\prime\prime}}\}_{unique}) & =
\{\overline{F:sup_u(V^l,V^r)},\\
& \;\;\;\;\;\; \overline{F^\prime:V^\prime},\\
& \;\;\;\;\;\; \overline{F^{\prime\prime}:V^{\prime\prime}}\}_{unique}\\
& \;\;\;\;\; \mathrm{the\;next\;case\;applies\;if} \\
& \;\;\;\;\; \overline{V^l \lesssim_c V_r \,\vee\, V^r \lesssim_c V_l}\\
& \;\;\;\;\; \mathrm{and\;the\;right\; side\; is\;}wf\\
merget(\{\overline{F:V^l},\overline{F^{\prime}:V^{\prime}}\}_{unique|open},\\
\{\overline{F:V^r},\overline{F^{\prime\prime}:V^{\prime\prime}}\}_{unique|open}) & =
\{\overline{F:sup_c(V^l,V^r)},\\
& \;\;\;\;\;\; \overline{F^\prime:V^\prime},\\
& \;\;\;\;\;\; \overline{F^{\prime\prime}:V^{\prime\prime}}\}_{open}\\
merget(T_1, T_2) & = \bot \;\;\; \mathrm{otherwise}
\end{align*}

\begin{align*}
sup_u(V_1, V_2) & = V_2 \;\;\; \mathrm{if}\; V_1 \lesssim_u V_2\\
sup_u(V_1, V_2) & = V_1 \;\;\; \mathrm{if}\; V_2 \lesssim_u V_1
\end{align*}

\begin{align*}
sup_c(V_1, V_2) & = V_2 \;\;\; \mathrm{if}\; V_1 \lesssim_c V_2\\
sup_c(V_1, V_2) & = V_1 \;\;\; \mathrm{if}\; V_2 \lesssim_c V_1
\end{align*}

\begin{align*}
join(\env_1,\env_2) & = \{\overline{(id,joint(\env_1(id),\env_2(id))}\}\\
joint(T_1, T_2) & = T_1 \;\;\; \mathrm{if\;} T_2 \lesssim T_1\\
joint(T_1, T_2) & = T_2 \;\;\; \mathrm{if\;} T_1 \lesssim T_2\\
& \;\;\;\;\; \mathrm{the\;next\;case\;applies\;if}\; \\
& \;\;\;\;\; \overline{V^l \lesssim_u V_r \,\vee\, V^r \lesssim_u V_l}\\
& \;\;\;\;\; \mathrm{and\;the\;right\; side\; is\;}wf\\
joint(\{\overline{F:V^l},\overline{F^{\prime}:V^{\prime}}\}_{unique},
\{\overline{F:V^r},\overline{F^{\prime\prime}:V^{\prime\prime}}\}_{unique}) & =
\{\overline{F:sup_u(V^l,V^r)},\\
& \;\;\;\;\;\; \overline{F^\prime:nil(V^\prime)},\\
& \;\;\;\;\;\; \overline{F^{\prime\prime}:nil(V^{\prime\prime})}\}_{unique}\\
joint(T_1, T_2) & = \bot \;\;\; \mathrm{otherwise}
\end{align*}

\begin{align*}
filter(F_{1} \cup F_{2}, F_{1}) & = filter(F_{2}, F_{1})\\
filter(F_{1} \cup F_{2}, F_{2}) & = filter(F_{1}, F_{2})\\
filter(F_{1} \cup F_{2}, F_{3}) & = filter(F_{1}, F_{3}) \cup filter(F_{2}, F_{3})\\
filter(F_{1}, F_{2}) & = F_{1}
\end{align*}

\begin{align*}
fopt(P_1 \sqcup P_2, F, i) & = P_2 & \mathrm{if} \; fot(proj(P_1, i),F) = {\bf void} \\
fopt(P_1 \sqcup P_2, F, i) & = P_1 & \mathrm{if} \; fot(proj(P_2, i),F) = {\bf void} \\
fopt(P_1 \sqcup P_2, F, i) & = P_1 \sqcup P_2 & \mathrm{otherwise}\\
fopt(P \sqcup S, F, i) & = fopt(S,F,i) & \mathrm{if} \; fot(proj(P, i),F) = {\bf void} \\
fopt(P \sqcup S, F, i) & = P \sqcup fopt(S,F,i) & \mathrm{otherwise}\\
fopt(S \sqcup P_2, F, i) & = fopt(S,F,i) & \mathrm{if} \; fot(proj(P, i),F) = {\bf void} \\
fopt(S \sqcup P, F, i) & = fopt(S,F,i) \sqcup P & \mathrm{otherwise}\\
fopt(S_1 \sqcup S_2, F, i) & = fopt(S_1,F,i) \sqcup fopt(S_2,F,i)
\end{align*}

\begin{align*}
fot(F_1 \cup F_2,F_3) & = fot(F_1,F_3) & \mathrm{if}\;fot(F_2,F_3) = {\bf void}\\
fot(F_1 \cup F_2,F_3) & = fot(F_2,F_3) & \mathrm{if}\;fot(F_1,F_3) = {\bf void}\\
fot(F_1 \cup F_2,F_3) & = fot(F_1,F_3) \cup fot(F_2,F_3) & \mathrm{otherwise}\\
fot(F_1,F_2) & = {\bf void} & \mathrm{if} \; F_1 \subtype F_2 \; \mathrm{and} \; F_2 \subtype F_1\\
fot(F_1,F_2) & = F_1 & \mathrm{otherwise}
\end{align*}

\begin{align*}
fipt(P_1 \sqcup P_2, F, i) & = P_2 & \mathrm{if} \; fit(proj(P_1, i),F) = {\bf void} \\
fipt(P_1 \sqcup P_2, F, i) & = P_1 & \mathrm{if} \; fit(proj(P_2, i),F) = {\bf void} \\
fipt(P_1 \sqcup P_2, F, i) & = P_1 \sqcup P_2 & \mathrm{otherwise}\\
fipt(P \sqcup S, F, i) & = fipt(S,F,i) & \mathrm{if} \; fit(proj(P, i),F) = {\bf void} \\
fipt(P \sqcup S, F, i) & = P \sqcup fipt(S,F,i) & \mathrm{otherwise}\\
fipt(S \sqcup P_2, F, i) & = fipt(S,F,i) & \mathrm{if} \; fit(proj(P, i),F) = {\bf void} \\
fipt(S \sqcup P, F, i) & = fipt(S,F,i) \sqcup P & \mathrm{otherwise}\\
fipt(S_1 \sqcup S_2, F, i) & = fipt(S_1,F,i) \sqcup fipt(S_2,F,i)
\end{align*}

\begin{align*}
fit(F_1 \cup F_2,F_3) & = fit(F_1,F_3) & \mathrm{if}\;fit(F_2,F_3) = {\bf void}\\
fit(F_1 \cup F_2,F_3) & = fit(F_2,F_3) & \mathrm{if}\;fit(F_1,F_3) = {\bf void}\\
fit(F_1 \cup F_2,F_3) & = fit(F_1,F_3) \cup fit(F_2,F_3) & \mathrm{otherwise}\\
fit(F_1,F_2) & = F_1 & \mathrm{if} \; F_1 \subtype F_2 \; \mathrm{and} \; F_2 \subtype F_1\\
fit(F_1,F_2) & = {\bf void} & \mathrm{otherwise}
\end{align*}




\end{document}
